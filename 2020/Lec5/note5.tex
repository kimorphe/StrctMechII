\documentclass[10pt,a4j]{jarticle}
%\usepackage{graphicx,wrapfig}
\usepackage{graphicx}
\usepackage{showkeys}
\setlength{\topmargin}{-1.5cm}
%\setlength{\textwidth}{16.5cm}
\setlength{\textheight}{25.2cm}
\newlength{\minitwocolumn}
\setlength{\minitwocolumn}{0.5\textwidth}
\addtolength{\minitwocolumn}{-\columnsep}
%\addtolength{\baselineskip}{-0.1\baselineskip}
%
\def\Mmaru#1{{\ooalign{\hfil#1\/\hfil\crcr
\raise.167ex\hbox{\mathhexbox 20D}}}}
%
\begin{document}
\newcommand{\fat}[1]{\mbox{\boldmath $#1$}}
\newcommand{\D}{\partial}
\newcommand{\w}{\omega}
\newcommand{\ga}{\alpha}
\newcommand{\gb}{\beta}
\newcommand{\gx}{\xi}
\newcommand{\gz}{\zeta}
\newcommand{\vhat}[1]{\hat{\fat{#1}}}
\newcommand{\spc}{\vspace{0.7\baselineskip}}
\newcommand{\halfspc}{\vspace{0.3\baselineskip}}
\bibliographystyle{unsrt}
%\pagestyle{empty}
\newcommand{\twofig}[2]
 {
   \begin{figure}
     \begin{minipage}[t]{\minitwocolumn}
         \begin{center}   #1
         \end{center}
     \end{minipage}
         \hspace{\columnsep}
     \begin{minipage}[t]{\minitwocolumn}
         \begin{center} #2
         \end{center}
     \end{minipage}
   \end{figure}
 }
%%%%%%%%%%%%%%%%%%%%%%%%%%%%%%%%%
%\vspace*{\baselineskip}
\begin{flushright}
	構造力学II\\
	2020/05/25
\end{flushright}
\begin{center}
	{\LARGE \bf 講義ノート 5} \\
\end{center}
\setcounter{section}{4}
%%%%%%%%%%%%%%%%%%%%%%%%%%%%%%%%%%%%%%%%%%%%%%%%%%%%%%%%%%%%%%%%
\section{骨組構造に対する仮想仕事式}
\subsection{部材番号と座標系}
$n$本の直線部材を連結してできる骨組み構造を考える.
図\ref{fig:fig5}-(a)は骨組み構造の一例を示したもので,部材数が$n=3$の場合である. 
以下ではこれを具体例として参照しながら,一般の骨組み構造に対して成立する仮想仕事式を導く.
なお,ここで対象とする骨組み構造は,部材間の連結は剛結でもピン結合でもよい.
また,一つの節点(部材の接合点)には何本の部材が連結されていてもよい.
各部材には部材番号が与えられており,一般の部材の番号を$e(=1,2,\dots n)$と書くことにする.
断面力や断面係数等,部材に関する諸量にはインデックス$e$をつけて表す.
例えば,部材$e$における軸力を$N_e$,曲げモーメントを$M_e$,部材長さを$l_e$等と書く.
なお,部材中の断面位置や荷重や変位などのベクトルを成分表記する際は,
図\ref{fig:fig5}-(b)に示す,部材軸方向に$x_e$軸,部材軸直角方向に$y_e$軸をとった
$x_ey_e$直交座標系を用いる.このような座標系は局所座標系,部材座標,あるいは要素座標系と呼ばれる.
\subsection{荷重条件}
各部材には分布荷重(ベクトル)が作用すると考える.
部材$e$に作用する,部材軸方向の分布力を$p_e(x_e)$,部材軸直角方向の分布力を$q_e(x_e)$と書き,
両者をまとめて,分布荷重ベクトルとして次のように表す.
\begin{equation}
	\fat{p}_e=\left( p_e(x_e),\, q_e(x_e) \right)^T, \ \ (e=1,\dots n)
	\label{eqn:}
\end{equation}
同様に,部材$e$の$x_e$における軸方向,軸直角方向の変位をそれぞれ$u_e(x_e),\, v_e(x_e)$とし,
変位ベクトルを
\begin{equation}
	\fat{u}_e=\left( p_e(x_e),\, q_e(x_e) \right)^T,  \ \ (e=1,\dots n)
	\label{eqn:}
\end{equation}
と表し,軸力$N_e$とせん断力$Q_e$の対として与えられる内力ベクトルを
\begin{equation}
	\fat{F}_e=\left( N_e(x_e),\, Q_e(x_e) \right)^T,  \ \ (e=1,\dots n)
	\label{eqn:}
\end{equation}
と書くことにする.なお,
骨組み構造に作用する全ての分布力や,変位分布,断面力分布を表す際には,
部材番号$e$をつけず
\begin{equation}
	\fat{p}=\left\{ \fat{p}_e \right\}_{e=1}^n, 
	\\
	\fat{u}=\left\{ \fat{u}_e \right\}_{e=1}^n, 
	\\
	\fat{F}=\left\{ \fat{F}_e \right\}_{e=1}^n,
	\label{eqn:}
\end{equation}
と表す.以下では,簡単のため,節点あるいは部材端部に外部から作用する集中荷重は
存在しないものと仮定しておく.
\subsection{仮想仕事式}
図\ref{fig:fig5}に示す骨組み構造について,荷重条件$\fat{p}$の異なる2つの系を考える.
これらの系を系1,系2と名付け,いずれの系に関する量であるかを区別する際には,右肩に
系の番号をつけて諸量を表すことにする.例えば,系1の部材$e$における変位ベクトルは
$\fat{u}^{(1)}_e$,系2の部材$s$における軸力と曲げモーメントを$N^{(2)}_e$および$M^{(2)}_e$等と表す.
以上の表記に従い,系1と系2の間で成立する仮想仕事式を部材$e$について導けば,
\begin{equation}
	\int_{x_e=0}^{l_e} 
	\left(
	\frac{N_e^{(1)}N_e^{(2)}}{EA_e}
	+
	\frac{M_e^{(1)}M_e^{(2)}}{EI_e}
	\right)
	dx_e
	=
	\left[ 
		\fat{F}_e^{(1)}\cdot \fat{u}_e^{(2)}
		-
		M_e^{(1)}
		(v_e^{(2)})'
	\right]_{x_e=0}^{l_e}
	+
	\int_{x_e=0}^{l_e} 
	\fat{p}_e^{(1)}\cdot \fat{u}_e^{(2)}
	dx_e
	\label{eqn:vwk_e}
\end{equation}
となる.この関係は単一の部材だけに関するものであることから,導出は単一部材の
曲げ−軸力問題と同様にして行うことができる.ただし,式(\ref{eqn:vwk_e})では,
部材端部($x_e=0,\,l_e$)で,変位やたわみが指定されていない場合を想定したものである.
図\ref{fig:fig5}-(a)に示した骨組み構造で言えば,$e=2$の場合は
式(\ref{eqn:vwk_e})はこのままの形で成立する.一方,$e=1$の場合は,
$x_e=0$において変位ベクトルとたわみ角がゼロであることから,
\begin{equation}
	\left. \fat{u}^{(1)}_1\right|_{x_1=0}
	=
	\left. \fat{u}^{(2)}_1\right|_{x_1=0}
	=\fat{0}, \ \ 
	\left. (v^{(1)}_1)'\right|_{x_1=0}
	=(
	\left.
	v^{(2)}_1)'\right|_{x_1=0}
	=0
	\label{eqn:fixed_end}
\end{equation}
であり,仮想仕事式は
\begin{equation}
	\int_{x_1=0}^{l_1} 
	\left(
	\frac{N_1^{(1)}N_1^{(2)}}{EA_1}
	+
	\frac{M_1^{(1)}M_1^{(2)}}{EI_1}
	\right)
	dx_1
	=
	\left. 
		\fat{F}_1^{(1)}\cdot \fat{u}_1^{(2)}
	\right|_{x_1=l_1}
		-
	\left. 
		M_1^{(1)}
		(v_1^{(2)})'
	\right|_{x_1=l_1}
	+
	\int_{x_1=0}^{l_1} 
	\fat{p}_1^{(1)}\cdot \fat{u}_1^{(2)}
	dx_1
	\label{eqn:}
\end{equation}
となる.また,部材$e=3$では$x_3=l_3$すなわちD点で断面力は零でなければならないので,
\begin{equation}
	\left. \fat{F}^{(1)}_3 \right|_{x_3=l_3}=\fat{0}, \ \ 
	\left. M_3^{(1)} \right|_{x_3=l_3}=0
	\label{eqn:free_end}
\end{equation}
とすることができる.
ここで,$e=1,\dot n(=3)$に対して,式(\ref{eqn:vwk_e})の辺々の和を取る.
その際,
\begin{eqnarray}
	a\left(
		\fat{u}^{(1)}
		,\,  
		\fat{u}^{(2)}
	\right) 
	&:=& 
	\sum_{e=1}^n 
	\int_{x_e=0}^{l_e} 
	\left(
	\frac{N_e^{(1)}N_e^{(2)}}{EA_e}
	+
	\frac{M_e^{(1)}M_e^{(2)}}{EI_e}
	\right)
	dx_e
	\label{eqn:def_ae}
	\\
%
	b\left( 
		 \fat{u}^{(2)}
	\right)
	&:=&
	\sum_{e=1}^n
	\left[ 
		\fat{F}_e^{(1)}\cdot \fat{u}_e^{(2)}
		-
		M_e^{(1)}
		(v_e^{(2)})'
	\right]_{x_e=0}^{l_e}
	+
	\sum_{e=1}^n 
	\int_{x_e=0}^{l_e} 
	\fat{p}_e^{(1)} \cdot \fat{u}_e^{(2)}dx_e
	\label{eqn:def_be}
\end{eqnarray}
とおけば,式(\ref{eqn:vwk_e})より,
骨組み構造に対する仮想仕事式が
\begin{equation}
	a\left(
		 \fat{u}^{(1)}
		,\,  
		\fat{u}^{(2)}
	\right) 
	=
	b\left( 
		\fat{u}^{(2)}
	\right)
	\label{eqn:VW_frame}
\end{equation}
と得られる.
ただし,式(\ref{eqn:def_be})右辺の第1項の和は,
式(\ref{eqn:fixed_end})および(\ref{eqn:free_end})と,
部材2両端の接合条件:
\begin{equation}
	\left. \fat{u}_1^{(2)} \right|_{x_1=l_1}
	=
	\left. \fat{u}_2^{(2)} \right|_{x_2=0}, \ \ 
	\left. \fat{u}_2^{(2)} \right|_{x_2=l_2}
	=
	\left. \fat{u}_3^{(2)} \right|_{x_3=0}
	\label{eqn:}
\end{equation}
\begin{equation}
	\left. (v_1^{(2)})' \right|_{x_1=l_1}
	=
	\left. (v_2^{(2)})' \right|_{x_2=0}, \ \ 
	\left. (v_2^{(2)})' \right|_{x_2=l_2}
	=
	\left. (v_3^{(2)})' \right|_{x_3=0}
	\label{eqn:}
\end{equation}
\begin{equation}
	\left. \fat{F}_1^{(1)} \right|_{x_1=l_1}
	+
	\left. \fat{F}_2^{(1)} \right|_{x_2=0}
	=\fat{0}	
	, \ \ 
	\left. \fat{F}_2^{(1)} \right|_{x_2=l_2}
	+
	\left. \fat{F}_3^{(1)} \right|_{x_3=0}
	=\fat{0}	
	\label{eqn:}
\end{equation}
\begin{equation}
	\left. M_1^{(2)} \right|_{x_1=l_1}
	+
	\left. M_2^{(2)} \right|_{x_2=0}=0
	, \ \ 
	\left. M_2^{(2)} \right|_{x_2=l_2}
	+
	\left. M_3^{(2)} \right|_{x_3=0}=0
	\label{eqn:}
\end{equation}
より,零となることが示される.
従って,式(\ref{eqn:def_be})は
\begin{equation}
	b\left( 
		 \fat{u}^{(2)}
	\right)
	=
	\sum_{e=1}^n 
	\int_{x_e=0}^{l_e} 
	\fat{p}_e^{(1)} \cdot \fat{u}_e^{(2)}dx_e
	\label{eqn:def_be2}
\end{equation}
と書き直すことができる.
\subsection{単位荷重法}
\subsubsection{一般の骨組み構造}
系1の荷重条件として,部材$e'$にのみ単位集中荷重が加えられた場合を考える.
すなわち,$e=1,\dots n$に対して
\begin{equation}
	\fat{p}_e^{(1)}=
	\left\{
	\begin{array}{cc}
	\hat{\fat{e}}\delta (x_{e'}-a) & (e=e') \\
	\fat{0} & (e\neq e')
	\end{array}
	\right.
	\label{eqn:unit_load}
\end{equation}
とする.
ただし,$\hat{\fat{e}}$は,集中荷重の方向を表す単位ベクトルを意味する.
式(\ref{eqn:unit_load})は,部材$e=e'$の$x_{e'}=a$において,
単位ベクトル$\hat{\fat{e}}$の方向に単位集中荷重のみが作用する場合を表す.
このとき,仮想仕事式(\ref{eqn:VW_frame})は
\begin{equation}
	\left.
	\hat{\fat{e}}\cdot \fat{u}^{(2)}
	\right|_{x_{e'}=a}
	=
	\sum_{e=1}^n 
	\int_{x_e=0}^{l_e} 
	\left(
	\frac{N_e^{(1)}N_e^{(2)}}{EA_e}
	+
	\frac{M_e^{(1)}M_e^{(2)}}{EI_e}
	\right)
	dx_e
	\label{eqn:uload_frame}
\end{equation}
となり,骨組み構造に対する単位荷重法の式が導かれる.
なお,図\ref{fig:fig6}は,$e'=2$とした場合の,系1の荷重条件を示したものである.
最後に,系1を補助系,系2を変位を求めるべき系とみる立場を明確にするために,
\begin{equation}
	\left(M^{(1)}_e,\, N_e^{(1)}\right)
	=
	\left(	\tilde M_e,\, \tilde N_e \right)
	\ \ (e=1,\dots n)
\end{equation}
\begin{equation}
	\left(M^{(2)}_e,\,N_e^{(2)}\right)
	=
	\left( M_e,\, N_e \right), \ \ 
	\fat{u}^{(2)}=\fat{u}
	\ \ (e=1,\dots n)
	\label{eqn:}
\end{equation}
と書き直せば,式(\ref{eqn:uload_frame})は次のようになる.
\begin{equation}
	\left.
	\hat{\fat{e}}\cdot \fat{u}
	\right|_{x_{e'}=a}
	=
	\sum_{e=1}^n 
	\int_{x_e=0}^{l_e} 
	\left(
	\frac{N_e \tilde N_e}{EA_e}
	+
	\frac{M_e \tilde M_e}{EI_e}
	\right)
	dx_e
	\label{eqn:uload_frame2}
\end{equation}
この結果は一般の骨組み構造に対して成立するものである.
\subsubsection{トラス構造}
トラス構造の場合,各部材には軸力のみが作用することから,曲げモーメントに関する項は現れない.
そのため,式(\ref{eqn:uload_frame2})において$M_e=0$とすることができ,単位荷重法の式は
\begin{equation}
	\left.
	\hat{\fat{e}}\cdot \fat{u}
	\right|_{x_{e'}=a}
	=
	\sum_{e=1}^n 
	\int_{x_e=0}^{l_e} 
	\frac{N_e \tilde N_e}{EA_e}
	dx_e
	\label{eqn:uload_truss0}
\end{equation}
また,トラスの節点にのみ外力が働く場合,部材内部で軸力は変化しない.さらに,
断面剛性$EA_e$がどの部材でも一定値のときには,式(\ref{eqn:uload_truss0})は
直ちに積分を行うことができ,単位荷重法の式は次のように非常に単純なものとなる.
\begin{equation}
	\left.
	\hat{\fat{e}}\cdot \fat{u}
	\right|_{x_{e'}=a}
	=
	\sum_{e=1}^n 
	\frac{N_e \tilde N_e}{EA_e}l_e
\end{equation}
\begin{figure}[h]
	\begin{center}
	\includegraphics[width=0.8\linewidth]{fig5.eps} 
	\end{center}
	\caption{(a)骨組み構造の例. 
	 (b)部材$e$に関する諸量と局所座標($x_e,\,y_e$).} 
	\label{fig:fig5}
\end{figure}
\begin{figure}
	\begin{center}
	\includegraphics[width=0.5\linewidth]{fig6.eps} 
	\end{center}
	\caption{単位集中荷重を受ける骨組み構造.} 
	\label{fig:fig6}
\end{figure}
\subsection{例題}
以下の例題では,部材の断面$A$,ヤング率$E$,断面2次モーメント$I$は,全ての部材と断面で一定とする.
\subsubsection{例題1}
\paragraph{問題:}
図\ref{fig:fig7}に示した3種類の骨組み構造について以下の問に答えよ.
\begin{enumerate}
\item
	骨組み構造(a),(b),(c)のそれぞれにおける曲げモーメント分布を求めよ.
\item
	骨組み構造(a)の点Cにおける鉛直変位$v_C$を求めよ.
\item
	骨組み構造(a)の点Cにおける水平変位$u_C$を求めよ.
\end{enumerate}
\begin{figure}
	\begin{center}
	\includegraphics[width=1.0\linewidth]{fig7.eps} 
	\end{center}
	\caption{荷重条件が異なる3種類の骨組み構造(a)-(c).}
	\label{fig:fig7}
\end{figure}
\paragraph{解答:}
\begin{enumerate}
\item
	\begin{enumerate}
	\item
		\begin{equation}
		N_1=-q_0l, \ \ M_1=-\frac{q_0l^2}{2}
		\label{eqn:}
		\end{equation}
		\begin{equation}
		N_2=0, \ \ M_2=-\frac{q_0s_2^2}{2}
		\label{eqn:}
		\end{equation}
	\item
		\begin{equation}
		\tilde N_1=-1, \ \ \tilde M_1=-l
		\label{eqn:}
		\end{equation}
		\begin{equation}
		\tilde N_2=0, \ \ \tilde M_2=-s_2
		\label{eqn:}
		\end{equation}
	\item
		\begin{equation}
			\tilde{\tilde N}_1=0, \ \ \tilde{\tilde M}_1=-s_1
		\label{eqn:}
		\end{equation}
		\begin{equation}
			\tilde{\tilde N}_2=1, \ \ \tilde{\tilde M}_2=0
		\label{eqn:}
		\end{equation}
	\end{enumerate}
\item
	\begin{equation}
		\int_0^l N_1\tilde N_1ds_1=q_0l^2, \ \ 
		\int_0^l N_2\tilde N_2ds_2=0
		\label{eqn:}
	\end{equation}
	\begin{equation}
		\int_0^l M_1\tilde M_1ds_1=\frac{1}{2}q_0l^4, \ \ 
		\int_0^l M_2\tilde M_2ds_2=\frac{1}{8}q_0l^4
		\label{eqn:}
	\end{equation}
	\begin{equation}
		v_C=
		\frac{q_0l^2}{EA}+ \frac{5}{8}\frac{q_0l^4}{EI}
		\label{eqn:}
	\end{equation}
\item
	\begin{equation}
		\int_0^l N_1\tilde{\tilde N}_1ds_1=0, \ \ 
		\int_0^l N_2\tilde{\tilde N}_2ds_2=0
		\label{eqn:}
	\end{equation}
	\begin{equation}
		\int_0^l M_1\tilde{\tilde M}_1ds_1=\frac{1}{4}q_0l^4, \ \ 
		\int_0^l M_2\tilde{\tilde M}_2ds_2=0
		\label{eqn:}
	\end{equation}
	\begin{equation}
		u_C=
		\frac{1}{4}\frac{q_0l^4}{EI}
		\label{eqn:}
	\end{equation}
\end{enumerate}
\begin{figure}[h]
	\begin{center}
	\includegraphics[width=0.9\linewidth]{fig11.eps} 
	\end{center}
	\caption{(a)部材座標$s_1,s_2$と構造の切断位置.(b)部材1,(c)部材2の自由物体図(例題1).} 
	\label{fig:fig11}
\end{figure}
\subsubsection{例題2}
\paragraph{問題:}
図\ref{fig:fig12}に示す骨組み構造の,点Bにおける水平変位$u_B$を求めよ.
\begin{figure}[h]
	\begin{center}
	\includegraphics[width=0.4\linewidth]{fig12.eps} 
	\end{center}
	\caption{大きさ$F$の水平荷重を受ける単純支持された骨組み構造.} 
	\label{fig:fig12}
\end{figure}
\paragraph{解答:}
\begin{equation}
	R_A=-\frac{F}{2}, \ \ H_A=-F, \ \ R_C=\frac{F}{2}
	\label{eqn:}
\end{equation}
\begin{equation}
	N_1=\frac{3F}{2\sqrt{2}}, \ \ M_1=\frac{Fx_1}{2\sqrt{2}}
	\label{eqn:}
\end{equation}
\begin{equation}
	N_2=-\frac{F}{2\sqrt{2}}, \ \ M_2=\frac{Fx_2}{2\sqrt{2}}
	\label{eqn:}
\end{equation}
\begin{equation}
	\int_0^l N_1\tilde N_1dx_1=\frac{9}{8}Fl, \ \ 
	\int_0^l N_2\tilde N_2ds_2=\frac{1}{8}Fl
	\label{eqn:}
\end{equation}
\begin{equation}
	\int_0^l M_1\tilde M_1dx_1=\frac{1}{24}Fl^3, \ \ 
	\int_0^l M_2\tilde M_2ds_2=\frac{1}{24}Fl^3
	\label{eqn:}
\end{equation}
\begin{equation}
	u_B=\frac{5}{4}\frac{Fl}{EA}
	+
	\frac{1}{12}\frac{Fl^3}{EA}
	\label{eqn:}
\end{equation}
\begin{figure}[h]
	\begin{center}
	\includegraphics[width=0.8\linewidth]{fig13.eps} 
	\end{center}
	\caption{(a)部材座標$s_1,s_2$と構造の切断位置.(b)部材1,(c)部材2の自由物体図(例題2).} 
	\label{fig:fig13}
\end{figure}
\subsubsection{例題3}
\paragraph{問題:}
図\ref{fig:fig8}に示す2つのトラス構造について次の問に答えよ.
\begin{enumerate}
\item
	トラス構造(a)の各部材に作用する軸力を求めよ.
\item
	トラス構造(b)の各部材に作用する軸力を求めよ.
\item
	トラス構造(a)の点Cにおける水平変位$u_C$を求めよ.
\item
	トラス構造(a)の点Cにおける鉛直変位$v_C$を求めよ.
\end{enumerate}
\begin{figure}
	\begin{center}
	\includegraphics[width=0.6\linewidth]{fig8.eps} 
	\end{center}
	\caption{(a)節点Cで水平力を受けるトラス.(b)節点Cで鉛直力を受けるトラス.}
	\label{fig:fig8}
\end{figure}
\paragraph{解答:}
\begin{enumerate}
\item
\begin{equation}
	N_1=\frac{F}{2}, \ \ N_2=F, \ \ N_3=-F 
	\label{eqn:}
\end{equation}
\item
\begin{equation}
	N_1'=\frac{F}{2\sqrt{3}}, \ \ N_2'=-\frac{F}{\sqrt{3}}, \ \ N_3'=- \frac{F}{\sqrt{3}}
	\label{eqn:}
\end{equation}
\item
\begin{equation}
	u_C=\sum_{i=1}^3 \frac{N_i\tilde N_i}{EA}l=\frac{9}{4}\frac{Fl}{EA}
\end{equation}
\item
\begin{equation}
	v_C=\sum_{i=1}^3 \frac{N_i\tilde N_i}{EA}l=\frac{1}{4\sqrt{3}}\frac{Fl}{EA}
\end{equation}
\end{enumerate}
\subsection{練習問題}
以下の問題では,部材の断面$A$,ヤング率$E$,断面2次モーメント$I$は,全ての部材と断面で一定とする.
\subsubsection{練習問題1}
図\ref{fig:fig9}に示す骨組み構造について以下の問に答えよ. 
\begin{enumerate}
\item
	点Cにおける鉛直変位$v_C$を求めよ.
\item
	点Cにおける水平変位$u_C$を求めよ.
\end{enumerate}
\begin{figure}
	\begin{center}
	\includegraphics[width=0.4\linewidth]{fig9.eps} 
	\end{center}
	\caption{水平部材に等分布荷重を受ける骨組み構造.} 
	\label{fig:fig9}
\end{figure}
\subsubsection{練習問題2}
図\ref{fig:fig10}に示すトラスについて以下の問に答えよ. 
\begin{enumerate}
\item
	点Dにおける水平変位$u_D$を求めよ.
\item
	点Eにおける水平変位$u_E$を求めよ.
\end{enumerate}
\begin{figure}
	\begin{center}
	\includegraphics[width=0.5\linewidth]{fig10.eps} 
	\end{center}
	\caption{単純支持されたトラス.} 
	\label{fig:fig10}
\end{figure}
\end{document}

