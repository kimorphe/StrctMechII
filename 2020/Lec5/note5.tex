\documentclass[10pt,a4j]{jarticle}
%\usepackage{graphicx,wrapfig}
\usepackage{graphicx}
%\usepackage{showkeys}
\setlength{\topmargin}{-1.5cm}
%\setlength{\textwidth}{16.5cm}
\setlength{\textheight}{25.2cm}
\newlength{\minitwocolumn}
\setlength{\minitwocolumn}{0.5\textwidth}
\addtolength{\minitwocolumn}{-\columnsep}
%\addtolength{\baselineskip}{-0.1\baselineskip}
%
\def\Mmaru#1{{\ooalign{\hfil#1\/\hfil\crcr
\raise.167ex\hbox{\mathhexbox 20D}}}}
%
\begin{document}
\newcommand{\fat}[1]{\mbox{\boldmath $#1$}}
\newcommand{\D}{\partial}
\newcommand{\w}{\omega}
\newcommand{\ga}{\alpha}
\newcommand{\gb}{\beta}
\newcommand{\gx}{\xi}
\newcommand{\gz}{\zeta}
\newcommand{\vhat}[1]{\hat{\fat{#1}}}
\newcommand{\spc}{\vspace{0.7\baselineskip}}
\newcommand{\halfspc}{\vspace{0.3\baselineskip}}
\bibliographystyle{unsrt}
%\pagestyle{empty}
\newcommand{\twofig}[2]
 {
   \begin{figure}
     \begin{minipage}[t]{\minitwocolumn}
         \begin{center}   #1
         \end{center}
     \end{minipage}
         \hspace{\columnsep}
     \begin{minipage}[t]{\minitwocolumn}
         \begin{center} #2
         \end{center}
     \end{minipage}
   \end{figure}
 }
%%%%%%%%%%%%%%%%%%%%%%%%%%%%%%%%%
%\vspace*{\baselineskip}
\begin{flushright}
	構造力学II\\
	2020/05/25
\end{flushright}
\begin{center}
	{\LARGE \bf 講義ノート 5} \\
\end{center}
\setcounter{section}{4}
%%%%%%%%%%%%%%%%%%%%%%%%%%%%%%%%%%%%%%%%%%%%%%%%%%%%%%%%%%%%%%%%
\section{骨組構造に対する仮想仕事式}
\subsection{部材番号と座標系}
$n$本の直線部材を連結してできる骨組み構造を考える.
図\ref{fig:fig4}-(a)はそのような骨組み構造の一例を示したもので,
部材数$n$が3の場合である. 
以下ではこれを具体例として参照しながら,一般の骨組み構造に対して成立する仮想仕事式を導く.
なお,ここで対象とする骨組み構造は,部材間の連結が剛結でもピン結合でもよい.
また,一つの節点(部材の接合点)には何本の部材が連結されていてもよい.
各部材には部材番号が与えられており,一般の部材の番号を$e(=1,2,\dots n)$と書くことにする.
断面力や断面係数等,部材に関する諸量にはインデックス$e$をつけて表す.
例えば,部材$e$における軸力を$N_e$,曲げモーメントを$M_e$,部材長さを$l_e$等と書く.
なお,部材中の断面位置や荷重や変位などのベクトル成分を示す際は,
図\ref{fig:fig5}-(b)に示すように,部材軸方向に$x_e$軸,部材軸直角方向に$y_e$軸をとった
$x_ey_e-$直交座標系を用いる.このような座標系は局所座標系,部材座標,あるいは要素座標系と呼ばれる.
\begin{figure}[h]
	\begin{center}
	\includegraphics[width=0.8\linewidth]{fig4.eps} 
	\end{center}
	\caption{(a)骨組み構造の例. 
	 (b)部材$e$に関する諸量と局所座標($x_e,\,y_e$).} 
	\label{fig:fig4}
\end{figure}
\subsection{荷重条件}
各部材には分布荷重(ベクトル)が作用すると考える.
部材$e$に作用する,部材軸方向の分布力を$p_e(x_e)$,部材軸直角方向の分布力を$q_e(x_e)$と書き,
両者をまとめ,分布荷重ベクトル$\fat{p}_e$として次のように表す.
\begin{equation}
	\fat{p}_e=\left( p_e(x_e),\, q_e(x_e) \right)^T, \ \ (e=1,\dots n)
	\label{eqn:}
\end{equation}
ここに$(\cdot)^T$は$(\cdot)$の転置を意味する.
同様に,部材$e$の$x_e$における軸方向,軸直角方向の変位をそれぞれ$u_e(x_e),\, v_e(x_e)$とし,
変位ベクトルを
\begin{equation}
	\fat{u}_e=\left( u_e(x_e),\, v_e(x_e) \right)^T,  \ \ (e=1,\dots n)
	\label{eqn:}
\end{equation}
と表し,軸力$N_e$とせん断力$Q_e$の対として与えられる内力ベクトルを
\begin{equation}
	\fat{F}_e=\left( N_e(x_e),\, Q_e(x_e) \right)^T,  \ \ (e=1,\dots n)
	\label{eqn:}
\end{equation}
と書くことにする.なお,
骨組み構造に作用する分布力や変位分布,断面力分布の全体を表す際には,
部材番号$e$をつけず
\begin{equation}
	\fat{p}=\left\{ \fat{p}_e \right\}_{e=1}^n, 
	\\
	\fat{u}=\left\{ \fat{u}_e \right\}_{e=1}^n, 
	\\
	\fat{F}=\left\{ \fat{F}_e \right\}_{e=1}^n,
	\label{eqn:}
\end{equation}
と書く.以下では,簡単のため,節点あるいは部材端部に外部から作用する集中荷重は
存在しないと仮定する.
\subsection{仮想仕事式}
図\ref{fig:fig5}に示すような荷重条件$\fat{p}$の異なる2つの系を考える.
これらの系を系1,系2と名付け,いずれの系に関する量であるかを区別する際には,
右肩に系の番号をつけて諸量を表すことにする.例えば,系1の部材$e$における変位ベクトルは
$\fat{u}^{(1)}_e$,系2の部材$i$における軸力と曲げモーメントを$N^{(2)}_i$および$M^{(2)}_i$等と表す.
\begin{figure}[h]
	\begin{center}
	\includegraphics[width=0.9\linewidth]{fig5.eps} 
	\end{center}
	\caption{仮想仕事式が対象とする2つの骨組み構造.} 
	\label{fig:fig5}
\end{figure}
以上の表記に従い,系1と系2の間で成立する仮想仕事式を部材$e$について導けば,
\begin{equation}
	\int_{x_e=0}^{l_e} 
	\left(
	\frac{N_e^{(1)}N_e^{(2)}}{EA_e}
	+
	\frac{M_e^{(1)}M_e^{(2)}}{EI_e}
	\right)
	dx_e
	=
	\left[ 
		\fat{F}_e^{(1)}\cdot \fat{u}_e^{(2)}
		-
		M_e^{(1)}
		(v_e^{(2)})'
	\right]_{x_e=0}^{l_e}
	+
	\int_{x_e=0}^{l_e} 
	\fat{p}_e^{(1)}\cdot \fat{u}_e^{(2)}
	dx_e
	\label{eqn:vwk_e}
\end{equation}
となる.この関係は単一の部材に関するものであることから,導出は単一部材の
曲げ−軸力問題と全く同様である,式(\ref{eqn:vwk_e})では,
部材端部($x_e=0,\,l_e$)で,変位が未知の場合を想定したものである.
図\ref{fig:fig5}-(a)に示した骨組み構造で言えば,$e=2$の場合,
式(\ref{eqn:vwk_e})はこのままの形で成立する.一方,$e=1$の場合,
$x_e=0$において変位ベクトルとたわみ角がゼロであることから,
\begin{equation}
	\left. \fat{u}^{(1)}_1\right|_{x_1=0}
	=
	\left. \fat{u}^{(2)}_1\right|_{x_1=0}
	=\fat{0}, \ \ 
	\left. (v^{(1)}_1)'\right|_{x_1=0}
	=(
	\left.
	v^{(2)}_1)'\right|_{x_1=0}
	=0
	\label{eqn:fixed_end}
\end{equation}
であり,仮想仕事式は
\begin{equation}
	\int_{x_1=0}^{l_1} 
	\left(
	\frac{N_1^{(1)}N_1^{(2)}}{EA_1}
	+
	\frac{M_1^{(1)}M_1^{(2)}}{EI_1}
	\right)
	dx_1
	=
	\left. 
		\fat{F}_1^{(1)}\cdot \fat{u}_1^{(2)}
	\right|_{x_1=l_1}
		-
	\left. 
		M_1^{(1)}
		(v_1^{(2)})'
	\right|_{x_1=l_1}
	+
	\int_{x_1=0}^{l_1} 
	\fat{p}_1^{(1)}\cdot \fat{u}_1^{(2)}
	dx_1
	\label{eqn:}
\end{equation}
となる.また,部材3($e=3$)では,D点すなわち$x_3=l_3$で断面力は零でなければならないので,
\begin{equation}
	\left. \fat{F}^{(1)}_3 \right|_{x_3=l_3}=\fat{0}, \ \ 
	\left. M_3^{(1)} \right|_{x_3=l_3}=0
	\label{eqn:free_end}
\end{equation}
とすることができる.
ここで,$e=1,\dots n(=3)$に対して,式(\ref{eqn:vwk_e})の辺々の和を取る.
その際,
\begin{eqnarray}
	a\left(
		\fat{u}^{(1)}
		,\,  
		\fat{u}^{(2)}
	\right) 
	&:=& 
	\sum_{e=1}^n 
	\int_{x_e=0}^{l_e} 
	\left(
	\frac{N_e^{(1)}N_e^{(2)}}{EA_e}
	+
	\frac{M_e^{(1)}M_e^{(2)}}{EI_e}
	\right)
	dx_e
	\label{eqn:def_ae}
	\\
%
	b\left( 
		 \fat{u}^{(2)}
	\right)
	&:=&
	\sum_{e=1}^n
	\left[ 
		\fat{F}_e^{(1)}\cdot \fat{u}_e^{(2)}
		-
		M_e^{(1)}
		(v_e^{(2)})'
	\right]_{x_e=0}^{l_e}
	+
	\sum_{e=1}^n 
	\int_{x_e=0}^{l_e} 
	\fat{p}_e^{(1)} \cdot \fat{u}_e^{(2)}dx_e
	\label{eqn:def_be}
\end{eqnarray}
とおけば,式(\ref{eqn:vwk_e})より,
骨組み構造に対する仮想仕事式が
\begin{equation}
	a\left(
		 \fat{u}^{(1)}
		,\,  
		\fat{u}^{(2)}
	\right) 
	=
	b\left( 
		\fat{u}^{(2)}
	\right)
	\label{eqn:VW_frame}
\end{equation}
と得られる.
ただし,式(\ref{eqn:def_be})の右辺第1項の和は,
式(\ref{eqn:fixed_end})および(\ref{eqn:free_end})に加え,
部材2両端の接合条件:
\begin{equation}
	\left. \fat{u}_1^{(2)} \right|_{x_1=l_1}
	=
	\left. \fat{u}_2^{(2)} \right|_{x_2=0}, \ \ 
	\left. \fat{u}_2^{(2)} \right|_{x_2=l_2}
	=
	\left. \fat{u}_3^{(2)} \right|_{x_3=0}
	\label{eqn:}
\end{equation}
\begin{equation}
	\left. (v_1^{(2)})' \right|_{x_1=l_1}
	=
	\left. (v_2^{(2)})' \right|_{x_2=0}, \ \ 
	\left. (v_2^{(2)})' \right|_{x_2=l_2}
	=
	\left. (v_3^{(2)})' \right|_{x_3=0}
	\label{eqn:}
\end{equation}
\begin{equation}
	\left. \fat{F}_1^{(1)} \right|_{x_1=l_1}
	+
	\left. \fat{F}_2^{(1)} \right|_{x_2=0}
	=\fat{0}	
	, \ \ 
	\left. \fat{F}_2^{(1)} \right|_{x_2=l_2}
	+
	\left. \fat{F}_3^{(1)} \right|_{x_3=0}
	=\fat{0}	
	\label{eqn:}
\end{equation}
\begin{equation}
	\left. M_1^{(1)} \right|_{x_1=l_1}
	+
	\left. M_2^{(1)} \right|_{x_2=0}=0
	, \ \ 
	\left. M_2^{(1)} \right|_{x_2=l_2}
	+
	\left. M_3^{(1)} \right|_{x_3=0}=0
	\label{eqn:}
\end{equation}
より,
\begin{equation}
	\sum_{e=1}^n
	\left[ 
		\fat{F}_e^{(1)}\cdot \fat{u}_e^{(2)}
		-
		M_e^{(1)}
		(v_e^{(2)})'
	\right]_{x_e=0}^{l_e}
	=0
\end{equation}
となることが示される.
従って,式(\ref{eqn:def_be})は
\begin{equation}
	b\left( 
		 \fat{u}^{(2)}
	\right)
	=
	\sum_{e=1}^n 
	\int_{x_e=0}^{l_e} 
	\fat{p}_e^{(1)} \cdot \fat{u}_e^{(2)}dx_e
	\label{eqn:def_be2}
\end{equation}
と書き直すことができる.
\subsection{単位荷重法}
\subsubsection{一般の骨組み構造}
系1の荷重条件として,部材$e'$にのみ単位集中荷重が加えられた場合を考える.
すなわち,$e=1,\dots n$に対して
\begin{equation}
	\fat{p}_e^{(1)}=
	\left\{
	\begin{array}{cc}
	\hat{\fat{e}}\delta (x_{e'}-a) & (e=e') \\
	\fat{0} & (e\neq e')
	\end{array}
	\right.
	\label{eqn:unit_load}
\end{equation}
とする.
ただし,$\hat{\fat{e}}$は,集中荷重の方向を表す単位ベクトルを意味する.
式(\ref{eqn:unit_load})は,部材$e=e'$の$x_{e'}=a$において,
単位ベクトル$\hat{\fat{e}}$の方向に単位集中荷重のみが作用する場合を表す.
このとき,仮想仕事式(\ref{eqn:VW_frame})は
\begin{equation}
	\left.
	\hat{\fat{e}}\cdot \fat{u}^{(2)}
	\right|_{x_{e'}=a}
	=
	\sum_{e=1}^n 
	\int_{x_e=0}^{l_e} 
	\left(
	\frac{N_e^{(1)}N_e^{(2)}}{EA_e}
	+
	\frac{M_e^{(1)}M_e^{(2)}}{EI_e}
	\right)
	dx_e
	\label{eqn:uload_frame}
\end{equation}
となり,骨組み構造に対する単位荷重法の式が導かれる.
なお,図\ref{fig:fig6}は,$e'=2$とした場合の,系1の荷重条件を示したものである.
最後に,系1を補助系,系2を変位を求めるべき系とみる立場を明確にするために,
\begin{equation}
	\left(M^{(1)}_e,\, N_e^{(1)}\right)
	=
	\left(	\tilde M_e,\, \tilde N_e \right)
	\ \ (e=1,\dots n)
\end{equation}
\begin{equation}
	\left(M^{(2)}_e,\,N_e^{(2)}\right)
	=
	\left( M_e,\, N_e \right), \ \ 
	\fat{u}^{(2)}=\fat{u}
	\ \ (e=1,\dots n)
	\label{eqn:}
\end{equation}
と書き直せば,式(\ref{eqn:uload_frame})は次のようになる.
\begin{equation}
	\left.
	\hat{\fat{e}}\cdot \fat{u}
	\right|_{x_{e'}=a}
	=
	\sum_{e=1}^n 
	\int_{x_e=0}^{l_e} 
	\left(
	\frac{N_e \tilde N_e}{EA_e}
	+
	\frac{M_e \tilde M_e}{EI_e}
	\right)
	dx_e
	\label{eqn:uload_frame2}
\end{equation}
この結果は一般の骨組み構造に対して成立するものである.
\subsubsection{トラス構造}
トラス構造の場合,各部材には軸力のみが作用することから,曲げモーメントに関する項は現れない.
そのため,式(\ref{eqn:uload_frame2})において$M_e=\tilde M_e=0$とすることができ,単位荷重法の式は
\begin{equation}
	\left.
	\hat{\fat{e}}\cdot \fat{u}
	\right|_{x_{e'}=a}
	=
	\sum_{e=1}^n 
	\int_{x_e=0}^{l_e} 
	\frac{N_e \tilde N_e}{EA_e}
	dx_e
	\label{eqn:uload_truss0}
\end{equation}
となる.また,トラスの節点にのみ外力が働く場合に限定すると,部材内部で軸力は変化しない.
さらに, 断面剛性$EA_e$がどの部材でも一定値のときには,式(\ref{eqn:uload_truss0})
の積分を行うことができ,単位荷重法の式は次のように非常に単純なものとなる.
\begin{equation}
	\left.
	\hat{\fat{e}}\cdot \fat{u}
	\right|_{x_{e'}=a}
	=
	\sum_{e=1}^n 
	\frac{N_e \tilde N_e}{EA_e}l_e
	\label{eqn:uload_truss}
\end{equation}
\begin{figure}
	\begin{center}
	\includegraphics[width=0.5\linewidth]{fig6.eps} 
	\end{center}
	\caption{単位集中荷重を受ける骨組み構造.} 
	\label{fig:fig6}
\end{figure}
\subsection{例題}
以下の例題では,部材の断面$A$,ヤング率$E$,断面2次モーメント$I$は,全ての部材と断面で一定とする.
\subsubsection{例題1}
\paragraph{問題:}
図\ref{fig:fig7}に示した3種類の骨組み構造について以下の問に答えよ.
\begin{enumerate}
\item
	骨組み構造(a),(b),(c)のそれぞれにおける曲げモーメント分布を求めよ.
\item
	骨組み構造(a)の点Cにおける鉛直変位$v_C$を求めよ.
\item
	骨組み構造(a)の点Cにおける水平変位$u_C$を求めよ.
\end{enumerate}
\begin{figure}
	\begin{center}
	\includegraphics[width=1.0\linewidth]{fig7.eps} 
	\end{center}
	\caption{荷重条件が異なる3種類の骨組み構造(a)-(c).}
	\label{fig:fig7}
\end{figure}
\paragraph{解答:}
\begin{enumerate}
\item
	断面力分布を表すための座標$s_1$と$s_2$を,図\ref{fig:fig11}-(a)
	に示すように取る.また,$a-a'$断面,$b-b'$断面に作用する
	断面力の正方向を図\ref{fig:fig11}-(b)と(c)のように定める.
	\begin{enumerate}
	\item
		図\ref{fig:fig11}-(b)と(c)に示した自由物体図を参照して
		釣り合い条件式を立てれば,各々の部材における軸力と
		曲げモーメントが次のように求められる.
		\begin{equation}
		N_1=-q_0l, \ \ M_1=-\frac{q_0l^2}{2}
		\label{eqn:}
		\end{equation}
		\begin{equation}
		N_2=0, \ \ M_2=-\frac{q_0s_2^2}{2}
		\label{eqn:}
		\end{equation}
	\item
		部材$i\,(=1,2)$の軸力と曲げモーメントを$\tilde N_i, \tilde M_i$と表す.
		これらは,上の問題と同様,釣り合い条件式から
		\begin{equation}
		\tilde N_1=-1, \ \ \tilde M_1=-l
		\label{eqn:}
		\end{equation}
		\begin{equation}
		\tilde N_2=0, \ \ \tilde M_2=-s_2
		\label{eqn:}
		\end{equation}
		と求められる.	
	\item
		部材$i\, (=1,2)$の軸力と曲げモーメントを$\tilde {\tilde N}_i, \tilde{\tilde M}_i$
		とすれば,これらは
		\begin{equation}
			\tilde{\tilde N}_1=0, \ \ \tilde{\tilde M}_1=-s_1
		\label{eqn:}
		\end{equation}
		\begin{equation}
			\tilde{\tilde N}_2=1, \ \ \tilde{\tilde M}_2=0
		\label{eqn:}
		\end{equation}
		となる.
	\end{enumerate}
\item
	図\ref{fig:fig7}-(b)を補助系として単位荷重法(\ref{eqn:uload_frame2})
	を適用すればよい.
	\begin{equation}
		\int_0^l N_1\tilde N_1ds_1=q_0l^2, \ \ 
		\int_0^l N_2\tilde N_2ds_2=0
		\label{eqn:}
	\end{equation}
	\begin{equation}
		\int_0^l M_1\tilde M_1ds_1=\frac{1}{2}q_0l^4, \ \ 
		\int_0^l M_2\tilde M_2ds_2=\frac{1}{8}q_0l^4
		\label{eqn:}
	\end{equation}
	だから,
	\begin{equation}
		v_C=
		\frac{q_0l^2}{EA}+ \frac{5}{8}\frac{q_0l^4}{EI}
		\label{eqn:}
	\end{equation}
	となる.
\item
	図\ref{fig:fig7}-(c)を補助系として単位荷重法(\ref{eqn:uload_frame2})
	を適用すれば,
	\begin{equation}
		\int_0^l N_1\tilde{\tilde N}_1ds_1=0, \ \ 
		\int_0^l N_2\tilde{\tilde N}_2ds_2=0
		\label{eqn:}
	\end{equation}
	\begin{equation}
		\int_0^l M_1\tilde{\tilde M}_1ds_1=\frac{1}{4}q_0l^4, \ \ 
		\int_0^l M_2\tilde{\tilde M}_2ds_2=0
		\label{eqn:}
	\end{equation}
	より,
	\begin{equation}
		u_C=
		\frac{1}{4}\frac{q_0l^4}{EI}
		\label{eqn:}
	\end{equation}
	を得る.
\end{enumerate}
\begin{figure}[h]
	\begin{center}
	\includegraphics[width=0.9\linewidth]{fig11.eps} 
	\end{center}
	\caption{(a)部材座標$s_1,s_2$と構造の切断位置.(b)部材1,(c)部材2の自由物体図(例題1).} 
	\label{fig:fig11}
\end{figure}
\subsubsection{例題2}
\paragraph{問題:}
図\ref{fig:fig12}に示す骨組み構造の,点Bにおける水平変位$u_B$を求めよ.
\begin{figure}[h]
	\begin{center}
	\includegraphics[width=0.4\linewidth]{fig12.eps} 
	\end{center}
	\caption{大きさ$F$の水平荷重を受ける単純支持された骨組み構造.} 
	\label{fig:fig12}
\end{figure}
\paragraph{解答:}
図\ref{fig:fig13}-(a)の自由物体図を参照して,構造全体の釣り合い条件を
たてれば,支点反力が次のように求められる.
\begin{equation}
	R_A=-\frac{F}{2}, \ \ H_A=-F, \ \ R_C=\frac{F}{2}
	\label{eqn:}
\end{equation}
ここで,図\ref{fig:fig13}-(a)の$a-a'$と$b-b'$における断面力の
正方向を,図\ref{fig:fig13}-(b)と(c)のように定める.
これらの自由物体図をもとに,軸方向の力と,切断面に関するモーメントの
釣り合い条件を考えることで,以下のように断面力を決定することができる.
\begin{equation}
	N_1=\frac{3F}{2\sqrt{2}}, \ \ M_1=\frac{Fx_1}{2\sqrt{2}}
	\label{eqn:Nx_hat}
\end{equation}
\begin{equation}
	N_2=-\frac{F}{2\sqrt{2}}, \ \ M_2=\frac{Fs_2}{2\sqrt{2}}
	\label{eqn:Mx_hat}
\end{equation}
点Bの水平変位を単位荷重法で求める場合,図\ref{fig:fig12}の構造に:
おいて$F=1$としたものをの補助系として用いればよい.その軸力
$\tilde N_i$と曲げモーメント$\tilde M_i,\,(i=1,2)$は,
式(\ref{eqn:Nx_hat})と式(\ref{eqn:Mx_hat})において$F=1$とすることで
得られる.その結果
\begin{equation}
	\int_0^l N_1\tilde N_1dx_1=\frac{9}{8}Fl, \ \ 
	\int_0^l N_2\tilde N_2ds_2=\frac{1}{8}Fl
	\label{eqn:}
\end{equation}
\begin{equation}
	\int_0^l M_1\tilde M_1dx_1=\frac{1}{24}Fl^3, \ \ 
	\int_0^l M_2\tilde M_2ds_2=\frac{1}{24}Fl^3
	\label{eqn:}
\end{equation}
となり,これらを式(\ref{eqn:uload_frame2})に代入して
\begin{equation}
	u_B=\frac{5}{4}\frac{Fl}{EA}
	+
	\frac{1}{12}\frac{Fl^3}{EA}
	\label{eqn:}
\end{equation}
と,水平変位が求められる.
\begin{figure}[h]
	\begin{center}
	\includegraphics[width=0.9\linewidth]{fig13.eps} 
	\end{center}
	\caption{(a)支点反力の正方向と部材の切断位置.(b)部材1,(c)部材2の自由物体図(例題2).} 
	\label{fig:fig13}
\end{figure}
\subsubsection{例題3}
\paragraph{問題:}
図\ref{fig:fig8}に示す2つのトラスについて次の問に答えよ.
\begin{enumerate}
\item
	トラス(a)の各部材に作用する軸力を求めよ.
\item
	トラス(b)の各部材に作用する軸力を求めよ.
\item
	トラス(a)の点Cにおける水平変位$u_C$を求めよ.
\item
	トラス(a)の点Cにおける鉛直変位$v_C$を求めよ.
\end{enumerate}
\begin{figure}
	\begin{center}
	\includegraphics[width=0.6\linewidth]{fig8.eps} 
	\end{center}
	\caption{(a)節点Cで水平力を受けるトラス.(b)節点Cで鉛直力を受けるトラス.}
	\label{fig:fig8}
\end{figure}
\paragraph{解答:}
\begin{enumerate}
\item
	$i$番目の部材の軸力を$N_i\,(i=1,2,3)$とする.
	支点反力を求めた後,節点Aと節点Bにおける力の
	釣り合いを考えれば,軸力が次のように求められる.
	\begin{equation}
		N_1=\frac{F}{2}, \ \ N_2=F, \ \ N_3=-F 
		\label{eqn:}
	\end{equation}
\item
	$i$番目の部材の軸力を$N_i'\,(i=1,2,3)$とする.
	支点反力を求めた後,節点Aと節点Bにおける力の
	釣り合いを考えれば,軸力が次のように求められる.
	\begin{equation}
	N_1'=\frac{F}{2\sqrt{3}}, \ \ N_2'=-\frac{F}{\sqrt{3}}, \ \ N_3'=- \frac{F}{\sqrt{3}}
	\label{eqn:}
	\end{equation}
\item
	トラス(a)において$F=1$としたものを補助系として単位荷重法
	を適用すればよい.すなわち,$\tilde N_i=\left. N_i\right|_{F=1}$として
	式(\ref{eqn:uload_truss})に代入すれば,
	\begin{equation}
		u_C=\sum_{i=1}^3 \frac{N_i\tilde N_i}{EA}l=\frac{9}{4}\frac{Fl}{EA}
	\end{equation}
	となる.
\item
	トラス(b)において$F=1$としたものを補助系として単位荷重法
	を適用すればよい.すなわち,$\tilde N_i'=\left. N_i\right|_{F=1}$として
	式(\ref{eqn:uload_truss})に代入すれば,
	\begin{equation}
		v_C=\sum_{i=1}^3 \frac{N_i\tilde N_i}{EA}l=\frac{1}{4\sqrt{3}}\frac{Fl}{EA}
	\end{equation}
	となる.	
\end{enumerate}
\newpage
\subsection{練習問題}
以下の問題では,部材の断面$A$,ヤング率$E$,断面2次モーメント$I$は,全ての部材と断面で一定とする.
\subsubsection{練習問題1}
図\ref{fig:fig9}に示す骨組み構造について以下の問に答えよ. 
\begin{enumerate}
\item
	点Cにおける鉛直変位$v_C$を求めよ.
\item
	点Cにおける水平変位$u_C$を求めよ.
\end{enumerate}
\begin{figure}[h]
	\begin{center}
	\includegraphics[width=0.4\linewidth]{fig9.eps} 
	\end{center}
	\caption{水平部材に等分布荷重を受ける骨組み構造.} 
	\label{fig:fig9}
\end{figure}
\subsubsection{練習問題2}
図\ref{fig:fig10}に示すトラスについて以下の問に答えよ. 
\begin{enumerate}
\item
	点Dにおける水平変位$u_D$を求めよ.
\item
	点Eにおける水平変位$u_E$を求めよ.
\end{enumerate}
\begin{figure}[h]
	\begin{center}
	\includegraphics[width=0.5\linewidth]{fig10.eps} 
	\end{center}
	\caption{単純支持されたトラス.} 
	\label{fig:fig10}
\end{figure}
\end{document}

