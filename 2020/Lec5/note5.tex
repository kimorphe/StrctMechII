\documentclass[10pt,a4j]{jarticle}
%\usepackage{graphicx,wrapfig}
\usepackage{graphicx}
\usepackage{showkeys}
\setlength{\topmargin}{-1.5cm}
%\setlength{\textwidth}{16.5cm}
\setlength{\textheight}{25.2cm}
\newlength{\minitwocolumn}
\setlength{\minitwocolumn}{0.5\textwidth}
\addtolength{\minitwocolumn}{-\columnsep}
%\addtolength{\baselineskip}{-0.1\baselineskip}
%
\def\Mmaru#1{{\ooalign{\hfil#1\/\hfil\crcr
\raise.167ex\hbox{\mathhexbox 20D}}}}
%
\begin{document}
\newcommand{\fat}[1]{\mbox{\boldmath $#1$}}
\newcommand{\D}{\partial}
\newcommand{\w}{\omega}
\newcommand{\ga}{\alpha}
\newcommand{\gb}{\beta}
\newcommand{\gx}{\xi}
\newcommand{\gz}{\zeta}
\newcommand{\vhat}[1]{\hat{\fat{#1}}}
\newcommand{\spc}{\vspace{0.7\baselineskip}}
\newcommand{\halfspc}{\vspace{0.3\baselineskip}}
\bibliographystyle{unsrt}
%\pagestyle{empty}
\newcommand{\twofig}[2]
 {
   \begin{figure}
     \begin{minipage}[t]{\minitwocolumn}
         \begin{center}   #1
         \end{center}
     \end{minipage}
         \hspace{\columnsep}
     \begin{minipage}[t]{\minitwocolumn}
         \begin{center} #2
         \end{center}
     \end{minipage}
   \end{figure}
 }
%%%%%%%%%%%%%%%%%%%%%%%%%%%%%%%%%
%\vspace*{\baselineskip}
\begin{flushright}
	構造力学II\\
	2020/05/18
\end{flushright}
\begin{center}
	{\LARGE \bf 講義ノート 5} \\
\end{center}
%%%%%%%%%%%%%%%%%%%%%%%%%%%%%%%%%%%%%%%%%%%%%%%%%%%%%%%%%%%%%%%%
\section{骨組構造に対する仮想仕事式}
$n$本の直線部材を連結してできる骨組み構造を考える.
図\ref{fig:fig5}-(a)は部材数が$n=3$の骨組み構造を示したもので,以下ではこれを
具体例として参照しながら,一般の骨組み構造に対する仮想仕事式を導く.
なお,ここで対象とする骨組み構造では,部材間の連結は剛結でもピン結合でもよく,
一つの節点(部材の接合点)には何本の部材が連結されていてもよい.
各部材は,部材番号が与えられており,一般の部材の番号を$e$とし
断面力や断面係数等,その部材に関する諸量にはインデックス$e$をつけて表す.
例えば,部材$e$における軸力を$N_e$, 部材長さを$l_e$等と書く.
なお,部材中の断面位置や,荷重,変位方向を指定するためには, 図\ref{fig:fig5}-(b)に示す
ように,部材軸方向に$x_e$軸を,部材軸直角方向に$y_e$軸をとった$(x_e,\,y_e)$直交座標系を用いる.
\subsection{仮想仕事式}
図\ref{fig:fig5}に示す骨組み構造について,2つの異なる荷重条件:
\begin{eqnarray}
	条件1:
	\left( \bar{\fat{F}}, \, \bar M,  \fat{p}\right) 
	&=& \left( \bar{\fat{F}}^{(1)}, \, \bar M ^{(1)}, \fat{p}^{(1)}\right) 
	\label{eqn:sys1}
	\\
	条件2:
	\left( \bar{\fat{F}}, \, \bar M, \fat{p}\right) 
	&=&
	\left( \bar{\fat{F}}^{(2)}, \, \bar M ^{(2)}, \fat{p}^{(2)}\right) 
	\label{eqn:sys2}
\end{eqnarray}
に対して生じる変位を,それぞれ$\fat{u}^{(1)}, \fat{u}^{(2)}$と書く.
条件1にある骨組み構造を系1, 同じ骨組み構造が条件2の状態にある場合を系2とよぶ.
これら2つの系の,部材$e$に対する仮想仕事式は,部材両端部が図\ref{fig:fig5}-(b)のように荷重が既知の
境界と考えて仮想仕事式を立てれば,
\begin{equation}
	\int_{x_e=0}^{l_e} 
	\left(
	\frac{N_e^{(1)}N_e^{(2)}}{EA_e}
	+
	\frac{M_e^{(1)}M_e^{(2)}}{EI_e}
	\right)
	dx_e
	=
	\left[ 
		\bar{\fat{F}}_e^{(1)}\cdot \fat{u}^{(2)}
		-
		\bar{M}_e^{(1)}
		(v^{(2)})'
	\right]_{x_e=0}^{l_e}
	+
	\int_{x_e=0}^{l_e} 
	\fat{p}^{(1)}\cdot \fat{u}^{(2)}
	dx_e
	\label{eqn:}
\end{equation}
となる.ただし,
\begin{eqnarray}
	& & 
	\bar{\fat{F}}_e(0)=\left( \bar{N}^e_1, \, \bar{Q}^e_1 \right)^T, \ \  \bar M_e (0)=\bar M^e_1
	\label{eqn:}
	\\
	& & 
	\bar{\fat{F}}_e(l_e)=\left( \bar{N}^e_2, \, \bar{Q}^e_2 \right)^T, \ \  \bar M_e (l_e)=\bar M^e_2
	\label{eqn:}
\end{eqnarray}
で,右肩の括弧内の数字は,いずれの系に関する量であるかを示すためのものである.
なお,図\ref{fig:fig5}-(a)に示す構造において,部材$e=1$では,$x_e=0$において変位ベクトルと
たわみ角がゼロであることから,仮想仕事式は
\begin{equation}
	\int_{x_1=0}^{l_1} 
	\left(
	\frac{N_1^{(1)}N_1^{(2)}}{EA_1}
	+
	\frac{M_1^{(1)}M_1^{(2)}}{EI_1}
	\right)
	dx_1
	=
	\left. 
		\bar{\fat{F}}_1^{(1)}\cdot \fat{u}^{(2)}
	\right|_{x_1=l_1}
		-
	\left. 
		\bar{M}_1^{(1)}
		(v^{(2)})'
	\right|_{x_1=l_1}
	+
	\int_{x_1=0}^{l_1} 
	\fat{p}^{(1)}\cdot \fat{u}^{(2)}
	dx_1
	\label{eqn:}
\end{equation}
となる.また,部材$e=3$では$x_3=l_3$すなわちD点において,実際に荷重が指定された境界と
なっていることから
\begin{equation}
	\bar{\fat{F}}_3(l_3)=\left( \bar{N}^e_2=\bar{N}, \, \bar{Q}^3_2=\bar{Q} \right)^T, 
	\ \  \bar{ M}_3 (l_3)=\bar M
	\label{eqn:}
\end{equation}
とできる.この他の節点(B,C)における力は全て内力である.
以上を踏まえ,全ての部材に対する仮想仕事式の辺々の和をとると,骨組み構造全体に対する
仮想仕事式が
\begin{equation}
	a\left(
		 \fat{u}^{(1)}
		,\,  
		\fat{u}^{(2)}
	\right) 
	=
	b\left( 
		\fat{u}^{(2)}
	\right)
	\label{eqn:VW_frame}
\end{equation}
\begin{eqnarray}
	a\left(
		\fat{u}^{(1)}
		,\,  
		\fat{u}^{(2)}
	\right) 
	&=& 
	\sum_{e=1}^n 
	\int_{x_e=0}^{l_1} 
	\left(
	\frac{N_e^{(1)}N_e^{(2)}}{EA_e}
	+
	\frac{M_e^{(1)}M_e^{(2)}}{EI_e}
	\right)
	dx_e
	\label{eqn:}
	\\
	b\left( 
		 \fat{u}^{(2)}
	\right)
	&=&
	\left.
	\bar{\fat{F}}^{(1)} \cdot \fat{u}^{(2)}
	\right|_{x_3=l_3}
	-
	\left.
	\bar{M}^{(1)}(v^{(2)})'
	\right|_{x_3=l_3}
	+
	\sum_{e=1}^n 
	\int_{x_e=0}^{l_e} 
	\fat{p}^{(1)} \cdot \fat{u}^{(2)}dx_e
\end{eqnarray}
と得られる.
ここで,系1に対する荷重条件を
\begin{equation}
	\bar{\fat{F}}^{(1)}=\fat{0}, \ \ \bar M^{(1)} =0, 
	\ \ 
	\fat{p}^{(1)}=
	\left\{
	\begin{array}{cc}
	\hat{\fat{e}}\delta (x_{e}-a) & (e=e') \\
	\fat{0} & (e\neq e')
	\end{array}
	\right.
	\label{eqn:}
\end{equation}
とする.これは,部材$e=e'$の$x_{e'}=a$で,単位ベクトル$\hat{\fat{e}}$の方向に作用する
単位荷重のみが作用する場合を表す.このとき,仮想仕事式(\ref{eqn:VW_frame})は
\begin{equation}
	\left.
	\hat{\fat{e}}\cdot \fat{u}^{(2)}
	\right|_{x_{e'}=a}
	=
	\sum_{e=1}^n 
	\int_{x_e=0}^{l_1} 
	\left(
	\frac{N_e^{(1)}N_e^{(2)}}{EA_e}
	+
	\frac{M_e^{(1)}M_e^{(2)}}{EI_e}
	\right)
	dx_e
	\label{eqn:uload_frame}
\end{equation}
となり,骨組み構造に対する単位荷重法の式が導かれる.
なお,図\ref{fig:fig6}は,$e'=2$とした場合の,系1の荷重条件を示したものである.
最後に,
\begin{equation}
	\left(M^{(1)}_e,\, N_e^{(1)}\right)
	=
	\left(	\tilde M_e,\, \tilde N_e \right)
	\ \ 
	\left(M^{(2)}_e,\,N_e^{(2)}\right)
	=
	\left( M_e,\, N_e \right), \ \ 
	\fat{u}^{(2)}=\fat{u}, \ \ (e=1,\dots n)
	\label{eqn:}
\end{equation}
と書き直せば,式(\ref{eqn:uload_frame})は次のようになる.
\begin{equation}
	\left.
	\hat{\fat{e}}\cdot \fat{u}
	\right|_{x_{e'}=a}
	=
	\sum_{e=1}^n 
	\int_{x_e=0}^{l_1} 
	\left(
	\frac{N_e \tilde N_e}{EA_e}
	+
	\frac{M_e \tilde M_e}{EI_e}
	\right)
	dx_e
	\label{eqn:uload_frame2}
\end{equation}
この結果は一般の骨組み構造に対して成立するものである.一方,トラス構造の場合,各部材には
軸力のみが作用することから,曲げモーメントに関する項は現れない.また,節点の変位を単位荷重法で求める場合,問題
と補助系ともに,外力は全て節点に加えられるおとから,軸力が部材内で一定となる.
その結果,軸力に関する積分は簡単に実行することができ,単位荷重法の式は次のようになる.
\begin{equation}
	\left.
	\hat{\fat{e}}\cdot \fat{u}
	\right|_{x_{e'}=a}
	=
	\sum_{e=1}^n 
	\frac{N_e \tilde N_e}{EA_e}l_e
\end{equation}
\begin{figure}[h]
	\begin{center}
	\includegraphics[width=0.8\linewidth]{fig5.eps} 
	\end{center}
	\caption{(a)骨組み構造の例. 
	 (b)部材$e$に関する諸量と局所座標($x_e,\,y_e$).} 
	\label{fig:fig5}
\end{figure}
\begin{figure}
	\begin{center}
	\includegraphics[width=0.5\linewidth]{fig6.eps} 
	\end{center}
	\caption{単位集中荷重を受ける骨組み構造.} 
	\label{fig:fig6}
\end{figure}
\subsection{例題2}
図\ref{fig:fig7}に示した3種類の骨組み構造について以下の問に答えよ.
なお,断面剛性$EA$と$EI$は全ての部材,全ての断面で一定とする.
\begin{enumerate}
\item
	骨組み構造(a),(b),(c)のそれぞれにおける曲げモーメント分布を求めよ.
\item
	骨組み構造(a)の点Cにおける鉛直変位$v_C$を求めよ.
\item
	骨組み構造(a)の点Cにおける水平変位$u_C$を求めよ.
\end{enumerate}
\begin{figure}
	\begin{center}
	\includegraphics[width=1.0\linewidth]{fig7.eps} 
	\end{center}
	\caption{荷重条件が異なる3種類の骨組み構造(a)-(c).}
	\label{fig:fig7}
\end{figure}
\subsection{例題3}
図\ref{fig:fig8}に示す2つのトラス構造について次の問に答えよ.
なお,トラス部材の断面剛性$EA$は,全ての部材,全ての断面で一定とする.
\begin{enumerate}
\item
	トラス構造(a)の点Cにおける水平変位を求めよ.
\item
	トラス構造(b)の点Cにおける鉛直変位を求めよ.
\item
	トラス構造(a)の点Cにおける鉛直変位を求めよ.
\end{enumerate}
\begin{figure}
	\begin{center}
	\includegraphics[width=0.6\linewidth]{fig8.eps} 
	\end{center}
	\caption{(a)節点Cで水平力を受けるトラス.(b)節点Cで鉛直力を受けるトラス.}
	\label{fig:fig8}
\end{figure}
\subsection{問題1}
図\ref{fig:fig9}に示す骨組み構造(a),(b)および(c)について以下の問に答えよ. 
なお,断面剛性$EI$と$EA$は全ての部材,全ての断面で一定とする.
\begin{figure}
	\begin{center}
	\includegraphics[width=1.0\linewidth]{fig9.eps} 
	\end{center}
	\caption{支持条件と荷重条件の異なる3種類の骨組み構造.} 
	\label{fig:fig9}
\end{figure}
\begin{enumerate}
\item
	骨組み構造(a)の点Bにおける水平変位を求めよ.
\item
	骨組み構造(a)の点Bにおける鉛直変位を求めよ.
\item
	骨組み構造(b)の点Aにおける水平変位を求めよ.
\item
	骨組み構造(c)の点Cにおける鉛直変位を求めよ.
\item
	骨組み構造(c)の点Cにおける水平変位を求めよ.
\end{enumerate}
\subsection{問題2}
図\ref{fig:fig9}に示す,2つのトラス骨組(a)と(b)について以下の問に答えよ. 
なお,トラス部材の断面剛性$EA$は,全ての部材,全ての断面で一定とする.
\begin{figure}
	\begin{center}
	\includegraphics[width=0.8\linewidth]{fig10.eps} 
	\end{center}
	\caption{単純支持された二つのトラス構造.} 
	\label{fig:fig10}
\end{figure}
\begin{enumerate}
\item
	トラス構造(a)の点Dにおける水平変位を求めよ.
\item
	トラス構造(a)の点Eにおける水平変位を求めよ.
\item
	トラス構造(b)の点Dにおける荷重方向の変位を求めよ.
\end{enumerate}
\end{document}

