\documentclass[10pt,a4j]{jarticle}
%\usepackage{graphicx,wrapfig}
\usepackage{graphicx}
\usepackage{showkeys}
\setlength{\topmargin}{-1.5cm}
\setlength{\textwidth}{16.5cm}
\setlength{\textheight}{25.2cm}
\newlength{\minitwocolumn}
\setlength{\minitwocolumn}{0.5\textwidth}
\addtolength{\minitwocolumn}{-\columnsep}
%\addtolength{\baselineskip}{-0.1\baselineskip}
%
\def\Mmaru#1{{\ooalign{\hfil#1\/\hfil\crcr
\raise.167ex\hbox{\mathhexbox 20D}}}}
%
\begin{document}
\newcommand{\fat}[1]{\mbox{\boldmath $#1$}}
\newcommand{\D}{\partial}
\newcommand{\w}{\omega}
\newcommand{\ga}{\alpha}
\newcommand{\gb}{\beta}
\newcommand{\gx}{\xi}
\newcommand{\gz}{\zeta}
\newcommand{\vhat}[1]{\hat{\fat{#1}}}
\newcommand{\spc}{\vspace{0.7\baselineskip}}
\newcommand{\halfspc}{\vspace{0.3\baselineskip}}
\bibliographystyle{unsrt}
%\pagestyle{empty}
\newcommand{\twofig}[2]
 {
   \begin{figure}
     \begin{minipage}[t]{\minitwocolumn}
         \begin{center}   #1
         \end{center}
     \end{minipage}
         \hspace{\columnsep}
     \begin{minipage}[t]{\minitwocolumn}
         \begin{center} #2
         \end{center}
     \end{minipage}
   \end{figure}
 }
%%%%%%%%%%%%%%%%%%%%%%%%%%%%%%%%%
%\vspace*{\baselineskip}
\begin{flushright}
	構造力学II\\
	2020/04/27
\end{flushright}
\begin{center}
	{\Large \bf 講義ノート2} \\
\end{center}
%%%%%%%%%%%%%%%%%%%%%%%%%%%%%%%%%%%%%%%%%%%%%%%%%%%%%%%%%%%%%%%%
\setcounter{section}{1}
\section{曲げ問題}
\subsection{強形式(復習)}
曲げ問題のたわみに関する支配方程式は
\begin{equation}
	\frac{d^2}{dx^2}\left(EI \frac{d^2v}{dx^2}\right) -q =0, \ \ \left(x\in (0,l)\right)
	\label{eqn:gveq_beam}
\end{equation}
で与えられる.ここで,$v$はたわみを,$E$はヤング率を,
$I$は部材の断面2次モーメントを,$q(x)$は鉛直下向きに作用する分布力を表し,
$q(x)$は単位長さあたりの力の次元を持つ.
断面$S$に関する断面2次モーメント$I$は,
\begin{equation}
	I=\int_S y^2 dS
	\label{eqn:}
\end{equation}
で与えられる.$y$は中立軸からの距離を表し,中立軸位置は断面1次モーメント
に関する次の条件から決定できる.
\begin{equation}
	G=\int _S ydS=0
	\label{eqn:}
\end{equation}
式(\ref{eqn:gveq_beam})は, 曲げモーメント$M$とせん断力$Q$に関する釣り合い条件:
\begin{eqnarray}
	\frac{dM}{dx} &=& Q
	\label{eqn:equiv_M} \\
	\frac{dQ}{dx} &=& -q,
	\label{eqn:equiv_Q}
\end{eqnarray}
フックの法則と平面保持の仮定に由来する次の関係から導かれるものである.
\begin{equation}
	M=EI \kappa= -EI\frac{d^2v}{dx^2}
	\label{eqn:M_k}
\end{equation}
ただし$\kappa$は曲率を表し,たわみ$v$と$\kappa\approx -v''$の関係にある. 
なお,釣り合い条件式(\ref{eqn:equiv_M})と(\ref{eqn:equiv_Q})をまとめると,
\begin{equation}
	\frac{d^2M}{dx^2}+q=0
	\label{eqn:equiv_MQ}
\end{equation}
となる.梁が図\ref{fig:fig2_1}のように,片持梁として支持されている場合,
$x=0$と$x=l$における境界条件(支持条件)は書くことができる.
\begin{eqnarray}
	v (0) &= & 0 
		\label{eqn:BC_v}
	\\
	v' (0) &= & 0 
		\label{eqn:BC_th}
	\\
	M (l) &= & -EI v''(l)=\bar M 
		\label{eqn:BC_M} \\
	Q (l) &= & \left(-EI v''\right)'(l)=\bar Q 
		\label{eqn:BC_Q}
	\label{eqn:BC_Q} 
\end{eqnarray}
\begin{figure}[h]
	\begin{center}
	\includegraphics[width=0.6\linewidth]{fig1.eps} 
	\end{center}
	\caption{鉛直下向きの分布荷重$q(x)$を受ける片持梁.} 
	\label{fig:fig2_1}
\end{figure}
\subsection{弱形式}
$\eta (x)$を,$0\leq x \leq l$上で定義され,
\begin{equation}
		\eta(0)=0, \ \  \eta'(0)=0
	\label{eqn:eta_x0}
\end{equation}
を満たす任意の関数とする.$\eta (x)$を式(\ref{eqn:equiv_MQ})の両辺に掛け,
$x=0$から$x=l$の範囲で次のように積分する.
\begin{equation}
	\int _0^l 
	\left(M''+q \right) \eta dx =0
	\label{eqn:int_gveq_eta}
\end{equation}
この式の左辺第一項を,部分積分を2回行って変形すると,
\begin{eqnarray}
	\int _0^l M''\eta dx &= & \left[ M'\eta \right]_0^l -\int_0^l M' \eta'dx  
	\label{eqn:ibp0}
	\\
	&= & \left[ M'\eta \right]_0^l -\left[M\eta'\right]_0^l+\int_0^l M \eta''dx  
	\label{eqn:ibp1}
	\\
	&= & \bar Q \eta(l) - \bar M \eta'(l)+\int_0^l M \eta''dx  
	\label{eqn:ibp2}
\end{eqnarray}
となる.なお,式(\ref{eqn:ibp1})から式(\ref{eqn:ibp2})を得るにあたり,
$\eta{x}$に関する条件(\ref{eqn:eta_x0})と,曲げモーメントおよびせん断力に関する
境界条件(\ref{eqn:BC_M}),(\ref{eqn:BC_Q})を用いた.
ここで,式(\ref{eqn:ibp2})を式(\ref{eqn:int_gveq_eta})に代入し,
\begin{equation}
	a(v,\eta) = -\int_0^l M \eta''dx  
	= \int_0^l EI v'' \eta''dx  
	\label{eqn:blinf_M}
\end{equation}
\begin{equation}
	b(\eta)= 
	\bar Q \eta (l) - 
	\bar M \eta' (l) + 
	\int_0^l q\eta dx 
	\label{eqn:linf_M}
\end{equation}
とおけば,
\begin{equation}
	a(v,\eta)=b(\eta)
	\label{eqn:WF_M}
\end{equation}
となることが示される.
$a(v,\eta)$と$b(\eta)$は,ともに,関数$v$や$\eta$に作用して一つの値を返す関数と
見ることができる.このような,関数の関数は汎関数(functional)と呼ばれ,$a(v,\eta)$を
双線形形式(bilinear form),$b$を線形形式(linear form)と呼ぶ.
以上より,曲げ問題の弱形式による定式化が次のように与えられる.\\
\begin{itemize}
	\item
	{\bf 曲げ問題の弱形式:}
	任意の$\eta(x)$(ただし$\eta(0)=0,\, \eta'(0)=0$)について,式(\ref{eqn:WF_M})を満足する
	$v(x)$のうち,$v(0)=0,\, v'(0)=0$となるものを(弱解)を求めよ.
\end{itemize}
弱解が強解に一致することの証明は,軸力問題の場合と同様にして行うことができる.
\subsection{仮想仕事式}
図\ref{fig:fig2_2}に示すような2つの系を考える.これらの系は互いに外力のみが異なり,
支持条件と曲げ剛性$EI$は共通とする.ここで,系$i,\, (i=1,2)$に加えられた分布力を
$q_i(x)$,部材右端Bに作用する鉛直力と曲げモーメントをそれぞれ$\bar Q_i, \bar M_i$とする.
また,これらの外力に起因して系$i$に生じるたわみを$v_i$, たわみ角を$\theta_i$とし,
曲げモーメントとせん断力をそれぞれ$M_i(x), Q_i(x)$と表すことにする.
\begin{figure}[h]
	\begin{center}
	\includegraphics[width=0.8\linewidth]{fig2.eps} 
	\end{center}
	\caption{荷重条件のみが異なる2つの系.(a)系1, (b)系2.} 
	\label{fig:fig2_2}
\end{figure}
ここで,式(\ref{eqn:WF_M})において,$v=v_1$, $\eta=v_2$とすれば,
式(\ref{eqn:WF_M})の各辺は,
\begin{equation}
	a(v_1,v_2)=-\int_0^l M_1v''_2dx = \int_0^l\frac{M_1M_2}{EI}dx
	\label{eqn:vw_int_M}
\end{equation}
\begin{equation}
	b(v_2)=\bar Q_1 v_2(l)-\bar M_1  \theta_2(l) +\int_0^l q_1 v_2dx
	\label{eqn:vw_ext_M}
\end{equation}
となる.式(\ref{eqn:vw_int_M})と式(\ref{eqn:vw_ext_M})は,ともに
力$\times$長さ,すなわち仕事の次元を持つ量となっている.
前者は内力である$M_1$と$Q_1$, 後者は外力である$\bar M_1, \bar Q_1$と$q_1(x)$
に関するものであることから,$a(v_1,v_2)$を内部仮想仕事,$b(v_2)$を外部仮想仕事と呼ぶ.
系1と系2の間に成り立つ関係
\begin{equation}
	a(v_1,v_2)=b(v_2)
	\label{eqn:vw_eq_M}
\end{equation}
を,曲げ問題に対する仮想仕事式と呼ぶ.
%%%%%%%%%%%%%%%%%%%%%%%%%5
\subsection{単位荷重法}
仮想仕事式(\ref{eqn:vw_eq_M})における系1として,次のような特別な場合を考える.
\begin{equation}
	\bar Q_1=0,\ \ \bar M_1=0\ \ q_1(x)=\delta(x-a), \ \ (0<a<l)
	\label{eqn:sys1_M}
\end{equation}
すなわち,系1として$x=a$に単位集中荷重が外力として作用する場合を考える.
このとき,デルタ関数の性質を用いれば,式(\ref{eqn:vw_ext_M})は次のようになる. 
\begin{equation}
	b(v_2)=\int_0^l \delta(x-a)v_2(x)dx= v_2(a).
	\label{eqn:vw_eq_dlt}
\end{equation}
これを仮想仕事式(\ref{eqn:vw_eq_M})に代入すれば,
\begin{equation}
	v_2(a)=\int_0^l \frac{M_1M_2}{EI}dx
	\label{eqn:u2_a}
\end{equation}
を得る.これは,2つの系における曲げモーメント$M_1$と$M_2$から,
$x=a$における系2のたわみ$v_2(a)$が求められることを示している.
ここで, 系1を補助系,系2を解くべき問題とみるために,
系1と2に関する諸量を,あらためて 
\begin{eqnarray}
	\left( v_1, \theta_1, M_1, Q_1;\, \bar M_1, \bar Q_1, q_1 \right)& = &
		\left(\tilde v, \tilde \theta, \tilde M, \tilde Q;\, 
		\bar M_1=0, \bar Q_1=0, q_1= \delta(x-a) \right) 
	\label{eqn:aux}
	\\
	\left( v_2, \theta_2, M_2, Q_2 ;\, \bar M_2, \bar Q_2, q_2 \right)& = &
		\left( v,\theta,M, Q;\, \bar M,\bar Q, q \right) 
	\label{eqn:prb_M}
\end{eqnarray}
と書き直す.このとき式(\ref{eqn:u2_a})は,
\begin{equation}
	v(a)=\int_0^l \frac{M \tilde M}{EI}dx
	\label{eqn:uload_M}
\end{equation}
と表される.式(\ref{eqn:uload_M})を利用してたわみの計算を行う解法を曲げ問題に対する単位荷重法と呼ぶ.
\subsubsection{例題1}
図\ref{fig:fig2_3}-(a)に示す片持梁ABの点Bにおけるたわみ$u_B$と
点Cにおけるたわみ$u_C$を単位荷重法を用いて求めよ.
曲げ剛性$EI$は全断面で一定とする.
\paragraph{解答:}
問題で与えら得れた系の曲げモーメントは,図\ref{f:g:fig2_3}に示した座標$s$により
\begin{equation}
	M(s)=-\frac{q_0s^2}{2}
	\label{eqn:}
\end{equation}
と与えられる.
単位荷重法を適用するための補助系には,たわみを求めたい点に鉛直荷重を加えた系を用いればよい。
点Bにおけるたわみを求める際には$x=l/2$に,	
点Cにおけるたわみを求める際には$x=l$に,集中荷重が加えられた系を補助系として,
その曲げモーメント$\tilde M$を求める.
ここでは,より一般の場合を考え、位置$s=l-x=b$に集中荷重が加えられた場合の
曲げモーメントを求めておけば、
\begin{equation}
	\tilde M(s) = -H(s-b)(s-b), \ \ (b=l-a)
	\label{eqn:}
\end{equation}
となる.ただし$H(s)$は単位ステップ関数:
\begin{equation}
	H(s)=
	\left\{
	\begin{array}{cc}
		0 &  (s<0) \\
		1 & ( s>0)
	\end{array}
	\right.
	\label{eqn:}
\end{equation}
を表す.また、$\int_SM\tilde Mds$を計算すると
\begin{eqnarray}
	\int_0^l M(s)\tilde M(s) ds
	&=&
	\int_b^l\frac{q_0s^2}{2}(s-b)ds \\
	&=&
	\frac{q_0}{2}\int_b^ls^3-bs^2ds \\
	&=&
	\frac{q_0}{2}\left[
		\frac{s^4}{4}-\frac{bs^3}{3}
		\right]_b^l \\
	&=&
	\frac{q_0}{2}
	\left(
	\frac{l^4-b^4}{4}-\frac{bl^3-b^4}{3}
	\right)
	\\
	&=&
	\frac{q_0l^4}{2}\left\{
		\frac{1}{4}
		-\frac{1}{3}
		\left(\frac{b}{l}\right)
		+
		\frac{1}{12}
		\left(\frac{b}{l}\right)^4
	\right\}
	\label{eqn:}
\end{eqnarray}
となることが示される.この結果において$b=l/2$および$b=0$とすれば
\begin{equation}
	\int_0^l M(s)\tilde M(s) ds =
	\left\{
	\begin{array}{cc}
		\frac{1}{8}q_0l^4& (b=0)  \\
		\frac{17}{384}q_0l^4& (b=l/2)  
	\end{array}
	\right.
	\label{eqn:}
\end{equation}
となり,以上を式(\ref{eqn:uload_M})に代入すれば、
求めるべきたわみが次のように得られる.
\begin{equation}
	u_B= \frac{17}{384}\frac{q_0l^4}{EI}
	, \ \ 
	u_C= \frac{1}{8}\frac{q_0l^4}{EI}
	\label{eqn:}
\end{equation}
\begin{figure}[h]
	\begin{center}
	\includegraphics[width=0.8\linewidth]{fig3.eps} 
	\end{center}
	\caption{(a)等分布荷重を受ける片持梁(問題).
	 (b)単位荷重法の適用において用いる補助系.} 
	\label{fig:fig2_3}
\end{figure}
\subsubsection{例題2}
	図\ref{fig:fig2_5}-(a)に示す単純支持梁の点Bにおけるたわみを求めよ.
\paragraph{解答:}
	点Bにおけるたわみを求めるためには,点Bに単位集中荷重を加えた図\ref{fig:fig2_5}-(b)を
	補助系として単位荷重法を適用すればよい.
	単純支持梁は,静定構造であるため,曲げモーメント分布は力の釣り合いから決定することができ,
	その結果を問題の系,補助系,それぞれについて示すと図\ref{fig:fig2_6}のようになる.
	この図にあるように,曲げモーメントはいずれの系でも左右対称で,左半分の区間ABにおける
	曲げモーメント分布を位置$x$の関数として表せば,以下のようになる.
	\begin{eqnarray}
		M(x) &=& \frac{q_0lx}{2}-\frac{q_0x^2}{2}, \ \ \left(x<\frac{l}{2}\right) \\
		\tilde M(x) &=& \frac{\tilde P x}{2}
		\label{eqn:}
	\end{eqnarray}
	これを踏まえて$\int_0^{l/2} M\tilde Mdx$を求めると,
	\begin{equation}
		2\int_0^{l/2} M\tilde Mdx
		=\frac{\tilde Pq_0l^4}{2}
		\int_0^{l/2} 
		\left\{
			\left(\frac{x}{l}\right)- \left(\frac{x}{l} \right)^2
		\right\}
			\left(\frac{x}{l}\right)
		\frac{dx}{l}
		=\frac{5}{384}q_0l^4
		\label{eqn:}
	\end{equation}
	となる.よって、
	\begin{equation}
		u\left(\frac{l}{2}\right)=
		\int_0^l \frac{M\tilde M}{EI}dx=
		2\int_0^{l/2} \frac{M\tilde M}{EI}dx
		=\frac{5}{384}\frac{q_0l^4}{EI}
		.
		\label{eqn:}
	\end{equation}
\begin{figure}[h]
	\begin{center}
	\includegraphics[width=0.75\linewidth]{fig4_1.eps} 
	\end{center}
	\caption{(a)等分布荷重を受ける単純支持梁(問題).
	 (b)単位荷重法の適用において用いる補助系.} 
	\label{fig:fig2_5}
\end{figure}
\begin{figure}[h]
	\begin{center}
	\includegraphics[width=0.8\linewidth]{fig4_1_2.eps} 
	\end{center}
	\caption{(a)問題,(b)補助系の曲げモーメント図(例題2).}
	\label{fig:fig2_6}
\end{figure}
\subsubsection{問題}
図\ref{fig:fig2_4}に示す2種類の系に関する以下の問に答えよ.
なお,梁の曲げ剛性$EI$は全ての部材と断面で一定とする.
\begin{enumerate}
\item
	図\ref{fig:fig2_4}-(a)に示す単純支持梁の, 点Bにおけるたわみを求めよ.\\
	{\bf 解答:}$\frac{Pl^3}{48EI}$
\item
	図\ref{fig:fig2_4}-(b)に示す単純支持梁の, 点Bにおけるたわみ$v_B$と
	点Cにおけるたわみ$v_C$を求めよ.\\
	{\bf 解答:}$v_B=\frac{q_0l^4}{128EI}, v_C=\frac{11}{768}\frac{q_0l^4}{EI}$
\end{enumerate}
\begin{figure}[h]
	\begin{center}
	\includegraphics[width=0.8\linewidth]{fig4_2.eps} 
	\end{center}
	\caption{支持条件,荷重条件の異なる2種類の単径間梁.} 
	\label{fig:fig2_4}
\end{figure}
\end{document}

