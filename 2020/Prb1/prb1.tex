\documentclass[10pt,a4j]{jarticle}
\usepackage{graphicx,wrapfig}
\setlength{\topmargin}{-1.5cm}
%\setlength{\textwidth}{15.5cm}
\setlength{\textheight}{25.2cm}
\newlength{\minitwocolumn}
\setlength{\minitwocolumn}{0.5\textwidth}
\addtolength{\minitwocolumn}{-\columnsep}
%\addtolength{\baselineskip}{-0.1\baselineskip}
%
\def\Mmaru#1{{\ooalign{\hfil#1\/\hfil\crcr
\raise.167ex\hbox{\mathhexbox 20D}}}}
%
\begin{document}
\newcommand{\fat}[1]{\mbox{\boldmath $#1$}}
\newcommand{\D}{\partial}
\newcommand{\w}{\omega}
\newcommand{\ga}{\alpha}
\newcommand{\gb}{\beta}
\newcommand{\gx}{\xi}
\newcommand{\gz}{\zeta}
\newcommand{\vhat}[1]{\hat{\fat{#1}}}
\newcommand{\spc}{\vspace{0.7\baselineskip}}
\newcommand{\halfspc}{\vspace{0.3\baselineskip}}
\bibliographystyle{unsrt}
\pagestyle{empty}
\newcommand{\twofig}[2]
 {
   \begin{figure}[here]
     \begin{minipage}[t]{\minitwocolumn}
         \begin{center}   #1
         \end{center}
     \end{minipage}
         \hspace{\columnsep}
     \begin{minipage}[t]{\minitwocolumn}
         \begin{center} #2
         \end{center}
     \end{minipage}
   \end{figure}
 }
%%%%%%%%%%%%%%%%%%%%%%%%%%%%%%%%%
%\vspace*{\baselineskip}
\begin{center}
{\Large \bf 2020年度 構造力学II レポート課題1} \\
\end{center}
\begin{flushright}
	提出期限:2020年5月15日(金)18:00\\
%提出先:環境理工学部棟3Fレポートボックス
	提出先:{\bf MoodleからPDFファイルで提出}
\end{flushright}
%%%%%%%%%%%%%%%%%%%%%%%%%%%%%%%%%%%%%%%%%%%%%%%%%%%%%%%%%%%%%%%%
\vspace{15mm}
図\ref{fig:fig1}に示す(a),(b)の梁について,以下の問題に答えよ.
ただし,梁の曲げ剛性$EI$は全断面で一定とし,たわみは鉛直下向きを正とする.
\subsubsection*{問題 1}
梁(a)の点Dにおけるたわみ$v_D$を求めよ.
\subsubsection*{問題 2}
梁(b)の受ける支点反力を求め,曲げモーメント図を描け.
%--------------------
\begin{figure}[h]
	\begin{center}
	\includegraphics[width=0.55\linewidth]{fig1.eps} 
	\end{center}
	\caption{区間ABに等分布荷重を受ける梁.} 
	\label{fig:fig1}
\end{figure}
%--------------------
%%%%%%%%%%%%%%%%%%%%%%%%%%%%%%%%%%%%%%%%%%%%
\end{document}

