\documentclass[10pt,a4j]{jarticle}
\usepackage{graphicx,wrapfig}
%\usepackage{showkeys}
\setlength{\topmargin}{-1.5cm}
%\setlength{\textwidth}{15.5cm}
\setlength{\textheight}{25.2cm}
\newlength{\minitwocolumn}
\setlength{\minitwocolumn}{0.5\textwidth}
\addtolength{\minitwocolumn}{-\columnsep}
%\addtolength{\baselineskip}{-0.1\baselineskip}
%
\begin{document}
\def\Mmaru#1{{\ooalign{\hfil#1\/\hfil\crcr
\raise.167ex\hbox{\mathhexbox 20D}}}}


\newcommand{\fat}[1]{\mbox{\boldmath $#1$}}
\newcommand{\D}{\partial}

\newcommand{\w}{\omega}
\newcommand{\ga}{\alpha}
\newcommand{\gb}{\beta}
\newcommand{\gx}{\xi}
\newcommand{\gz}{\zeta}
\newcommand{\vhat}[1]{\hat{\fat{#1}}}
\newcommand{\spc}{\vspace{0.7\baselineskip}}
\newcommand{\halfspc}{\vspace{0.3\baselineskip}}
\bibliographystyle{unsrt}
%\pagestyle{empty}
\newcommand{\twofig}[2]
 {
   \begin{figure}[h]
     \begin{minipage}[t]{\minitwocolumn}
         \begin{center}   #1
         \end{center}
     \end{minipage}
         \hspace{\columnsep}
     \begin{minipage}[t]{\minitwocolumn}
         \begin{center} #2
         \end{center}
     \end{minipage}
   \end{figure}
 }
%%%%%%%%%%%%%%%%%%%%%%%%%%%%%%%%%
%\vspace*{\baselineskip}
\begin{center}
{\Large \bf 2020年度 構造力学II レポート課題1解答} \\
\end{center}
%%%%%%%%%%%%%%%%%%%%%%%%%%%%%%%%%%%%%%%%%%%%%%%%%%%%%%%%%%%%%%%%
\subsubsection*{問題1.}
問題で与えれた系と,単位荷重法における補助系を図\ref{fig:fig1}に示す.
これらはいずれも静定構造のため,それぞれ,曲げモーメント分布を
釣り合い条件から求めることができる.同図の下段は,その結果を
曲げモーメント図として示したものである.
この図にあるように,曲げモーメント$M$は区間ABで,$\tilde M$は区間CDで零である.
従って,単位荷重法の計算で必要となるのは,区間BCにおける曲げモーメントで,
それらは,次のように座標$s$を用いて表すことができる.
\begin{equation}
	M(s)=-\frac{q_0l}{2}s, \ \ \tilde M(s)=-\tilde P s, \ \ (0<s<l)
\end{equation}
この結果より,
\begin{equation}
	\int_A^D M\tilde Mds= \int_B^CM \tilde Mds=\frac{1}{12}q_0l^4
\end{equation}
となることが分かる.よって,$v_D=\frac{1}{12}\frac{q_0l^4}{EI}$.
%%%%%%%%%%%%%%%%%
\begin{figure}[h]
	\begin{center}
	\includegraphics[width=1.0\linewidth]{fig1ans.eps} 
	\end{center}
	\caption{問題1で与えられた系と単位荷重法のおける補助系の曲げモーメント図. } 
	\label{fig:fig1}
\end{figure}
\subsubsection*{問題2.}
問題で与えられた梁を図\ref{fig:fig2}に示す2つの系(系1,系2)の重ね合わせで表現する.
系1の点Dにおけるたわみを$v^{(1)}_D$,系2の点Dにおけるたわみを$v^{(2)}_D$とすれば,
これらが満足すべき適合条件は
\begin{equation}
	v^{(1)}_D
	+
	v^{(2)}_D
	=0
\end{equation}
である.$v_D^{(1)}$は,問題1での計算の結果$\frac{1}{12}\frac{q_0l^4}{EI}$となる
ことは既に分かっている.$v^{(2)}_D$の計算にはは,問題1と同じ補助系を用いることができる.
また系2の曲げモーメントは,図\ref{fig:fig1}-(b)において$\tilde =-R_D$と置くことで得られる.
以上を踏まえて曲げモーメントに関する積分$\int_A^DM\tilde M ds $を計算すると,
\begin{equation}
	\int_A^D M\tilde Mds= 2\int_B^CM \tilde Mds=\frac{1}{12}q_0l^4=-\frac{2}{3}q_0l^4
\end{equation}
となる.よって,$v_D^{(2)}=\frac{2}{3}\frac{q_0l^4}{EI}$で,これを式(\ref{eqn:vD_vanish})
に代入すれば,
\begin{equation}
	\frac{1}{12}\frac{q_0l^4}{EI}
	+
	D^{(2)}=\frac{2}{3}\frac{q_0l^4}{EI}
	=0 \ \ \Rightarrow \ \ 
	R_D=\frac{q_0l}{8}
\end{equation}
となることが分かる.
$R_D$が既知となったため,梁全体の釣り合い条件を使えば残りの支点反力が求められ,
その結果は次のようになる.
\begin{equation}
	R_B=\frac{13}{8}q_0l, \ \ R_C=-\frac{3}{4}q_0l
\end{equation}
最後に,釣り合い条件から曲げモーメントを求め,曲げモーメント図として表せば,最終的に
図\ref{fig:fig3}のようになることが示される.
%%%%%%%%%%%%%%%%%%%%%%%%%%%%
\begin{figure}
	\vspace{10mm}
	\begin{center}
	\includegraphics[width=0.6\linewidth]{fig2ans.eps} 
	\end{center}
	\caption{問題2で与えられた系の2つの系1と2への分解. } 
	\label{fig:fig2}
\end{figure}
\begin{figure}
	\vspace{10mm}
	\begin{center}
	\includegraphics[width=0.5\linewidth]{fig3ans.eps} 
	\end{center}
	\caption{曲げモーメント図(問題2). } 
	\label{fig:fig3}
\end{figure}
\end{document}
