\documentclass[10pt,a4j]{jarticle}
\usepackage{graphicx,wrapfig}
%\usepackage{showkeys}
\setlength{\topmargin}{-1.5cm}
\setlength{\textwidth}{15.5cm}
\setlength{\textheight}{25.2cm}
\newlength{\minitwocolumn}
\setlength{\minitwocolumn}{0.5\textwidth}
\addtolength{\minitwocolumn}{-\columnsep}
%\addtolength{\baselineskip}{-0.1\baselineskip}
%
\begin{document}
\def\Mmaru#1{{\ooalign{\hfil#1\/\hfil\crcr
\raise.167ex\hbox{\mathhexbox 20D}}}}


\newcommand{\fat}[1]{\mbox{\boldmath $#1$}}
\newcommand{\D}{\partial}

\newcommand{\w}{\omega}
\newcommand{\ga}{\alpha}
\newcommand{\gb}{\beta}
\newcommand{\gx}{\xi}
\newcommand{\gz}{\zeta}
\newcommand{\vhat}[1]{\hat{\fat{#1}}}
\newcommand{\spc}{\vspace{0.7\baselineskip}}
\newcommand{\halfspc}{\vspace{0.3\baselineskip}}
\bibliographystyle{unsrt}
%\pagestyle{empty}
\newcommand{\twofig}[2]
 {
   \begin{figure}[h]
     \begin{minipage}[t]{\minitwocolumn}
         \begin{center}   #1
         \end{center}
     \end{minipage}
         \hspace{\columnsep}
     \begin{minipage}[t]{\minitwocolumn}
         \begin{center} #2
         \end{center}
     \end{minipage}
   \end{figure}
 }
%%%%%%%%%%%%%%%%%%%%%%%%%%%%%%%%%
%\vspace*{\baselineskip}
\begin{center}
{\Large \bf 2020年度 構造力学II レポート課題1解答} \\
\end{center}
%%%%%%%%%%%%%%%%%%%%%%%%%%%%%%%%%%%%%%%%%%%%%%%%%%%%%%%%%%%%%%%%
\subsubsection*{問題1.}
問題で与えられた梁を図\ref{fig:fig1}に示す2つの系の重ね合わせで表現する.
このとき,2つの系の曲げモーメント$M_1$と$M_2$を,釣り合い条件から求めると,
\begin{equation}
	M_1=\left\{
	\begin{array}{ll}
		-Fx & (0< x < l/2) \\ 
		-\frac{F}{2}s=-\frac{F}{2}(l-t)& (0< s < l)
	\end{array}
	\right.
\end{equation}
\begin{equation}
	M_2=\left\{
	\begin{array}{ll}
		0 & (0< x < l/2) \\ 
		\frac{F}{2}t & (0< t < l/2) \\ 
		\frac{F}{2}s & (0< s < l/2) \\ 
	\end{array}
	\right.
\end{equation}
と表され, 問題で与えられた系の曲げモーメント$M$はこれらの和で与えられる.
ただし,$x,s$および$t$は図\ref{fig:fig1}に示した座標を表す.
これらの結果を曲げモーメント図として表すと,同図の下側に示したようになる.
単位荷重法を用いて点Aのたわみを求めるには,図\ref{fig:fig1}-(a)において
$F=\tilde F=1$としたものを補助系として用いればよい.このときの曲げモーメント
を$\tilde M$とすれば,$i=1,2$に対して
\begin{equation}
	\int_A^D M_i\tilde M dx
	=
	\int_{x=0}^{l/2}M_i \tilde M dx
	+
	\int_{t=0}^{l/2}M_i \tilde M dt
	+
	\int_{s=0}^{l/2}M_i \tilde M ds, \ \ (i=1,2)
\end{equation}
であるから,右辺の積分を一つずつ計算すると,
\begin{eqnarray}
	\int_A^B M_1\tilde Mdx 
	&=& 
	\int_0^{l/2}M_1\tilde Mdx 
	=
	\int_0^l (-Fx)(-\tilde Fx)dx
	=\frac{Fl^3}{24}\tilde F \\
	\int_D^B M_1\tilde M ds 
	&=& 
	\int_0^{l}M_1\tilde Mds 
	= 
	\int_0^l 
	\left(-\frac{Fs}{2} \right) \left(-\frac{\tilde Fs}{2} \right)ds
	=
	\frac{Fl^3}{12}\tilde F \\
	\int_B^C M_2\tilde M dt 
	&=& 
	\int_0^{l/2}M_2\tilde Mdt 
	= 
	\int_0^{l/2}
	\left(\frac{Ft}{2} \right) \left(-\frac{\tilde F}{2}(l-t) \right)dt
	=-\frac{Fl^3}{48}\tilde F \\
	\int_D^C M_2\tilde M ds
	&=&
	\int_0^{l/2}M_2\tilde Mds 
	= 
	\left(\frac{Fs}{2} \right) \left(-\frac{\tilde Fs}{2} \right) ds
	=-\frac{Fl^3}{96}\tilde F 
\end{eqnarray}
となることが示される.
以上に$\tilde F=1$を代入して用いれば,
\begin{equation}
	\int_A^D M\tilde M dx
	=
	\int_A^D (M_1+M_2) \tilde M dx
	=
	\frac{3}{32}Fl^3
\end{equation}
となることから,A点のたわみ$v_A$が
\begin{equation}
	v_A= \frac{3}{32}\frac{Fl^3}{EI}
\end{equation}
と求められる.
%%%%%%%%%%%%%%%%%
\begin{figure}
	\begin{center}
	\includegraphics[width=1.0\linewidth]{fig1ans.eps} 
	\end{center}
	\caption{問題1で与えられた系の2つの系1と2への分解と曲げモーメント図. } 
	\label{fig:fig1}
\end{figure}
\subsubsection*{問題2.}
%%%%%%%%%%%%%%%%%%%%%%%%%%%%
問題で与えられた系を,図\ref{fig:fig2}に示す2つの系1と2に分解する.
系1と系2の点Bにおけるたわみを, それぞれ$\delta_1$, $\delta_2$と表す.
系1と2の曲げモーメント$M_1$と$M_2$を力の釣り合いから求めると, 
$M_1,_M2$は
\begin{equation}
	M_1=\left\{
	\begin{array}{ll}
		-\frac{1}{2}q_0x^2 & (0< x < 2l/3) \\ 
		-\frac{2}{3}q_0l^2 \left(\frac{s}{l}+\frac{1}{3}\right) & (0< s < l/3) 
	\end{array}
	\right.
	\label{eqn:M1_2}
\end{equation}
\begin{equation}
	M_2=\left\{
	\begin{array}{ll}
		0 & (0< x < 2l/3) \\ 
		R_Bs & (0< s < l/3) 
	\end{array}
	\right.
	\label{eqn:M2_2}
\end{equation}
となる.ただし,$x$と$s$は図\ref{fig:fig2}に示したような座標を表す.
これらの結果を曲げモーメント図として示せば,図\ref{fig:fig2}の下側
のようである.
単位荷重法によって$\delta_1,\delta_2$を求めるための補助系は,
系2において$R_B=-1$とおけばよい.その結果として得られる曲げモーメント分布を
$\tilde M$と表せば,$\int_A^C M_1\tilde M dx,\,(i=1,2)$は,
\begin{equation}
	\int_A^CM_1\tilde Mdx=
	\int_0^{l/3}M_1\tilde Mds=
	\int_0^{l/3}
	-\frac{2}{3}q_0l^2\left(\frac{s}{l}+\frac{1}{3}\right)\left(-s\right)ds
	=\frac{5}{243}q_0l^4
\end{equation}
\begin{equation}
	\int_A^CM_2\tilde Mdx=
	\int_0^{l/3}M_2\tilde Mds=
	\left(R_Bs \right)
	\left(-s \right)
	ds
	=-\frac{R_Bl^3}{81}
\end{equation}
となるので,
\begin{equation}
	\delta_1=\frac{5}{243}\frac{q_0l^4}{EI}, \ \ 
	\delta_2=-\frac{R_Bl^3}{81EI}
\end{equation}
が得られる.そこで,$\delta_1+\delta_2=0$とすることで,B点における反力$R_B$が
\begin{equation}
	R_B=\frac{5}{3}q_0l
\end{equation}
と決まる.これを式(\ref{eqn:M1_2})と式(\ref{eqn:M2_2})に代入して
$M_1+M_2$を計算することで,問題で与えられた梁の曲げモーメント分布を得ることができる.
その結果を曲げモーメント図として示せば,図\ref{fig:fig3}のようになる.
%%%%%%%%%%%%%%%%%%%%%%%%%%%%
\begin{figure}
	\vspace{10mm}
	\begin{center}
	\includegraphics[width=0.6\linewidth]{fig2ans.eps} 
	\end{center}
	\caption{問題2で与えられた系の2つの系1と2への分解と曲げモーメント図. } 
	\label{fig:fig2}
\end{figure}
\begin{figure}
	\vspace{10mm}
	\begin{center}
	\includegraphics[width=0.5\linewidth]{fig3ans.eps} 
	\end{center}
	\caption{曲げモーメント図(問題2). } 
	\label{fig:fig3}
\end{figure}
\end{document}
