\documentclass[10pt,a4j]{jarticle}
\usepackage{graphicx,wrapfig}
%\usepackage{showkeys}
\setlength{\topmargin}{-1.5cm}
%\setlength{\textwidth}{15.5cm}
\setlength{\textheight}{25.2cm}
\newlength{\minitwocolumn}
\setlength{\minitwocolumn}{0.5\textwidth}
\addtolength{\minitwocolumn}{-\columnsep}
%\addtolength{\baselineskip}{-0.1\baselineskip}
%
\begin{document}
\def\Mmaru#1{{\ooalign{\hfil#1\/\hfil\crcr
\raise.167ex\hbox{\mathhexbox 20D}}}}


\newcommand{\fat}[1]{\mbox{\boldmath $#1$}}
\newcommand{\D}{\partial}

\newcommand{\w}{\omega}
\newcommand{\ga}{\alpha}
\newcommand{\gb}{\beta}
\newcommand{\gx}{\xi}
\newcommand{\gz}{\zeta}
\newcommand{\vhat}[1]{\hat{\fat{#1}}}
\newcommand{\spc}{\vspace{0.7\baselineskip}}
\newcommand{\halfspc}{\vspace{0.3\baselineskip}}
\bibliographystyle{unsrt}
%\pagestyle{empty}
\newcommand{\twofig}[2]
 {
   \begin{figure}[h]
     \begin{minipage}[t]{\minitwocolumn}
         \begin{center}   #1
         \end{center}
     \end{minipage}
         \hspace{\columnsep}
     \begin{minipage}[t]{\minitwocolumn}
         \begin{center} #2
         \end{center}
     \end{minipage}
   \end{figure}
 }
%%%%%%%%%%%%%%%%%%%%%%%%%%%%%%%%%
%\vspace*{\baselineskip}
\begin{center}
{\Large \bf 2020年度 構造力学II レポート課題1解答} \\
\end{center}
%%%%%%%%%%%%%%%%%%%%%%%%%%%%%%%%%%%%%%%%%%%%%%%%%%%%%%%%%%%%%%%%
\subsubsection*{問題1.}
図\ref{fig:fig1}に問題で与えられた系と単位荷重法における補助系を示す.
いずれも静定構造であり,曲げモーメント分布は釣り合い条件から求めることが
できる.同図の下段は釣り合い条件から求めた曲げモーメントを図示したもので,
曲げモーメント$M$は区間CDで,$\tilde M$は区間ABで零である.
従って,単位荷重法の計算で必要となるのは区間BCにおける曲げモーメントで,
区間BCのおける$M$と$\tilde M$は,座標$s$を用いて次のように表すことができる.
\begin{equation}
	M(s)=-\frac{q_0l}{2}(l-s), \ \ \tilde M(s)=-\tilde P s, \ \ (0<s<l)
\end{equation}
これより,
\begin{equation}
	\int_A^D M\tilde Mds= \int_B^CM \tilde Mds=\frac{q_0l^4}{12}
\end{equation}
となり,従って$v_D$が
\begin{equation}
	v_D=\frac{1}{12}\frac{q_0l^4}{EI}
\end{equation}
と求められる.
%%%%%%%%%%%%%%%%%
\begin{figure}[h]
	\begin{center}
	\includegraphics[width=1.0\linewidth]{fig1ans.eps} 
	\end{center}
	\caption{問題1で与えられた系と単位荷重法における補助系の曲げモーメント図. } 
	\label{fig:fig1}
\end{figure}
\subsubsection*{問題2.}
問題で与えられた梁を,図\ref{fig:fig2}に示す2つの静定系(系1,系2)の重ね合わせで表現する.
ただし,この図の$R_B,R_C$および$R_D$は鉛直方向の支点反力を意味する.
ここで,系1の点Dにおけるたわみを$v^{(1)}_D$,
系2の点Dにおけるたわみを$v^{(2)}_D$とすれば,
これらが満足すべき適合条件は
\begin{equation}
	v^{(1)}_D
	+
	v^{(2)}_D
	=0
	\label{eqn:vD_vanish}
\end{equation}
である.$v_D^{(1)}$は問題1で$\frac{1}{12}\frac{q_0l^4}{EI}$となることが示されている.
単位荷重法による$v^{(2)}_D$の計算には,問題1と同じ補助系(\ref{fig:fig1}-(b))を用いる
ことができる.また系2の曲げモーメント$M$は,図\ref{fig:fig1}-(b)において
$\tilde P=-R_D$と置くことで得られる.以上を踏まえて曲げモーメントに関する積分
$\int_A^DM\tilde M ds $を計算すると,
\begin{equation}
	\int_A^D M\tilde Mds= 2\int_B^CM \tilde Mds=-\frac{2}{3}R_Dl^3
\end{equation}
より,
\begin{equation}
	v_D^{(2)}=\frac{2}{3}\frac{q_0l^4}{EI}
\end{equation}
となる.これを式(\ref{eqn:vD_vanish})に代入すれば,
\begin{equation}
	\frac{1}{12}\frac{q_0l^4}{EI}
	-\frac{2}{3}\frac{R_Dl^3}{EI}
	=0 \ \ \Rightarrow \ \ 
	R_D=\frac{q_0l}{8}
\end{equation}
となることが分かる.$R_D$が既知であれば,梁全体の釣り合い条件から残りの支点反力
が求められ,その結果は次のようになる.
\begin{equation}
	R_B=\frac{13}{8}q_0l, \ \ R_C=-\frac{3}{4}q_0l
\end{equation}
最後に,釣り合い条件から曲げモーメントを求めると,
図\ref{fig:fig3}のようになることが示される.
%%%%%%%%%%%%%%%%%%%%%%%%%%%%
\begin{figure}
	\vspace{10mm}
	\begin{center}
	\includegraphics[width=0.6\linewidth]{fig2ans.eps} 
	\end{center}
	\caption{問題2で与えられた系の2つの静定系1と2への分解. } 
	\label{fig:fig2}
\end{figure}
\begin{figure}
	\vspace{10mm}
	\begin{center}
	\includegraphics[width=0.5\linewidth]{fig3ans.eps} 
	\end{center}
	\caption{曲げモーメント図(問題2). } 
	\label{fig:fig3}
\end{figure}
\end{document}
