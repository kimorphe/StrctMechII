\documentclass[10pt,a4j]{jarticle}
\usepackage{graphicx,wrapfig}
\usepackage{showkeys}
\setlength{\topmargin}{-1.5cm}
%\setlength{\textwidth}{15.5cm}
\setlength{\textheight}{25.2cm}
\newlength{\minitwocolumn}
\setlength{\minitwocolumn}{0.5\textwidth}
\addtolength{\minitwocolumn}{-\columnsep}
%\addtolength{\baselineskip}{-0.1\baselineskip}
%
\def\Mmaru#1{{\ooalign{\hfil#1\/\hfil\crcr
\raise.167ex\hbox{\mathhexbox 20D}}}}
%
\begin{document}
\newcommand{\fat}[1]{\mbox{\boldmath $#1$}}
\newcommand{\D}{\partial}
\newcommand{\w}{\omega}
\newcommand{\ga}{\alpha}
\newcommand{\gb}{\beta}
\newcommand{\gx}{\xi}
\newcommand{\gz}{\zeta}
\newcommand{\vhat}[1]{\hat{\fat{#1}}}
\newcommand{\spc}{\vspace{0.7\baselineskip}}
\newcommand{\halfspc}{\vspace{0.3\baselineskip}}
\bibliographystyle{unsrt}
\pagestyle{empty}
\newcommand{\twofig}[2]
 {
   \begin{figure}[h]
     \begin{minipage}[t]{\minitwocolumn}
         \begin{center}   #1
         \end{center}
     \end{minipage}
         \hspace{\columnsep}
     \begin{minipage}[t]{\minitwocolumn}
         \begin{center} #2
         \end{center}
     \end{minipage}
   \end{figure}
 }
%%%%%%%%%%%%%%%%%%%%%%%%%%%%%%%%%
%\vspace*{\baselineskip}
\begin{center}
	{\Large \bf 2020年度 構造力学II レポート課題2 解答} \\
\end{center}
%%%%%%%%%%%%%%%%%%%%%%%%%%%%%%%%%%%%%%%%%%%%%%%%%%%%%%%%%%%%%%%%
\subsubsection*{問題 1}
	問題で与えられた骨組み構造に対し, 支点反力の正方向を図\ref{fig:fig1}-(a)のように定める.
	これらの反力は,構造全体の力とモーメントのつり合い条件より
	\[
		R_A=\frac{q_0l}{4}, \ \ R_D=-\frac{q_0l}{4}, \ \ H_A=q_0l
	\]
	となる.ここで,部材$i, (i=1,2,3)$の軸力, 曲げモーメント, せん断力を
	$N_i, M_i$および$Q_i$と表す.これらの断面力を求めるための自由物体図は, 
	図\ref{fig:fig2}-(a)から(c)のようになる.
	この図を参照して 軸力とモーメントのつり合い式を立てれば,
	\[
		N_1=-\frac{5}{4\sqrt{2}}q_0l, \ \ 
		M_1=-\frac{3}{4\sqrt{2}}q_0lx_1
	\]
	\[
		N_2=-q_0l, \ \ 
		M_2=-\frac{q_0l^2}{2}-\frac{1}{4}q_0ls_2
	\]
	\[
		N_3=\frac{q_0l}{4}, \ \ 
		M_3=-\frac{1}{2}q_0s_3^2
	\]
	が得られる.この結果から軸力および曲げモーメント図を描くと, 
	図\ref{fig:fig4}-(a)と(b)のようになる.
\begin{figure}[h]
	\begin{center}
	\includegraphics[width=1.0\linewidth]{fig1ans.eps} 
	\end{center}
	\caption{支点反力の正方向.(a)問題1で与えられた系,(b)単位荷重法の計算に用いる補助系.} 
	\label{fig:fig1}
\end{figure}
\begin{figure}[h]
	\begin{center}
	\includegraphics[width=0.7\linewidth]{fig2ans.eps} 
	\end{center}
	\caption{部材1から3のそれぞれにおける断面力の正方向(問題1).}
	\label{fig:fig2}
\end{figure}

	単位荷重法における補助系図\ref{fig:fig1}-(b)のように設定し,
	この構造の支点反力を求めれば,
	\[
		\tilde R_A= \tilde R_D= \frac{1}{2}, \ \ \tilde H_D=0
	\]
	である.また, 図\ref{fig:fig3}-(a)から(c)の自由物体図
	を参照して断面力を計算すれば
	\[
		\tilde N_1=-\frac{1}{2\sqrt{2}}, \ \ \tilde M_1=\frac{x_1}{2\sqrt{2}}
	\]
	\[
		\tilde N_2=0, \ \ \tilde M_2=\frac{s_2}{2}
	\]
	\[
		\tilde N_3=-\frac{1}{2}, \ \ \tilde M_3=0
	\]
	となることから,補助系の軸力と曲げモーメント図は, 図\ref{fig:fig4}-(c)と(d)のようになる.
	ここで, $N_i\tilde N_i$と$M_i\tilde M_i$の部材$i$に関する積分を
	$i=1,2$と3に対して求めると,
	\[
		\int _0^{\sqrt{2}l} N_1 \tilde N_1dx_1=\frac{5\sqrt{2}}{16}q_0l^2
		, \ \ 
		\int _0^{\sqrt{2}l} M_1 \tilde M_1dx_1=-\frac{\sqrt{2}q_0l^4}{8}
	\]
	\[
		\int _0^l N_2 \tilde N_2ds_2=0, \ \ 
		\int _0^l M_2 \tilde M_2ds_2=-\frac{q_0l^4}{6}
	\]
	\[
		\int _0^l N_3 \tilde N_3ds_3=-\frac{q_0l^4}{8}, \ \ 
		\int _0^l M_3 \tilde M_3ds_3=0
	\]
	の結果が得られる.
	以上より,求めるべき鉛直変位$v_B$は
	\[
		v_B=
		\sum_{i=1}^3
		\int_0^{l_i} \left(
			\frac{N_i\tilde N_i}{EA}
			+
			\frac{M_i\tilde M_i}{EI}
		\right) dx_i
		=
		\frac{5\sqrt{2}-2}{16}\frac{q_0l^2}{EA}
		-
		\frac{4+3\sqrt{2}}{24}
		\frac{q_0l^4}{EI}
	\]
	である.ただし$l_i$と$x_i$は,それぞれ部材$i$の長さと
	部材軸方向にとった座標を表す.
%--------------------
\begin{figure}[h]
	\begin{center}
	\includegraphics[width=0.7\linewidth]{fig3ans.eps} 
	\end{center}
	\caption{部材1から3のそれぞれにおける断面力の正方向(問題1の補助系).}
	\label{fig:fig3}
\end{figure}
\begin{figure}[h]
	\begin{center}
	\includegraphics[width=1.0\linewidth]{fig4ans.eps} 
	\end{center}
	\caption{軸力および曲げモーメント図(問題1).}
	\label{fig:fig4}
\end{figure}
\clearpage
%%%%%%%%%%%%%%%%%%%%%%%%%%%%%%%%%%%%%%%%%%%%%%%%%%%%%%%%%
\subsubsection*{問題 2}
図\ref{fig:fig5}-(a)のように支点反力の正方向を定めるとき, 
これらの支点反力はトラス構造全体のつり合い条件より, 
\[
	H_A=0, \ \ V_A=V_B=\frac{P}{2}
\]
となる.ここで,部材番号を\ref{fig:fig5}-(a)のように定め,第$i$部材の軸力を$N_i$と表す.
トラス構造と荷重条件がともに左右対称であることから,軸力は
\[
	N_i=N_{i'}, \ \ (i=1,\dots 6, \, i\neq 2)
\]
であるため,以下では$N_1 \dots N_6$を求める.

はじめに,節点Aに関する力の釣り合を考えると, 図\ref{fig:fig5}-(b)より
\[
	N_1=\frac{P}{2}, \ \ N_3=-\frac{P}{\sqrt{2}}
\]
となることが分かる.次に,同図(c)のような部分構造に着目し,
節点Fに関するモーメントの釣り合い条件を立てれば,
\[
	N_2\times l +V_D\times 2l=0 \ \ \Rightarrow N_2=P 
\]
が得られ,これを踏まえて水平方向と鉛直方向のちからの釣り合いを考えることで,
\[
	N_6=-\frac{P}{2}, \ \ N_5=-\frac{P}{\sqrt{2}}
\]
最後に,図\ref{fig:fig5}-(d)を参照して節点Dに関する力の釣り合いから
$N_4=\frac{P}{2}$が求められる.

一方,単位荷重法のおける補助系は,問題で与えられた系において$P=1$としたものでよく,
軸力$\tilde N_i$も,上で求めた$N_i$に$P=1$を代入して得られる.以上より,
節点Fの鉛直変位が,
\[
	v_F=
	\sum_{i=1}^6\frac{N_i\tilde N_il_i}{EA}
	+
	\sum_{i'=1,i'\neq 2}^6\frac{N_i\tilde N_il_i}{EA}
	=\left(\frac{7}{2}+2\sqrt{2}\right)\frac{Pl}{EA}
\]
と得られる.ただし$l_i$は部材$i$の長さを表し,$l$あるいは$\sqrt{2}l$である.
\begin{figure}[h]
	\begin{center}
	\includegraphics[width=0.8\linewidth]{fig5ans.eps} 
	\end{center}
	\caption{トラス構造の反力および軸力計算に用いた自由物体図(問題2).}
	\label{fig:fig5}
\end{figure}
%%%%%%%%%%%%%%%%%%%%%%%%%%%%%%%%%%%%%%%%%%%%
\end{document}
\twofig{
	\includegraphics[width=0.9\linewidth]{tri_elem.eps} 
	\caption{微小三角形領域ABCと, 各辺に作用する応力および表面力.} 
	\label{fig:tri_elem}
}
{
	\includegraphics[width=0.9\linewidth]{N_plane.eps} 
	\caption{点$\fat{x}$を通り, $x$軸に対して$\theta$だけ傾いた面とその法線$\fat{N}(\theta)$および接線ベクトル$\fat{T}(\theta)$.} 
	\label{fig:N_plane}
}
\begin{figure}[h]
	\vspace{10mm}
	\begin{center}
	\includegraphics[width=0.7\linewidth]{fig1.eps} 
	\end{center}
	\caption{問題1で用いる座標系.(i)$xy$および$x'y'$座標系, (ii)$xy$座標系と$xy''$座標系. } 
