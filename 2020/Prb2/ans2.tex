\documentclass[10pt,a4j]{jarticle}
\usepackage{graphicx,wrapfig}
\setlength{\topmargin}{-1.5cm}
\setlength{\textwidth}{15.5cm}
\setlength{\textheight}{25.2cm}
\newlength{\minitwocolumn}
\setlength{\minitwocolumn}{0.5\textwidth}
\addtolength{\minitwocolumn}{-\columnsep}
%\addtolength{\baselineskip}{-0.1\baselineskip}
%
\def\Mmaru#1{{\ooalign{\hfil#1\/\hfil\crcr
\raise.167ex\hbox{\mathhexbox 20D}}}}
%
\begin{document}
\newcommand{\fat}[1]{\mbox{\boldmath $#1$}}
\newcommand{\D}{\partial}
\newcommand{\w}{\omega}
\newcommand{\ga}{\alpha}
\newcommand{\gb}{\beta}
\newcommand{\gx}{\xi}
\newcommand{\gz}{\zeta}
\newcommand{\vhat}[1]{\hat{\fat{#1}}}
\newcommand{\spc}{\vspace{0.7\baselineskip}}
\newcommand{\halfspc}{\vspace{0.3\baselineskip}}
\bibliographystyle{unsrt}
\pagestyle{empty}
\newcommand{\twofig}[2]
 {
   \begin{figure}[h]
     \begin{minipage}[t]{\minitwocolumn}
         \begin{center}   #1
         \end{center}
     \end{minipage}
         \hspace{\columnsep}
     \begin{minipage}[t]{\minitwocolumn}
         \begin{center} #2
         \end{center}
     \end{minipage}
   \end{figure}
 }
%%%%%%%%%%%%%%%%%%%%%%%%%%%%%%%%%
%\vspace*{\baselineskip}
\begin{center}
	{\Large \bf 2019年度 構造力学II 演習課題2 解答} \\
\end{center}
%%%%%%%%%%%%%%%%%%%%%%%%%%%%%%%%%%%%%%%%%%%%%%%%%%%%%%%%%%%%%%%%
\subsubsection*{問題 1}
	問題で与えられた骨組み構造に対し,
	支点反力の正方向を, 図\ref{fig:fig1}-(a)に示したように定める.
	これらの反力は,構造全体の力とモーメントのつり合い条件より
	\[
		R_A=R_D=\frac{q_0l}{2}, \ \ H_A=q_0l
	\]
	となる.部材$i, (i=1,2,3)$の軸力, 曲げモーメント, せん断力を
	$N_i, M_i$および$Q_i$と表すとき, 部材1から3の断面力を求める
	ための自由物体図は, 図\ref{fig:fig2}-(a)から(c)のようになる.
	これらの図を参照して, 軸力とモーメントのつり合い式を立てれば,
	\[
		N_1=\frac{q_0l}{2\sqrt{2}}, \ \ 
		M_1=-\frac{q_0l^2}{2}
			\left\{\left(\frac{x_1}{l}\right)^2
			-\frac{3}{\sqrt{2}}\left(\frac{x_1}{l}\right)\right\}
	\]
	\[
		N_2=0, \ \ 
		M_2=\frac{q_0l^2}{2}\frac{s_2}{l}	
	\]
	\[
		N_3=-\frac{q_0l}{2}, \ \ 
		M_3=0
	\]
	が得られる.この結果から軸力および曲げモーメント図を描くと, 図\ref{fig:fig4}-(a)と(b)のようになる.

	単位荷重法における補助系図\ref{fig:fig1}-(b)のように設定し,
	この構造の支点反力を求めれば,
	\[
		\tilde R_A= \tilde R_D= \frac{1}{2}, \ \ \tilde H_D=0
	\]
	である.また, 図\ref{fig:fig3}-(a)から(c)の自由物体図
	を参照して断面力を計算すれば
	\[
		\tilde N_1=-\frac{1}{2\sqrt{2}}, \ \ \tilde M_1=\frac{x_1}{2\sqrt{2}}
	\]
	\[
		\tilde N_2=0, \ \ \tilde M_2=\frac{s_2}{2}
	\]
	\[
		\tilde N_3=-\frac{1}{2}, \ \ \tilde M_3=0
	\]
	となることから,補助系の曲げモーメント図は, 図\ref{fig:fig3}-(b)のようになる.
	ここで, $N_i\tilde N_i$と$M_i\tilde M_i$の部材$i$に関する積分を
	$i=1,2$と3に対して求めると,
	\[
		\int _0^{\sqrt{2}l} N_1 \tilde N_1dx_1=-\frac{\sqrt{2}}{8}q_0l^2
		, \ \ 
		\int _0^{\sqrt{2}l} M_1 \tilde M_1dx_1=\frac{q_0l^4}{4\sqrt{2}}
	\]
	\[
		\int _0^l N_2 \tilde N_2ds_2=0, \ \ 
		\int _0^l M_2 \tilde M_2ds_2=\frac{q_0l^4}{12}
	\]
	\[
		\int _0^l N_3 \tilde N_3ds_3=\frac{q_0l^4}{4}, \ \ 
		\int _0^l M_3 \tilde M_3ds_3=0
	\]
	の結果が得られる.
	以上より,求めるべき鉛直変位$v_B$は
	\[
		v_B=
		\sum_{i=1}^3
		\int_0^{l_i} \left(
			\frac{N_i\tilde N_i}{EA}
			+
			\frac{M_i\tilde M_i}{EI}
		\right) dx_i
		=
		\frac{2-\sqrt{2}}{8}\frac{q_0l^2}{EA}
		+
		\frac{2+3\sqrt{2}}{24}
		\frac{q_0l^4}{EI}
	\]
	である.ただし$l_i$と$x_i$は,それぞれ部材$i$の長さと
	部材軸方向にとった座標を表す.
%--------------------
\begin{figure}[h]
	\begin{center}
	\includegraphics[width=1.0\linewidth]{fig1ans.eps} 
	\end{center}
	\caption{支点反力の正方向.(a)問題1で与えられた系,(b)単位荷重法の計算に用いる補助系.} 
	\label{fig:fig1}
\end{figure}
\begin{figure}[h]
	\begin{center}
	\includegraphics[width=0.7\linewidth]{fig2ans.eps} 
	\end{center}
	\caption{部材1から3のそれぞれにおける断面力の正方向(問題1).}
	\label{fig:fig2}
\end{figure}
\begin{figure}[h]
	\begin{center}
	\includegraphics[width=0.7\linewidth]{fig3ans.eps} 
	\end{center}
	\caption{部材1から3のそれぞれにおける断面力の正方向(問題1の補助系).}
	\label{fig:fig3}
\end{figure}
%%%%%%%%%%%%%%%%%%%%%%%%%%%%%%%%%%%%%%%%%%%%%%%%%%%%%%%%%
\subsubsection*{問題 2}
トラス構造の節点BとCにそれぞれ大きさ$F_1$および$F_2$の荷重が
作用する場合の軸力を求める.
その結果において$F_1=F_2=F$とすれば,問題で与えられたトラス構造の軸力が得られる.
一方,$F_1=0,F_2=1$とすれば,問題で指定された鉛直変位$v_C$を, 
単位荷重法によって求めるための補助系の軸力が得られる.\\

図\ref{fig:fig5}-(a)のように支点反力の正方向を定めるとき, 
これらの支点反力はトラス構造全体のつり合い条件より, 
\[
	H_A=-F_1-2F_2, \ \ V_A=F_1+F_2
	\ \ H_B=F_1+2F_2
\]
となる.ここで,第$i$部材の軸力を$N_i$と表し,節点Cに
関する力の釣り合を考えると, 図\ref{fig:fig5}-(b)より
\[
	N_2=F_2, \ \ N_6=-\sqrt{2}F_2
\]
となる.次に,図\ref{fig:fig5}-(c)のような部分構造について,
点Bに関するモーメントの釣り合い条件を考えれば,
\[
	N_7=-F_2
\]
となることが分かる.さらに,この部分構造の水平方向と鉛直方向の
力の釣り合いから
\[
	N_1=F_2+2F_2, \ \ N_4=-\sqrt{2}\left( F_1+2F_2 \right)
\]
が得られる.
以上を踏まえ,節点Dと節点Eについて
図\ref{fig:fig4}-(d)および(e)に従いつり合い式を
立てれば,
\[
	N_5=F_2, \ \ N_3=F_1+F_2 
\]
が得られる.
以上の結果から,問題で与えられたトラス構造の軸力が
\[
	N_1=3F, \ \ N_2=F, \ \ N_3=2F, \ \ N_4=-2\sqrt{2}F
\]
\[
	N_5=F, \ \ N_6=-\sqrt{2}F, \ \ N_7=-F
\]
と,補助系の軸力$\tilde N_i (i=1\sim 7)$が
\[
	\tilde N_1=2, \ \ \tilde N_2=1, \ \ \tilde N_3=1, \ \ \tilde N_4=-\sqrt{2}
\]
\[
	\tilde N_5=1, \ \ \tilde N_6=-\sqrt{2}, \ \ \tilde N_7=-1
\]
と求められる.これらを単位荷重法に用いることで,C点の鉛直変位$v_C$が
\[
	v_C=\sum_{i=1}^7 \frac{N_i \tilde N_i}{EA}l_i =
	\left( 11+6\sqrt{2}\right)
	\frac{Fl}{EA}.
\]
となることがわかる.
\begin{figure}[h]
	\begin{center}
	\includegraphics[width=1.0\linewidth]{fig4ans.eps} 
	\end{center}
	\caption{軸力および曲げモーメント図(問題1).}
	\label{fig:fig4}
\end{figure}
\begin{figure}[h]
	\begin{center}
	\includegraphics[width=0.8\linewidth]{fig5ans.eps} 
	\end{center}
	\caption{トラス構造の反力および軸力計算に用いた自由物体図(問題2).}
	\label{fig:fig5}
\end{figure}
%%%%%%%%%%%%%%%%%%%%%%%%%%%%%%%%%%%%%%%%%%%%
\end{document}
\twofig{
	\includegraphics[width=0.9\linewidth]{tri_elem.eps} 
	\caption{微小三角形領域ABCと, 各辺に作用する応力および表面力.} 
	\label{fig:tri_elem}
}
{
	\includegraphics[width=0.9\linewidth]{N_plane.eps} 
	\caption{点$\fat{x}$を通り, $x$軸に対して$\theta$だけ傾いた面とその法線$\fat{N}(\theta)$および接線ベクトル$\fat{T}(\theta)$.} 
	\label{fig:N_plane}
}
\begin{figure}[h]
	\vspace{10mm}
	\begin{center}
	\includegraphics[width=0.7\linewidth]{fig1.eps} 
	\end{center}
	\caption{問題1で用いる座標系.(i)$xy$および$x'y'$座標系, (ii)$xy$座標系と$xy''$座標系. } 
