\documentclass[10pt,a4j]{jarticle}
\usepackage{graphicx,wrapfig}
\setlength{\topmargin}{-1.5cm}
\setlength{\textwidth}{15.5cm}
\setlength{\textheight}{25.2cm}
\newlength{\minitwocolumn}
\setlength{\minitwocolumn}{0.5\textwidth}
\addtolength{\minitwocolumn}{-\columnsep}
%\addtolength{\baselineskip}{-0.1\baselineskip}
%
\def\Mmaru#1{{\ooalign{\hfil#1\/\hfil\crcr
\raise.167ex\hbox{\mathhexbox 20D}}}}
%
\begin{document}
\newcommand{\fat}[1]{\mbox{\boldmath $#1$}}
\newcommand{\D}{\partial}
\newcommand{\w}{\omega}
\newcommand{\ga}{\alpha}
\newcommand{\gb}{\beta}
\newcommand{\gx}{\xi}
\newcommand{\gz}{\zeta}
\newcommand{\vhat}[1]{\hat{\fat{#1}}}
\newcommand{\spc}{\vspace{0.7\baselineskip}}
\newcommand{\halfspc}{\vspace{0.3\baselineskip}}
\bibliographystyle{unsrt}
%\pagestyle{empty}
\newcommand{\twofig}[2]
 {
   \begin{figure}[h]
     \begin{minipage}[t]{\minitwocolumn}
         \begin{center}   #1
         \end{center}
     \end{minipage}
         \hspace{\columnsep}
     \begin{minipage}[t]{\minitwocolumn}
         \begin{center} #2
         \end{center}
     \end{minipage}
   \end{figure}
 }
%%%%%%%%%%%%%%%%%%%%%%%%%%%%%%%%%
%\vspace*{\baselineskip}
\begin{center}
{\Large \bf 2020年度 構造力学II レポート課題2} \\
\end{center}
\begin{flushright}
	提出期限:2020年5月29日(金)18:00\\
%	提出先:環境理工学部棟3Fレポートボックス
	提出先:{\bf MoodleからPDFファイルで提出}
\end{flushright}
%%%%%%%%%%%%%%%%%%%%%%%%%%%%%%%%%%%%%%%%%%%%%%%%%%%%%%%%%%%%%%%%
\vspace{10mm}
\subsubsection*{問題 1}
図\ref{fig:fig1}のような骨組み構造ABCDが,鉛直部材CDにおいて
水平方向の等分布荷重を受ける.等分布荷重の大きさは部材の単位長さあたり$q_0$であるとき,
点Bに生じる鉛直変位$v_B$を求めよ.なお,部材のヤング率$E$, 断面2次モーメント$I$, 
断面積$A$は全ての部材で同一かつ一定値とする.
%--------------------
\begin{figure}[h]
	\begin{center}
	\includegraphics[width=0.65\linewidth]{fig1.eps} 
	\end{center}
	\caption{問題1で考える骨組構造.} 
	\label{fig:fig1}
\end{figure}
\newpage
\subsubsection*{問題 2}
図\ref{fig:fig2}のようなトラス構造の点Fに, 大きさ$P$の鉛直荷重が加えられている.
このとき,点Fに生じる鉛直変位$v_F$を求めよ.
なお,トラス部材のヤング率は$E$, 断面積は$A$とし,これらはすべての
部材で同一かつ一定値とする.
%--------------------
\begin{figure}[h]
	\begin{center}
	\includegraphics[width=0.8\linewidth]{fig2.eps} 
	\end{center}
	\caption{問題2で考えるトラス構造.} 
	\label{fig:fig2}
\end{figure}
%--------------------

%%%%%%%%%%%%%%%%%%%%%%%%%%%%%%%%%%%%%%%%%%%%
%%%%%%%%%%%%%%%%%%%%%%%%%%%%%%%%%%%%%%%%%%%%
\end{document}
\twofig{
	\includegraphics[width=0.9\linewidth]{tri_elem.eps} 
	\caption{微小三角形領域ABCと, 各辺に作用する応力および表面力.} 
	\label{fig:tri_elem}
}
{
	\includegraphics[width=0.9\linewidth]{N_plane.eps} 
	\caption{点$\fat{x}$を通り, $x$軸に対して$\theta$だけ傾いた面とその法線$\fat{N}(\theta)$および接線ベクトル$\fat{T}(\theta)$.} 
	\label{fig:N_plane}
}
\begin{figure}[h]
	\vspace{10mm}
	\begin{center}
	\includegraphics[width=0.7\linewidth]{fig1.eps} 
	\end{center}
	\caption{問題1で用いる座標系.(i)$xy$および$x'y'$座標系, (ii)$xy$座標系と$xy''$座標系. } 
	\label{fig:fig1}
\end{figure}
