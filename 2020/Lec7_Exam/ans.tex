\documentclass[10pt,a4j]{jarticle}
\usepackage{graphicx,wrapfig}
\setlength{\topmargin}{-1.5cm}
\setlength{\textwidth}{15.5cm}
\setlength{\textheight}{25.2cm}
\newlength{\minitwocolumn}
\setlength{\minitwocolumn}{0.5\textwidth}
\addtolength{\minitwocolumn}{-\columnsep}
%\addtolength{\baselineskip}{-0.1\baselineskip}
%
\def\Mmaru#1{{\ooalign{\hfil#1\/\hfil\crcr
\raise.167ex\hbox{\mathhexbox 20D}}}}
%
\begin{document}
\newcommand{\fat}[1]{\mbox{\boldmath $#1$}}
\newcommand{\D}{\partial}
\newcommand{\w}{\omega}
\newcommand{\ga}{\alpha}
\newcommand{\gb}{\beta}
\newcommand{\gx}{\xi}
\newcommand{\gz}{\zeta}
\newcommand{\vhat}[1]{\hat{\fat{#1}}}
\newcommand{\spc}{\vspace{0.7\baselineskip}}
\newcommand{\halfspc}{\vspace{0.3\baselineskip}}
\bibliographystyle{unsrt}
\pagestyle{empty}
\newcommand{\twofig}[2]
 {
   \begin{figure}[h]
     \begin{minipage}[t]{\minitwocolumn}
         \begin{center}   #1
         \end{center}
     \end{minipage}
         \hspace{\columnsep}
     \begin{minipage}[t]{\minitwocolumn}
         \begin{center} #2
         \end{center}
     \end{minipage}
   \end{figure}
 }
%%%%%%%%%%%%%%%%%%%%%%%%%%%%%%%%%
%\vspace*{\baselineskip}
\begin{center}
{\Large \bf 2019年度 構造力学II 期末試験(解答)} \\
\end{center}
%%%%%%%%%%%%%%%%%%%%%%%%%%%%%%%%%%%%%%%%%%%%%%%%%%%%%%%%%%%%%%%%
\subsubsection*{問題1.}
\begin{enumerate}
\item
	支点反力の正方向を\ref{fig:fig1}-(a)のようにとる.これらは
	力とモーメントの釣り合い条件から.
	\[
		R_B=\left(1+\frac{a}{l}\right)F, \ \ R_C=-\frac{a}{l}F, \ \ H_B=0
	\]
	となる.
\item
	区間AB, BCおよびCDにおける曲げモーメントを, 順に$M_1,M_2,M_3$とすれば,
	これらは
	\[
		M_1(x)=-F x, \ \ M_2(s)=-Fa \frac{s}{l}, \ \ M_3=0
	\]
	となり,曲げモーメント図は図\ref{fig:fig1}-(b)のようになる.
\item
	図\ref{fig:fig1}-(a)において$F=1$としたものを補助系として単位荷重法を用いればよい.
	補助系の曲げモーメントを$\tilde M_i, (i=1,2,3)$とすれば, 
	\begin{eqnarray*}
	&& \int_A^B M_1\tilde M_1 dx=\frac{Fa^3}{3} \\  
		&& \int_C^B M_2 \tilde M_2 ds = \frac{F}{3}a^2l \\
	&& \int_D^C M_3 \tilde M_3 ds =0
	\end{eqnarray*}
	となるので,点Aのたわみ$v_A$は
	\[
		v_A= \frac{F}{3EI}a^2(a+l)
	\]
	と求まる.
\item
	図\ref{fig:fig2}を補助系として単位荷重法を適用すればよい.
	補助系の曲げモーメントを区間AB,BC,CDの順に$\tilde M_1, \tilde M_2, \tilde M_3$
	とすれば,
	\[
		\tilde M_1=0, \ \ 
		\tilde M_2(s)=-\tilde Fb \left(1-\frac{s}{l}\right), \ \ 
		\tilde M_3(x)=-\tilde{F} t
	\]
	となることから,
	\begin{eqnarray*}
		&& \int_C^B M_2\tilde M_2 ds=\frac{F}{6}abl\\  
	&& \int_A^BM_1 \tilde M_1dx=\int_D^C M_3 \tilde M_3 ds =0
	\end{eqnarray*}
	より,点Dのたわみ$v_D$は
	\[
		v_D=\frac{Fabl}{6EI}
	\]
	となる.
\item
	点Dで伝達される鉛直力を$R_D$とし,問題で与えられた系の区間ADとDGに関する
	自由物体図を描くと,図\ref{fig:fig3}のようになる.
	点Dに生じる鉛直変位を$v_D$とする.$v_D$は,区間ADの側からみれば,点Aに作用する
		荷重$F$に起因する成分$v_D^F$と, $R_D$に起因する成分$v_D^{R+}$の和として
	\[
		v_D=v_D^F+v_D^{R+}
	\]
	と表される.一方,区間DGの側からみれば,$v_D$はD点で伝達される鉛直力$R_D$のみによって
	生じると言え,この視点を明確にするために,
	\[
		v_D=v_D^{R-}
	\]
	と書くことにする.
	$v_D^{R-}$は,問3の結果において$a=b=\frac{l}{2}, F=R_D$とすることで,
	\[
		v_D^F=\frac{R_Dl^3}{8EI}
	\]
	と得られ, $v_D^{R+}$との関係は
	\[
		v_D^{R+}=-v_D^{R-}
	\]
	である.一方,$v_D^F$は,問4の結果において$a=b=\frac{l}{2}$とすることで
	\[
		v_D^F=\frac{Fl^3}{24EI}
		\]
	と得られる.
	\[
		v_D= v_D^F+v_D^{R+}=v_D^{R}
	\]
	だから,
	\[
		R_D=\frac{F}{6}
	\]
	と$R_D$が決定できる.
\end{enumerate}
%--------------------
\begin{figure}[h]
	\begin{center}
	\includegraphics[width=0.45\linewidth]{fig1ans.eps} 
	\end{center}
	\caption{支点反力の正方向と,曲げモーメント図(問題1).}
	\label{fig:fig1}
\end{figure}
%--------------------
%--------------------
\begin{figure}[h]
	\begin{center}
	\includegraphics[width=0.45\linewidth]{fig2ans.eps} 
	\end{center}
	\caption{単位荷重法における補助系とその曲げモーメント図(問題1).}
	\label{fig:fig2}
\end{figure}
%--------------------
\begin{figure}
	\begin{center}
	\includegraphics[width=0.9\linewidth]{fig3ans.eps} 
	\end{center}
	\caption{ヒンジ部で伝達される鉛直力$R_D$と,2つの部分構造への分割(問題1).}
	\label{fig:fig3}
\end{figure}
%--------------------
%%%%%%%%%%%%%%%%%%%%%%%%%%%%%%%%%%%%%%%%%%%%
\subsubsection*{問題2.}
部材1から3における断面力の正方向を図\ref{fig:fig4}のように取る.
\begin{enumerate}
\item
	支点反力の正方向を図\ref{fig:fig5}-(a)のようにとれば,
	\[
		R_A=\frac{q_0l}{4}, \ \ R_D=-\frac{q_0l}{4}, \ \ H_A=0
	\]
	となる.
\item
	部材ABの軸力と曲げモーメントを$N_1,M_1$とする.これらは,図\ref{fig:fig5}-(a)
	のような座標$x_1$を用いれば,
	\[
		N_1=-\frac{q_0l}{4}, \ \ M_1=-\frac{q_0x_1^2}{2}
	\]
	となり,断面力図は同図(b)と(c)のようになる.
\item
	部材BCの軸力と曲げモーメントを$N_2,M_2$とする.
	これらは,図\ref{fig:fig5}-(a)のような座標$s_2$を用いれば,
	\[ 
		N_2=-q_0l, \ \ M_2=-\frac{q_0l}{4}(l+s_2)
	\]
	となり,断面力図は同図(b)と(c)のようになる.
\item
	部材CDの軸力と曲げモーメントを$N_3,M_3$とする.
	これらは,図\ref{fig:fig5}-(a)のような座標$s_3$を用いれば,
	\[ 
		N_3=\frac{q_0l}{4}, \ \ M_3=-q_0s_3^2
	\]
	となり,断面力図は同図(b)と(c)のようになる.
\item
	単位荷重法により水平変位$u_C$を求めるための補助系を,
	図\ref{fig:fig6}-(a)に示す.
	支点反力の正方向をこの図にあるように定めるとき,その大きさは
	\[
		\tilde H_A=-1, \ \ \tilde R_A=-1, \ \  \tilde R_D=1
	\]
	となる.また,軸力と曲げモーメントを, 部材AB,BC,CDの順に
	$\tilde N_i, \tilde M_i,(i=1,2,3)$と表すことにすれば,
	図\ref{fig:fig6}-(a)に示した座標$x_1,s_2,s_3$用いて
	断面力分布を以下のように表すことができる.
	\[
		\tilde N_1=1, \ \ \tilde M_1=x_1
	\]
	\[
		\tilde N_2=1, \ \ \tilde M_2=s_2
	\]
	\[
		\tilde N_3=-1, \ \ \tilde M_3=0
	\]
	これを断面力図として示せば,\ref{fig:fig6}-(b)と(c)のようになる.
	次に,単位荷重法の適用において必要とされる積分を行うと,
	\[
		\int _0^l N_1\tilde N_1dx_1=-\frac{q_0l^2}{4}, \ \ 
		\int _0^l N_2\tilde N_2ds_2=-q_0l^2, \ \ 
		\int _0^{l/2} N_3\tilde N_3ds_3=-\frac{q_0l^2}{8} 
	\]
	\[
		\int_0^l M_1\tilde M_1dx_1=-\frac{q_0l^4}{8}, \ \ 
		\int_0^l M_2\tilde M_2ds_2=-\frac{5q_0l^4}{24},\ \ 
		\int_0^l M_3\tilde M_3ds_3=0
	\]
	となることから,
	求めるべき水平変位$u_C$が
	\[
		u_C=
		\sum_{i=1}^3 \int_0^{l_i}
		\left(
		\frac{N_i\tilde N_i}{EA}+\frac{M_i\tilde M_i}{EI}
		\right)dx_i
		=
		-\frac{11}{8}
		\frac{q_0l^2}{EA}
		-
		\frac{q_0l^4}{3EI}
	\]
	と得られる.ただし,$l_1=l_2=l, l_3=l/2, dx_i=-ds_i,(i=1,2,3) $とした.
\end{enumerate}
%--------------------
%--------------------
\begin{figure}[h]
	\begin{center}
	\includegraphics[width=0.7\linewidth]{fig4ans.eps} 
	\end{center}
	\caption{骨組み構造ABCDの部材1から3における断面力の正方向.}
	\label{fig:fig4}
\end{figure}
\begin{figure}[h]
	\begin{center}
	\includegraphics[width=1.0\linewidth]{fig5ans.eps} 
	\end{center}
	\caption{骨組み構造ABCDに働く支点反力と軸力および曲げモーメント図.}
	\label{fig:fig5}
\end{figure}
\begin{figure}[h]
	\begin{center}
	\includegraphics[width=1.0\linewidth]{fig6ans.eps} 
	\end{center}
	\caption{
	単位荷重法における補助系とその軸力および断面力図.
	断面力の正方向は図\ref{fig:fig4}に同じ.}
	\label{fig:fig6}
\end{figure}
%--------------------
\end{document}
%--------------------
%%%%%%%%%%%%%%%%%%%%%%%%%%%%%%%%%%%%%%%%%%%%
\end{document}
\twofig{
	\includegraphics[width=0.9\linewidth]{tri_elem.eps} 
	\caption{微小三角形領域ABCと, 各辺に作用する応力および表面力.} 
	\label{fig:tri_elem}
}
{
	\includegraphics[width=0.9\linewidth]{N_plane.eps} 
	\caption{点$\fat{x}$を通り, $x$軸に対して$\theta$だけ傾いた面とその法線$\fat{N}(\theta)$および接線ベクトル$\fat{T}(\theta)$.} 
	\label{fig:N_plane}
}
\begin{figure}[h]
	\vspace{10mm}
	\begin{center}
	\includegraphics[width=0.7\linewidth]{fig1.eps} 
	\end{center}
	\caption{問題1で用いる座標系.(i)$xy$および$x'y'$座標系, (ii)$xy$座標系と$xy''$座標系. } 
	\label{fig:fig1}
\end{figure}
