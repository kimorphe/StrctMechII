\documentclass[10pt,a4j]{jarticle}
\usepackage{graphicx,wrapfig}
\usepackage{showkeys}
\setlength{\topmargin}{-1.5cm}
%\setlength{\textwidth}{15.5cm}
\setlength{\textheight}{25.2cm}
\newlength{\minitwocolumn}
\setlength{\minitwocolumn}{0.5\textwidth}
\addtolength{\minitwocolumn}{-\columnsep}
%\addtolength{\baselineskip}{-0.1\baselineskip}
%
\def\Mmaru#1{{\ooalign{\hfil#1\/\hfil\crcr
\raise.167ex\hbox{\mathhexbox 20D}}}}
%
\begin{document}
\newcommand{\fat}[1]{\mbox{\boldmath $#1$}}
\newcommand{\D}{\partial}
\newcommand{\w}{\omega}
\newcommand{\ga}{\alpha}
\newcommand{\gb}{\beta}
\newcommand{\gx}{\xi}
\newcommand{\gz}{\zeta}
\newcommand{\vhat}[1]{\hat{\fat{#1}}}
\newcommand{\spc}{\vspace{0.7\baselineskip}}
\newcommand{\halfspc}{\vspace{0.3\baselineskip}}
\bibliographystyle{unsrt}
\pagestyle{empty}
\newcommand{\twofig}[2]
 {
   \begin{figure}[h]
     \begin{minipage}[t]{\minitwocolumn}
         \begin{center}   #1
         \end{center}
     \end{minipage}
         \hspace{\columnsep}
     \begin{minipage}[t]{\minitwocolumn}
         \begin{center} #2
         \end{center}
     \end{minipage}
   \end{figure}
 }
%%%%%%%%%%%%%%%%%%%%%%%%%%%%%%%%%
%\vspace*{\baselineskip}
\begin{center}
{\Large \bf 2020年度 構造力学II 期末試験(解答)} \\
\end{center}
%%%%%%%%%%%%%%%%%%%%%%%%%%%%%%%%%%%%%%%%%%%%%%%%%%%%%%%%%%%%%%%%
\subsubsection*{問題1.}
\begin{enumerate}
\item
	図\ref{fig:fig1}に示す座標$s$を用いて,
	\begin{equation}
	M(s)=-FsH(s)\left\{
		\begin{array}{cc}
			-Fs & (s\geq 0) \\
			0 & (s<0)
		\end{array}
	\right.
	\label{eqn:}
	\end{equation}
	と表わされる.ただし,$H(s)$は単位ステップ関数を意味する.
\item
	図\ref{fig:fig2}に示す座標$s_1$と$s_2$を用いれば,$M_1,M_2$はそれぞれ
	\begin{equation}
		M_1(s_1)= -F_1s_1H(s_1)
	\end{equation}
	\begin{equation}
		M_2(s_2)= -F_2s_2H(s_2)
	\end{equation}
	と表される. $s_2=s_1+(a_2-a_1)$に注意し,$s_1$を変数として積分を行えば,
	次の結果が得られる.
	\begin{equation}
		J=\int_{s_1=0}^{a_1} M_1(s_1)M_2(s_1)ds_1=\frac{F_1F_2}{6}a_1^2(3a_2-a_1)
		\label{eqn:}
	\end{equation}
\item
	単位荷重法を用いて計算を行う.$v_{11}$の計算では,系2で$F_2=1,a_2=a_1$としたものを補助系に
	とればよい.一方,$v_{12}$の計算では系2で$F_2=1$としたものが補助系となる.
	これらのことから,
	\begin{eqnarray}
		v_{11}&=& \frac{J(a_1,F_1,a_1,1)}{EI}=\frac{F_1a_1^3}{3} \\
		v_{12}&=& \frac{J(a_1,F_1,a_2,1)}{EI}=\frac{1}{6}\frac{F_1a_1^2}{EI}(3a_2-a_1) 
	\end{eqnarray}
	が求める結果となる.
\item
	前の問題と同様にして単位荷重法を適用すれば,次の結果が得られる.
	\begin{eqnarray}
		v_{21}&=& \frac{J(a_1,1,a_2,F_2)}{EI}=\frac{1}{6}\frac{F_2a_1^2}{EI}(3a_2-a_1) \\
		v_{22}&=& \frac{J(a_2,1,a_2,F_2)}{EI}=\frac{F_2a_2^3}{3} 
	\end{eqnarray}
\item
	問題で与えられた系に作用する支点反力の正方向を図\ref{fig:fig2_2}-(a)のようにとり,
	同図(b)と(c)に示す2つの静定系1と2に分割する.
	静定系1と2の点Bにおけるたわみを,それぞれ,$v_B^{(1)},v_B^{(2)}$とすれば,
	これらは,$v_{11}$と$v_{21}$を用いて
	\begin{eqnarray}
		v_B^{(1)} &=& \left. v_{21}\right|_{a_1=a,a_2=l, F_2=F}= \frac{1}{6}\frac{Fa^2}{EI}(3l-a) \\
		v_B^{(2)} &=& \left. v_{11}\right|_{a_1=a, F_1=-R_B}= -\frac{R_Ba^3}{3EI}  
	\end{eqnarray}
	と与えられる.B点での適合条件
	\begin{equation}
		v_B^{(1)}+v_B^{(2)}=0 
	\end{equation}
	に以上を代入すれば,B点の反力が
	\begin{equation}
		R_B=\frac{F}{2}\left(\frac{3l}{a}-1\right)
	\end{equation}
	と決まる.他の反力は,構造全体の釣り合い条件より
	\begin{equation}
		R_A=\frac{3F}{2}\left(1-\frac{l}{a}\right), \ \ 
		M_A=\frac{F}{2}(l-a) ,\ \ H_A=0
		\label{eqn:}
	\end{equation}
	となる.
\item
	図\ref{fig:fig3}の通り.
\item
	問題で与えられた系に作用する支点反力の正方向を図\ref{fig:fig4}-(a)のようにとり,
	これを,同図(b)から(d)に示す3つの静定系に分割する.
	静定系$i(=0,1,2)$の点Bと点Cにおけるたわみを,それぞれ$v^{(i)}_B,v^{(i)}_C$とすれば,
	これらは,以下のように求められる.
	\begin{eqnarray}
		v_B^{(0)} &=& \left. v_{21}\right|_{a_1=l, F_2=F,a_2=3l}= \frac{4}{3}\frac{Fl^3}{EI} \\ 
		v_C^{(0)} &=& \left. v_{21}\right|_{a_1=2l, F_2=F, a_2=3l}= \frac{14}{3}\frac{Fl^3}{EI}\\
		v_B^{(1)} &=& \left. v_{11}\right|_{a_1=l, F_1=-R_B}= -\frac{1}{3}\frac{R_Bl^3}{EI} \\ 
		v_C^{(1)} &=& \left. v_{12}\right|_{a_1=l, F_1=-R_B, a_2=2l}= -\frac{5}{6}\frac{R_Bl^3}{EI}\\
		v_B^{(2)} &=& \left. v_{21}\right|_{a_1=l, a_2=2l, F_2=-R_C}= -\frac{5}{6}\frac{R_Cl^3}{EI} \\ 
		v_C^{(2)} &=& \left. v_{22}\right|_{a_2=2l, F_2=-R_C}= -\frac{8}{3}\frac{R_Cl^3}{EI}
	\end{eqnarray}
	これを,点Bにおけるたわみの適合条件に代入すれば,
	\begin{equation}
		v_B^{(0)}+v_B^{(1)}+v_B^{(2)}=0 \ \ \Rightarrow \ \ 2R_B+5R_C=8F
		\label{eqn:}
	\end{equation}
	の関係が,点Cにおける適合条件に代入すれば,
	\begin{equation}
		v_C^{(0)}+v_C^{(1)}+v_C^{(2)}=0 \ \ \Rightarrow \ \ 5R_B+16R_C=28F
		\label{eqn:}
	\end{equation}
	が得られる.両者を連立方程式として解くことで,B点とC点における反力が次のように求められる.
	\begin{equation}
		R_B=-\frac{12}{7}F, \ \ R_C=\frac{16}{7}F
		\label{eqn:}
	\end{equation}
\end{enumerate}
%--------------------
\begin{figure}[h]
	\begin{center}
	\includegraphics[width=0.4\linewidth]{fig1ans.eps} 
	\end{center}
	\caption{集中荷重を受ける片持梁の曲げモーメント図).}
	\label{fig:fig1}
\end{figure}
%--------------------
%--------------------
\begin{figure}[h]
	\begin{center}
	\includegraphics[width=0.80\linewidth]{fig2ans2.eps} 
	\end{center}
	\caption{曲げモーメント$M_1$と$M_2$を表すための座標$s_1$と$s_2$.}
	\label{fig:fig2}
\end{figure}
%--------------------
\begin{figure}[h]
	\begin{center}
	\includegraphics[width=1.0\linewidth]{fig2ans3.eps} 
	\end{center}
	\caption{不静定梁の2つの静定系への分割.}
	\label{fig:fig2_2}
\end{figure}
\begin{figure}[h]
	\begin{center}
	\includegraphics[width=0.4\linewidth]{fig2ans4.eps} 
	\end{center}
	\caption{不静定梁の3つの静定系への分割.}
	\label{fig:fig2_3}
\end{figure}
\begin{figure}
	\begin{center}
	\includegraphics[width=0.4\linewidth]{fig2ans.eps} 
	\end{center}
	\caption{支点反力の正方向と不静定梁の曲げモーメント図.}
	\label{fig:fig3}
\end{figure}
%%%%%%%%%%%%%%%%%%%%%%%%%%%%%%%%%%%%%%%%%%%%
\newpage
\subsubsection*{問題2.}
部材1から4における断面力位置を表すための座標を図\ref{fig:fig4}-(a)に,
断面力の正方向を同図の(b)-(e)のように示すように取る.
\begin{enumerate}
\item
	各部材の軸力は,力の釣り合いから
	\begin{equation}
		N_1=N_3=0, \ \ N_2=N_4=-F
		\label{eqn:NiF}
	\end{equation}
	と求められ,これを軸力図として示すと図\ref{fig:fig5}-(a)のようになる。
\item
	曲げモーメント分布は
	\begin{equation}
		M_1=-Fx_1, \ \ M_2=-Fl, \ \ M_3=-F(l-x_3), \ \ M_4=0
		\label{eqn:MiF}
	\end{equation}
	となるので,これを断面力図として示すと,図\ref{fig:fig5}-(b)のようになる。
\item
	\begin{equation}
		N_1=N_3=0, \ \ N_2=N_4=-q_0l
		\label{eqn:Niq}
	\end{equation}
	となり、これを図示すると図\ref{fig:fig6}の通りとなる。
\item
	\begin{equation}
		M_1=-\frac{q_0x_1^2}{2}, \ \ M_2=-\frac{q_0l^2}{2}, \ \ M_3=-q_0l^2\left(\frac{1}{2}l-x_3\right), \ \ M_4=\frac{q_0l^2}{2}
		\label{eqn:Miq}
	\end{equation}
	となり、これを図示すると図\ref{fig:fig6}の通りとなる。
\item
	単位荷重法を用いてたわみの計算を行う。
	単位荷重法の補助系には、点Aに鉛直方向の単位荷重加えたものを用いれば良い。
	またそのときの断面力は、式(\ref{eqn:NiF})と(\ref{eqn:MiF})において$F=1$とすればよい。
	このとき,単位荷重法の計算に現れる断面力に関する部材毎の積分は以下のようになる。
	\begin{equation}
		\int_0^l 
			N_i \tilde{N}_i dx_i = 
		\left\{
		\begin{array}{cc}
			0 & (i=1) \\
			Fl& (i=2) \\
			0 & (i=3) \\
			Fl & (i=4)
		\end{array}
		\right.
		, \ \
		\int_0^l 
			M_i \tilde{M}_i dx_i = 
		\left\{
		\begin{array}{cc}
			\frac{Fl^3}{3} & (i=1) \\
			Fl^3 & (i=2) \\
			\frac{Fl^3}{3} & (i=3) \\
			0 & (i=4)
		\end{array}
		\right.
	\end{equation}
	以上より、たわみが次のように求められる.
	\begin{equation}
		v_a=
		\sum_{i=1}^4
		\int_0^l \left( \frac{N_i\tilde{N}_i}{EA} + \frac{M_i\tilde{M}_i}{EI} \right) dx_i
		=
		\frac{2Fl}{EA} + \frac{5}{3}\frac{Fl^3}{EI}
		\label{eqn:}
	\end{equation}
\item
	単位荷重法の補助系には、上の問題と同じものを用いることができる。
	この場合,部材毎の断面力に関する積分は以下のようになり、
	\begin{equation}
		\int_0^l 
			N_i \tilde{N}_i dx_i = 
		\left\{
		\begin{array}{cc}
			0 & (i=1) \\
			q_0l^2& (i=2) \\
			0 & (i=3) \\
			q_0l^2 & (i=4)
		\end{array}
		\right.
		, \ \
		\int_0^l 
			M_i \tilde{M}_i dx_i = 
		\left\{
		\begin{array}{cc}
			\frac{1}{8}q_0l^4 & (i=1) \\
			\frac{1}{2}q_0l^4 & (i=2) \\
			\frac{1}{12}q_0l^4 & (i=3) \\
			0 & (i=4)
		\end{array}
		\right.
	\end{equation}
	たわみが次のようになることが分かる.
	\begin{equation}
		v_b=
		\sum_{i=1}^4
		\int_0^l \left( \frac{N_i\tilde{N}_i}{EA} + \frac{M_i\tilde{M}_i}{EI} \right) dx_i
		=
		\frac{2q_0l^2}{EA} + \frac{17}{24}\frac{q_0l^4}{EI}
		\label{eqn:}
	\end{equation}
\item
	点Aにおけるたわみの適合条件は、$v_a$と$v_b$を用いて
	\begin{equation}
		\left. v_a\right|_{F=-R_A}+ v_b=0
	\end{equation}
	と書くことができる。また、
	\begin{equation}
		\left. v_a\right|_{F=-R_A}= -\frac{2R_Al}{EA} - \frac{5}{3}\frac{R_Bl^3}{EI}, \ \ 
		v_b= \frac{2q_0l^2}{EA} + \frac{17}{24}\frac{q_0l^4}{EI}
		\label{eqn:}
	\end{equation}
	だから、以上より
	\begin{equation}
		R_A=\frac
		{\frac{17}{40}+\frac{6}{5}\frac{I}{Al^2}}
		{1+\frac{6}{5}\frac{I}{Al^2}}q_0l.
		\label{eqn:}
	\end{equation}
\item
	軸力による変形を無視して、上の計算を繰り返せば,支点Aにおける反力は
	\begin{equation}
		R_A'=\frac{17}{40}q_0l
	\end{equation}	
	と求められる。断面2次モーメントと断面積は
	$I=\frac{a^4}{12},A=a^2$
	だから、$R_A$は
	\begin{equation}
		R_A=\frac{\frac{17}{40}+\varepsilon}{1+\varepsilon}q_0l, \ \ 
		 \left(\varepsilon=\frac{1}{10}\frac{a^2}{l^2}=10^{-3}\right)
		\label{eqn:}
	\end{equation}
	と表すことができる。よって$\eta$は
	\begin{equation}
		\eta=\frac{\left| R_A-R_A'\right|}{q_0l}=\frac{\frac{17}{40}+\varepsilon}{1+\varepsilon}-\frac{17}{40}
		\simeq 
		\left(\frac{17}{40}+\varepsilon \right) (1-\varepsilon) -\frac{17}{40}
		=\frac{23}{40}\varepsilon -\varepsilon^2
		\label{eqn:}
	\end{equation}
	となり、$\eta=0.058$\%であることが分かる。
\end{enumerate}
%--------------------
\begin{figure}
	\begin{center}
	\includegraphics[width=0.8\linewidth]{fig4ans.eps} 
	\end{center}
	\caption{断面位置を表すための部材座標系$x_1\sim x_4$と支点反力の正方向.}
	\label{fig:fig4}
\end{figure}
\begin{figure}
	\begin{center}
	\includegraphics[width=0.6\linewidth]{fig3ans.eps} 
	\end{center}
	\caption{支点反力の正方向と不静定梁の曲げモーメント図.}
	\label{fig:fig5}
\end{figure}

\begin{figure}
	\begin{center}
	\includegraphics[width=0.6\linewidth]{fig5ans.eps} 
	\end{center}
	\caption{支点反力の正方向と不静定梁の曲げモーメント図.}
	\label{fig:fig6}
\end{figure}
\end{document}
%--------------------
