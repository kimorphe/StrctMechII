\documentclass[10pt,a4j]{jarticle}
\usepackage{graphicx,wrapfig}
\setlength{\topmargin}{-1.5cm}
%\setlength{\textwidth}{15.5cm}
\setlength{\textheight}{25.2cm}
\newlength{\minitwocolumn}
\setlength{\minitwocolumn}{0.5\textwidth}
\addtolength{\minitwocolumn}{-\columnsep}
%\addtolength{\baselineskip}{-0.1\baselineskip}
%
\def\Mmaru#1{{\ooalign{\hfil#1\/\hfil\crcr
\raise.167ex\hbox{\mathhexbox 20D}}}}
%
\begin{document}
\newcommand{\fat}[1]{\mbox{\boldmath $#1$}}
\newcommand{\D}{\partial}
\newcommand{\w}{\omega}
\newcommand{\ga}{\alpha}
\newcommand{\gb}{\beta}
\newcommand{\gx}{\xi}
\newcommand{\gz}{\zeta}
\newcommand{\vhat}[1]{\hat{\fat{#1}}}
\newcommand{\spc}{\vspace{0.7\baselineskip}}
\newcommand{\halfspc}{\vspace{0.3\baselineskip}}
\bibliographystyle{unsrt}
%\pagestyle{empty}
\newcommand{\twofig}[2]
 {
   \begin{figure}[h]
     \begin{minipage}[t]{\minitwocolumn}
         \begin{center}   #1
         \end{center}
     \end{minipage}
         \hspace{\columnsep}
     \begin{minipage}[t]{\minitwocolumn}
         \begin{center} #2
         \end{center}
     \end{minipage}
   \end{figure}
 }
%%%%%%%%%%%%%%%%%%%%%%%%%%%%%%%%%
%\vspace*{\baselineskip}
\begin{center}
{\Large \bf 2020年度 構造力学II 期末試験} \\
\end{center}
\begin{flushright}
	2020年6月8日(月)
\end{flushright}
%%%%%%%%%%%%%%%%%%%%%%%%%%%%%%%%%%%%%%%%%%%%%%%%%%%%%%%%%%%%%%%%
\begin{itemize}
\item
	氏名,学生番号を全ての解答用紙に記入すること.
\item
	解答の過程を適宜記すこと.
\item
	最終的な解答は下線を入れる等して分かり易く示すこと.
\end{itemize}
\subsubsection*{問題 1}
図\ref{fig:fig1}$\sim$\ref{fig:fig3}に示す梁について以下の問に答えよ.なお,梁部材の
ヤング率$E$と断面2次モーメント$I$は全ての断面で一定とする.
\begin{enumerate}
\item
	図\ref{fig:fig1}に示す梁の,区間ACにおける曲げモーメント分布を求めよ.
\item
	図\ref{fig:fig2}に示す2つの系(系1,系2)に発生する曲げモーメントを
	それぞれ$M_1$と$M_2$とするとき,
	\begin{equation}
		J(a_1,F_1;a_2,F_2)=\int_{x=0}^{l}M_1M_2 dx
	\end{equation}
	の積分を求めよ.ただし,$a_2\geq a_1$とする.
\item
	図\ref{fig:fig2}の系1における点1のたわみ$v_{11}$と,点2のたわみを$v_{12}$をそれぞれ求めよ.
\item
	図\ref{fig:fig2}の系2における点1のたわみ$v_{21}$と,点2のたわみを$v_{22}$をそれぞれ求めよ.
\item
	図\ref{fig:fig3}-(a)の系に作用する支点反力を求めよ.
\item
	図\ref{fig:fig3}-(a)の系に対する曲げモーメント図を描け.
\item
	図\ref{fig:fig3}-(b)の系において,点BとCに作用する支点反力を求めよ
	({\small ヒント:与えられた構造を3つの片持梁に分解し,BとCにおけるたわみ
	が0となるように支点反力を決定すればよい}).
%\item
%	図\ref{fig:fig3}-(b)の構造に対する曲げモーメント図を描け.
\end{enumerate}
\newpage
\begin{figure}[h]
	\begin{center}
	\includegraphics[width=0.4\linewidth]{fig1.eps} 
	\end{center}
	\caption{鉛直方向の集中荷重を受ける片持梁.} 
	\label{fig:fig1}
\end{figure}
\begin{figure}[h]
	\begin{center}
	\includegraphics[width=0.8\linewidth]{fig2.eps} 
	\end{center}
	\caption{鉛直方向の集中荷重を受ける2つの片持梁$(a_2\geq a_1)$.} 
	\label{fig:fig2}
\end{figure}
\begin{figure}[h]
	\begin{center}
	\includegraphics[width=0.80\linewidth]{fig3.eps} 
	\end{center}
	\caption{鉛直方向の集中荷重を受ける2つ不静定構造.} 
	\label{fig:fig3}
\end{figure}
%\setcounter{enumi}{4}
%%%%%%%%%%%%%%%%%%%%%%%%%%%%%%%%%%%%%%%%%%%%%%%%%%%%%%%%%
\clearpage
\subsubsection*{問題 2}
図\ref{fig:fig4}に示す骨組み構造について以下の問に答えよ. 
なお,部材のヤング率$E$,断面2次モーメント$I$, 断面積$A$は全ての部材で共通かつ一定とする.
解答には,断面力の正方向や計算に用いた座標系を明記すること.
\begin{enumerate}
\item
	構造(a)の軸力図を描け.
\item
	構造(a)の曲げモーメント図を描け.
\item
	構造(b)の軸力図を描け.
\item
	構造(b)の曲げモーメント図を描け.
\item
	構造(a)の点Aにおける鉛直変位$v_a$を求めよ.
\item
	構造(b)の点Aにおける鉛直変位$v_b$を求めよ.
\item
	構造(c)の点Aに作用する鉛直方向の反力$R_A$を求めよ.
\item
	軸力による変形が無視できるとするときの,構造(c)の点Aにおける鉛直反力$R_A'$を求めよ.
\item
	$R_A$と$R_A'$の差を
	\begin{equation}
		\eta =\frac{|R_A-R_A'|}{q_0l}
		\label{eqn:}
	\end{equation}
	で評価する.全ての部材断面は一辺の長さが$a$の正方形で,$l/a=10$のとき$\eta$は何パーセントとなるか,
	有効数字3桁で答えよ.なお,必要に応じて$|x|\ll 1$でよい近似となる次の関係を用いてよい.
	\begin{equation}
		\frac{1}{1+x}\simeq 1-x
	\label{eqn:}
	\end{equation}
\end{enumerate}
%%%%%%%%%%%%%%%%%%%%%%%%%%%%%%%%%%%%%%%%%%%%%%%%%%%%%%%%%
\begin{figure}[h]
	\begin{center}
	\includegraphics[width=0.80\linewidth]{fig4.eps} 
	\end{center}
	\caption{鉛直荷重を受ける3種類の骨組み構造.} 
	\label{fig:fig4}
\end{figure}
\end{document}

