\documentclass[10pt,a4j]{jarticle}
%\usepackage{graphicx,wrapfig}
\usepackage{graphicx}
%\usepackage{showkeys}
\setlength{\topmargin}{-1.5cm}
%\setlength{\textwidth}{16.5cm}
\setlength{\textheight}{25.2cm}
\newlength{\minitwocolumn}
\setlength{\minitwocolumn}{0.5\textwidth}
\addtolength{\minitwocolumn}{-\columnsep}
%\addtolength{\baselineskip}{-0.1\baselineskip}
%
\def\Mmaru#1{{\ooalign{\hfil#1\/\hfil\crcr
\raise.167ex\hbox{\mathhexbox 20D}}}}
%
\begin{document}
\newcommand{\fat}[1]{\mbox{\boldmath $#1$}}
\newcommand{\D}{\partial}
\newcommand{\w}{\omega}
\newcommand{\ga}{\alpha}
\newcommand{\gb}{\beta}
\newcommand{\gx}{\xi}
\newcommand{\gz}{\zeta}
\newcommand{\vhat}[1]{\hat{\fat{#1}}}
\newcommand{\spc}{\vspace{0.7\baselineskip}}
\newcommand{\halfspc}{\vspace{0.3\baselineskip}}
\bibliographystyle{unsrt}
%\pagestyle{empty}
\newcommand{\twofig}[2]
 {
   \begin{figure}
     \begin{minipage}[t]{\minitwocolumn}
         \begin{center}   #1
         \end{center}
     \end{minipage}
         \hspace{\columnsep}
     \begin{minipage}[t]{\minitwocolumn}
         \begin{center} #2
         \end{center}
     \end{minipage}
   \end{figure}
 }
%%%%%%%%%%%%%%%%%%%%%%%%%%%%%%%%%
%\vspace*{\baselineskip}
\begin{flushright}
	構造力学II\\
	2020/05/11
\end{flushright}
\begin{center}
	{\LARGE \bf 講義ノート 3} \\
\end{center}
%%%%%%%%%%%%%%%%%%%%%%%%%%%%%%%%%%%%%%%%%%%%%%%%%%%%%%%%%%%%%%%%
\setcounter{section}{2}
\section{曲げ問題に対する単位荷重法の応用}
\subsection{概要}
本講では,曲げ問題に対する単位荷重法の応用について例題を通して学習する.
軸力問題において既に見たように,単位荷重法を用いて特定の点の変位を求める
ことができれば,不静定構造における反力や断面力計算を効率的に行う上で有用となる.
また,曲げ問題では支点位置が部材端部にあるとは限らず,例えば張出し梁やゲルバー梁
では,たわみの微分方程式を解く際,反力を未知荷重項として表現することや,
支間毎にたわみを求めるなどの作業が必要となる。一方,単位荷重法では支点位置に応じた
計算手順の変更は必要無く,微分方程式を解く場合のような煩わしさが無い.
以下では,張出し梁のたわみを単位荷重法で計算する方法についての例題をはじめに扱う.
次に,連続梁をはじめとする簡単な不静定構造の支点反力と断面力を,単位荷重法を
利用して求める方法を示す.
\subsection{単位荷重法}
前回の講義では,単位荷重法の公式が次のように表されることを示した.
\begin{equation}
	v(a)=\int_{梁全体} \frac{M\tilde M}{EI}dx
	\label{eqn:va}
\end{equation}
ここに,$M$は問題で与えられた系の曲げモーメントを,$\tilde M$は$x=a$に単位集中荷重が加えられた
補助系の曲げモーメントを表す.曲げ剛性$EI$が全断面で一定の場合,式(\ref{eqn:va})は
\begin{equation}
	v(a)=(EI)^{-1}\int_{梁全体} M\tilde Mdx
	\label{eqn:va2}
\end{equation}
となり,単位荷重法の計算は実質的に,
\begin{itemize}
\item
	曲げモーメント$M$と$\tilde M$の計算,
\item
	積分$\int M\tilde M dx$の計算,
\end{itemize}	
の2つの作業に帰着される.このことは,以下の例題で示すように,張出し梁も同様である.
\subsection{張出し梁のたわみ}
以下,曲げ剛性$EI$は全断面で一定とする.
\subsubsection{例題1}
\paragraph{問題:}
図\ref{fig:fig2_5}-(a)に示す張出し梁の,点Cにおけるたわみ$v_C$を単位荷重法を用いて求めよ.
\begin{figure}[h]
	\begin{center}
	\includegraphics[width=0.8\linewidth]{fig5.eps} 
	\end{center}
	\caption{(a)支間ABに等分布荷重を受ける張出し梁(問題).
	 (b)単位荷重法の適用において用いる補助系.} 
	\label{fig:fig2_5}
\end{figure}
\paragraph{解答:}
図\ref{fig:fig2_5}-(b)に,点Cにおけるたわみを求めるための補助系を示す.
図\ref{fig:fig2_5}に示した2つの系に作用する反力は,
図\ref{fig:fig3}の上段に示す3つで,これらは静定構造
である.そこで,力とモーメントの釣り合いから,曲げモーメント分布を
求めると,図\ref{fig:fig3}の下段に示したようになる.これらの
曲げモーメント分布は,図\ref{fig:fig3}に示した座標$x$と$s$
を用いて次のように表すことができる.
\begin{equation}
	M = \left\{
	\begin{array}{cc}
		-\frac{1}{2}q_0x^2+\frac{1}{2}q_0x & (0<x<l) \\
		0 & \left( 0< s< \frac{l}{2}\right)
	\end{array}
	\right.
	\label{eqn:}
\end{equation}
\begin{equation}
	\tilde M = \left\{
	\begin{array}{cc}
		-\frac{1}{2}\tilde P x & \left( 0<x<l \right) \\
		-\tilde P s & \left( 0< s< \frac{l}{2}\right)
	\end{array}
	\right.
	\label{eqn:}
\end{equation}
従って,区間BCにおける積分$\int_C^B M\tilde Mds$は零となる.
このことを踏まえ,梁全体,すなわち区間ACにおける曲げモーメント
に関する積分を行うと,
\begin{eqnarray}
	\int_A^CM\tilde M dx & = &  \int_A^BM\tilde M dx  \nonumber \\
	&=&
	\int_0^l
	\left(	-\frac{q_0x^2}{2}+\frac{q_0lx}{2}  \right)
	\left(	-\frac{\tilde P}{2}x \right)
	dx \\
	&= & -\frac{q_0l^4}{48}
	\label{eqn:}
\end{eqnarray}
となる.以上より,
\begin{equation}
	v_C=-\frac{1}{48} \frac{q_0l^4}{EI}
	\label{eqn:}
\end{equation}
と求められる.なお,たわみが負になるということは,
点Cでは鉛直上向きのたわみが生じていることを意味する.
\begin{figure}[h]
	\begin{center}
	\includegraphics[width=0.8\linewidth]{fig3.eps} 
	\end{center}
	\caption{問題で与えられた系と単位荷重法における補助系の曲げモーメント(例題1).} 
	\label{fig:fig3}
\end{figure}
\subsubsection{例題2}
\paragraph{問題:}
図\ref{fig:fig6_2}-(a)に示す張出し梁の点Cにおけるたわみ$v_C$を求めよ.
\begin{figure}[h]
	\begin{center}
	\includegraphics[width=0.8\linewidth]{fig6_2.eps} 
	\end{center}
	\caption{(a)集中荷重を受ける張出し梁と(b)その曲げモーメント図.} 
	\label{fig:fig6_2}
\end{figure}
\paragraph{解答:}
問題で与えられた系の曲げモーメント$M$を釣り合い条件から求めると,
図\ref{fig:fig6_2}-(b)に示すようになる.この図に示されるように,$M$は
点Bに関して左右対称である.
そこで,左半分の区間ABにおける曲げモーメントを,図中の座標$x$の
関数として表すと
\begin{equation}
	M(x)=-Px,  \ \ \left(0< x < \frac{l}{2} \right) 
	\label{eqn:Mx_ex1}
\end{equation}
となる.点Cのたわみを求めるための補助系は,図\ref{fig:fig6_2}-(a)
において$P=1$としたものであるから,補助系の曲げモーメント$\tilde M$
は,式(\ref{eqn:Mx_ex1})において$P=1$と置くことで
\begin{equation}
	\tilde M(x)=-x,  \ \ \left(0< x < \frac{l}{2} \right) 
	\label{eqn:Mxt_ex1}
\end{equation}
と得られ,やはり点Bに関して左右対称である.従って,$M\tilde M$も
左右対称となるため,$M\tilde M$に関する積分は, 
区間ABにおける積分:
\begin{equation}
	\int _A^B M\tilde M dx = 
	\int_0^{l/2} (-Px)(-x) dx = \frac{Pl^3}{24}
	\label{eqn:}
\end{equation}
の結果を2倍することで得られる.よって,$v_C$は
\begin{equation}
	v_C=\int_A^C \frac{M \tilde M}{EI}dx
	=2\int_A^B \frac{M \tilde M}{EI}dx=
	\frac{Pl^3}{12EI}
	\label{eqn:}
\end{equation}
となる.
\subsubsection{練習問題1}
\begin{enumerate}
\item
	図\ref{fig:fig6_1}-(a)に示す張出し梁の, 点Dにおけるたわみ$v_D$を求めよ.
\item
	図\ref{fig:fig6_1}-(b)に示す張出し梁の, 点Cにおけるたわみ$v_C$を求めよ.
\end{enumerate}
\begin{figure}[h]
	\begin{center}
	\includegraphics[width=0.8\linewidth]{fig6_1.eps} 
	\end{center}
	\caption{荷重条件の異なる2種類の張出し梁.} 
	\label{fig:fig6_1}
\end{figure}
%%%%%%%%%%%%%%%%%%%%%%%%%%%%%%%%%%%%%%%%%%%%%%%%%%%%%%%%%%%%%%%%%%%%
\subsection{不静定梁の反力と断面力}
以下の例題で示されるように,単位荷重法は不静定構造の反力を決定するためにも
役立てることができる.以下,引続き梁の曲げ剛性$EI$は全断面で一定とする.
\subsubsection{例題3}
\paragraph{問題:}
図\ref{fig:fig7_1}に示す2径間連続梁に働く支点反力を求め,曲げモーメント図を描け.
\begin{figure}[h]
	\begin{center}
	\includegraphics[width=0.4\linewidth]{fig7_1.eps} 
	\end{center}
	\caption{等分布荷重を受ける2径間連続梁.} 
	\label{fig:fig7_1}
\end{figure}
\paragraph{解答:}
2径間連続梁に作用する支点反力の正方向を図\ref{fig:fig2_7}-(a)のように定める.
ここで,支点Bにおける未知の鉛直反力$R_B$を用い,不静定構造である2径間連続梁を
図\ref{fig:fig2_7}-(b)と(c)に示す2つの静定系(系1,系2)に分割する.
系1と系2の点Bにおけるたわみをそれぞれ$v_B^{(1)},v_B^{(2)}$とすれば,
系1と系2の重ね合わせが2径間連続梁を表現するためには,
$v_B^{(1)}$と$v_B^{(2)}$が次の適合条件を満足しなければならない.
\begin{equation}
	v_B^{(1)} + v_B^{(2)}=0
	\label{eqn:vb_vanish}
\end{equation}
$v_B^{(1)}$と$v_B^{(2)}$は,前回の講義で示した通り単位荷重法を用いて
計算することができ,その結果は
\begin{equation}
	v_B^{(1)}=
	\frac{5}{384} \frac{q_0l^4}{EI}, \ \ 
	 v_B^{(2)}=
	 -
	\frac{1}{48} \frac{R_Bl^3}{EI}
	\label{eqn:vbs}
\end{equation}
となることが分かっている.そこで,式(\ref{eqn:vbs})$を式(\ref{eqn:vb_vanish})$に
用いれば,$R_B$が
\begin{equation}
	R_B=\frac{5}{8}q_0l
	\label{Rb}
\end{equation}
と決まる.さらに,この結果を踏まえて連続梁全体の釣り合い条を考えれば,
残る全ての支点反力が
\begin{equation}
	H_A=0, \ \ R_A=R_C=\frac{3}{16}q_0l
	\label{eqn:}
\end{equation}
と求められる.支点反力が求められれば,任意の断面における断面力は
力とモーメントの釣り合い条件から決定できる.
あるいは,系1の曲げモーメント$M_1$と系2の曲げモーメント$M_2$が,
図\ref{fig:fig9}のようになることを知っていれば,
\begin{equation}
	M=M_1+M_2
	\label{eqn:}
\end{equation}
に,式(\ref{eqn:Rb})を代入して曲げモーメント分布を求めることができる.
その結果,等分布荷重が作用する2径間連続梁の曲げモーメント図は,
図\ref{fig:fig10}に示したようになる.
\begin{figure}
	\begin{center}
	\includegraphics[width=0.5\linewidth]{fig7.eps} 
	\end{center}
	\caption{(a)2径間連続梁に作用する支点反力.
	(b),(c)支点反力を計算するために用いる2つの静定系1と2.} 
	\label{fig:fig2_7}
\end{figure}
\begin{figure}[h]
	\begin{center}
	\includegraphics[width=0.7\linewidth]{fig9.eps} 
	\end{center}
	\caption{2径間連続梁を分解した2つの静定系1と2の曲げモーメント図.} 
	\label{fig:fig9}
\end{figure}
\begin{figure}[h]
	\begin{center}
	\includegraphics[width=0.5\linewidth]{fig10.eps} 
	\end{center}
	\caption{等分布荷重を受ける2径間連続梁の曲げモーメント図.} 
	\label{fig:fig10}
\end{figure}
\subsubsection{例題4}
\paragraph{問題:}
図\ref{fig:fig8_1}に示す梁に作用する支点反力を求め,曲げモーメント図を描け.
\begin{figure}
	\begin{center}
	\includegraphics[width=0.4\linewidth]{fig8_1.eps} 
	\end{center}
	\caption{鉛直方向の集中荷重を受ける不静定梁.} 
	\label{fig:fig8_1}
\end{figure}
\paragraph{解答:}
問題で与えられた不静定構造を,図\ref{fig:fig11}の(b)と(c)に示す
ような2つの片持梁に分解する.ただし,$R_B$は元の不静定構造において
支点Bに加わる未知の反力である.静定系1と2の点Bにおけるたわみを,
それぞれ$v_B^{(1)},v_B^{(2)}$とすれば,たわみの適合条件はここでも
式(\ref{eqn:vb_vanish})で与えられる.
また,これら$v_B^{(1)}$と$v_B^{(2)}$を単位荷重法で求めると
\begin{equation}
	v_B^{(1)}=
	\frac{5}{48} \frac{Pl^3}{EI}, \ \ 
	 v_B^{(2)}=
	-\frac{1}{24} \frac{R_Bl^3}{EI}
	\label{eqn:vbs2}
\end{equation}
となる.式(\ref{eqn:vbs2})を式(\ref{eqn:vb_vanish})代入すれば,
\begin{equation}
	R_B=\frac{5}{2}P
	\label{eqn:Rb2}
\end{equation}	
が得られる.これ以外の反力$M_A$と$R_A$は,
不静定系全体の釣り合い条件から
\begin{equation}
	R_A=-\frac{3}{2}P, \ \ M_A=\frac{1}{4}Pl
	\label{eqn:}
\end{equation}
と求められる.最後に,静定系1の曲げモーメント$M_1$(図\ref{fig:fig12}-(a))と
静定系2の曲げモーメント$M_2$(図\ref{fig:fig12}-(b))を求め,
これに式(\ref{eqn:Rb2})を代入して$M_1+M_2$を計算すれば,求めるべき不静定梁の
曲げモーメント図を図\ref{fig:fig12}-(c)のように描くことができる.
図\ref{fig:fig12}のようになる.
なお,単位荷重法の適用にあたって必要となる曲げモーメント分布は,
全て,前回講義の例題1で求めた結果から直ちに得ることができる
(講義ノート2,図3-(b)および図3-(c)を参照).
\begin{figure}[h]
	\begin{center}
	\includegraphics[width=1.0\linewidth]{fig11.eps} 
	\end{center}
	\caption{不静定梁の2つの片持梁への分解.(a)不静定系に働く支点反力, (b)系1と(c)系2.} 
	\label{fig:fig11}
\end{figure}
\begin{figure}[h]
	\begin{center}
	\includegraphics[width=1.0\linewidth]{fig12.eps} 
	\end{center}
	\caption{曲げモーメント図(例題4の解答).} 
	\label{fig:fig12}
\end{figure}
\clearpage
\subsubsection{練習問題2}
図\ref{fig:fig8_2}に示す梁(a)から(c)について,梁に作用する支点反力を求め,
曲げモーメント図を描け.
\begin{figure}[h]
	\begin{center}
	\includegraphics[width=0.4\linewidth]{fig8_2.eps} 
	\end{center}
	\caption{支持条件と荷重条件の異なる3種類の不静定梁. } 
	\label{fig:fig8_2}
\end{figure}
\end{document}

