\documentclass[10pt,a4j]{jarticle}
%\usepackage{graphicx,wrapfig}
\usepackage{graphicx}
\setlength{\topmargin}{-1.5cm}
\setlength{\textwidth}{16.5cm}
\setlength{\textheight}{25.2cm}
\newlength{\minitwocolumn}
\setlength{\minitwocolumn}{0.5\textwidth}
\addtolength{\minitwocolumn}{-\columnsep}
%\addtolength{\baselineskip}{-0.1\baselineskip}
%
\def\Mmaru#1{{\ooalign{\hfil#1\/\hfil\crcr
\raise.167ex\hbox{\mathhexbox 20D}}}}
%
\begin{document}
\newcommand{\fat}[1]{\mbox{\boldmath $#1$}}
\newcommand{\D}{\partial}
\newcommand{\w}{\omega}
\newcommand{\ga}{\alpha}
\newcommand{\gb}{\beta}
\newcommand{\gx}{\xi}
\newcommand{\gz}{\zeta}
\newcommand{\vhat}[1]{\hat{\fat{#1}}}
\newcommand{\spc}{\vspace{0.7\baselineskip}}
\newcommand{\halfspc}{\vspace{0.3\baselineskip}}
\bibliographystyle{unsrt}
%\pagestyle{empty}
\newcommand{\twofig}[2]
 {
   \begin{figure}
     \begin{minipage}[t]{\minitwocolumn}
         \begin{center}   #1
         \end{center}
     \end{minipage}
         \hspace{\columnsep}
     \begin{minipage}[t]{\minitwocolumn}
         \begin{center} #2
         \end{center}
     \end{minipage}
   \end{figure}
 }
%%%%%%%%%%%%%%%%%%%%%%%%%%%%%%%%%
%\vspace*{\baselineskip}
\begin{center}
	{\Large \bf 2020年度 構造力学II 第3回} \\
\end{center}
%%%%%%%%%%%%%%%%%%%%%%%%%%%%%%%%%%%%%%%%%%%%%%%%%%%%%%%%%%%%%%%%
\setcounter{section}{2}
\section{曲げ問題に対する単位荷重法の応用}
\subsection{概要}
本講では,曲げ問題に対する単位荷重法の応用について,例題を通して学習する.
軸力問題において既に見たように,単位荷重法を用いて特定の点の変位を求める
ことができれば,不静定構造における反力や断面力の計算を効率的に行う上で有用となる.
また,曲げ問題では支点位置が部材端部にあるとは限らない.例えば張出し梁やゲルバー梁
では,たわみの微分方程式を解く際,反力を未知の荷重項として表現することや,
支間毎にたわみを求めるなどの作業が必要となる。一方,単位荷重法では支点位置に応じた
計算手順の変更は必要無く,微分方程式を解く場合のような煩わしさが無い.
以下では,張出し梁のたわみを単位荷重法で計算する方法についての例題をはじめに扱う.
次に,連続梁をはじめとする簡単な不静定構造の支点反力と断面力を,単位荷重法を
利用して求める方法を示す.
\subsection{単位荷重法}
単位荷重法の公式が,次のように表されることは,前回の講義において既に学習した.
\begin{equation}
	v(a)=\int_{梁全体} \frac{M\tilde M}{EI}dx
	\label{eqn:va}
\end{equation}
ここに,$M$は問題の曲げモーメントを表し,$\tilde M$は$x=a$に単位集中荷重が加えられた補助系の曲げモーメントを表す.
曲げ剛性$EI$が全断面で一定の場合,
式(\ref{eqn:va})は
\begin{equation}
	v(a)=(EI)^{-1}\int_{梁全体} M\tilde Mdx
	\label{eqn:va2}
\end{equation}
となるため,単位荷重法の計算は実質的に,曲げモーメント$M,\tilde M$を求め,
$\int M\tilde M dx$の積分を計算する作業に帰着される.
\subsection{張出し梁のたわみ}
\subsubsection{例題1}
\paragraph{問題:}
図\ref{fig:fig2_5}-(a)に示す張出し梁の,点Cにおけるたわみを
単位荷重法を用いて求めよ.なお,断面剛性$EA$は全断面で一定とする.
\begin{figure}[h]
	\begin{center}
	\includegraphics[width=0.8\linewidth]{fig5.eps} 
	\end{center}
	\caption{(a)支間ABに等分布荷重を受ける張出し梁(問題).
	 (b)単位荷重法の適用において用いる補助系.} 
	\label{fig:fig2_5}
\end{figure}
\paragraph{解答:}
単位荷重法の適用において用いる補助系を,図\ref{fig:fig2_5}-(b)に示す.
\begin{figure}[h]
	\begin{center}
	\includegraphics[width=0.8\linewidth]{fig3.eps} 
	\end{center}
	\caption{問題で与えられた系と単位荷重法における補助系の曲げモーメント(例題1).} 
	\label{fig:fig3}
\end{figure}

\subsubsection{問題1}
\begin{figure}
	\begin{center}
	\includegraphics[width=0.8\linewidth]{fig6.eps} 
	\end{center}
	\caption{荷重条件の異なる4種類の張出し梁.} 
	\label{fig:fig2_6}
\end{figure}
\begin{enumerate}
\item
	図\ref{fig:fig2_6}-(a)に示す張出し梁の, 点Cにおけるたわみを求めよ.
\item
	図\ref{fig:fig2_6}-(b)に示す張出し梁の, 点Dにおけるたわみを求めよ.
\item
	図\ref{fig:fig2_6}-(c)に示す張出し梁の, 点Cにおけるたわみを求めよ.
\item
	図\ref{fig:fig2_6}-(d)に示す張出し梁の,点Cにおけるたわみを求めよ.
\end{enumerate}
\subsection{不静定梁の反力と断面力}
\subsubsection{例題3}
図\ref{fig:fig2_7}-(a)に示す2径間連続梁の支点反力
$R_A, R_B$および$R_C$を求め,曲げモーメント図を描け.
なお,断面剛性$EA$は全断面で一定とする.
\begin{figure}
	\begin{center}
	\includegraphics[width=0.4\linewidth]{fig7.eps} 
	\end{center}
	\caption{(a)2径間連続梁と梁に作用する支点反力.(b),(c)2径間連続梁の
	支点反力計算に用いる2つの静定系.} 
	\label{fig:fig2_7}
\end{figure}
\subsubsection{問題3}
図\ref{fig:fig2_8}に示す梁(a)から(d)について,梁に作用する支点反力を求め,
曲げモーメント図を描け.なお,梁の曲げ剛性$EI$は全ての梁,全ての断面で一定とする.
\begin{figure}
	\begin{center}
	\includegraphics[width=0.8\linewidth]{fig8.eps} 
	\end{center}
	\caption{支持条件と荷重条件の異なる4種類の不静定梁.} 
	\label{fig:fig2_8}
\end{figure}
\subsection{問題}
\end{document}

