\documentclass[10pt,a4j]{jarticle}
%\usepackage{graphicx,wrapfig}
\usepackage{graphicx}
\usepackage{showkeys}
\setlength{\topmargin}{-1.5cm}
%\setlength{\textwidth}{16.5cm}
\setlength{\textheight}{25.2cm}
\newlength{\minitwocolumn}
\setlength{\minitwocolumn}{0.5\textwidth}
\addtolength{\minitwocolumn}{-\columnsep}
%\addtolength{\baselineskip}{-0.1\baselineskip}
%
\def\Mmaru#1{{\ooalign{\hfil#1\/\hfil\crcr
\raise.167ex\hbox{\mathhexbox 20D}}}}
%
\begin{document}
\newcommand{\fat}[1]{\mbox{\boldmath $#1$}}
\newcommand{\D}{\partial}
\newcommand{\w}{\omega}
\newcommand{\ga}{\alpha}
\newcommand{\gb}{\beta}
\newcommand{\gx}{\xi}
\newcommand{\gz}{\zeta}
\newcommand{\vhat}[1]{\hat{\fat{#1}}}
\newcommand{\spc}{\vspace{0.7\baselineskip}}
\newcommand{\halfspc}{\vspace{0.3\baselineskip}}
\bibliographystyle{unsrt}
%\pagestyle{empty}
\newcommand{\twofig}[2]
 {
   \begin{figure}
     \begin{minipage}[t]{\minitwocolumn}
         \begin{center}   #1
         \end{center}
     \end{minipage}
         \hspace{\columnsep}
     \begin{minipage}[t]{\minitwocolumn}
         \begin{center} #2
         \end{center}
     \end{minipage}
   \end{figure}
 }
%%%%%%%%%%%%%%%%%%%%%%%%%%%%%%%%%
%\vspace*{\baselineskip}
\begin{flushright}
	構造力学II\\
	2020/05/11
\end{flushright}
\begin{center}
	{\LARGE \bf 講義ノート 3} \\
\end{center}
%%%%%%%%%%%%%%%%%%%%%%%%%%%%%%%%%%%%%%%%%%%%%%%%%%%%%%%%%%%%%%%%
\setcounter{section}{2}
\section{曲げ問題に対する単位荷重法の応用}
\subsection{概要}
本講では,曲げ問題に対する単位荷重法の応用について例題を通して学習する.
軸力問題において既に見たように,単位荷重法を用いて特定の点の変位を求める
ことができれば,不静定構造における反力や断面力計算を効率的に行う上で有用となる.
また,曲げ問題では支点位置が部材端部にあるとは限らず,例えば張出し梁やゲルバー梁
では,たわみの微分方程式を解く際,反力を未知荷重項として表現することや,
支間毎にたわみを求めるなどの作業が必要となる。一方,単位荷重法では支点位置に応じた
計算手順の変更は必要無く,微分方程式を解く場合のような煩わしさが無い.
以下では,張出し梁のたわみを単位荷重法で計算する方法についての例題をはじめに扱う.
次に,連続梁をはじめとする簡単な不静定構造の支点反力と断面力を,単位荷重法を
利用して求める方法を示す.
\subsection{単位荷重法}
単位荷重法の公式が,次のように表されることは,前回の講義において既に学習した.
\begin{equation}
	v(a)=\int_{梁全体} \frac{M\tilde M}{EI}dx
	\label{eqn:va}
\end{equation}
ここに,$M$は問題で与えられた系の曲げモーメントを,$\tilde M$は$x=a$に単位集中荷重が加えられた
補助系の曲げモーメントを表す.曲げ剛性$EI$が全断面で一定の場合,式(\ref{eqn:va})は
\begin{equation}
	v(a)=(EI)^{-1}\int_{梁全体} M\tilde Mdx
	\label{eqn:va2}
\end{equation}
となるため,単位荷重法の計算は実質的に,曲げモーメント$M,\tilde M$を求め,
$\int M\tilde M dx$の積分を計算する作業に帰着される.
以下に示す問題では,曲げ剛性$EI$は全断面で一定とする.
\subsection{張出し梁のたわみ}
\subsubsection{例題1}
\paragraph{問題:}
図\ref{fig:fig2_5}-(a)に示す張出し梁の,点Cにおけるたわみ$v_C$を単位荷重法を用いて求めよ.
\begin{figure}[h]
	\begin{center}
	\includegraphics[width=0.8\linewidth]{fig5.eps} 
	\end{center}
	\caption{(a)支間ABに等分布荷重を受ける張出し梁(問題).
	 (b)単位荷重法の適用において用いる補助系.} 
	\label{fig:fig2_5}
\end{figure}
\paragraph{解答:}
単位荷重法の適用において用いる補助系を,図\ref{fig:fig2_5}-(b)に示す.
図\ref{fig:fig2_5}に示した2つの系に対する自由物体図は,
図\ref{fig:fig3}の上段に示したようになり,これらは静定構造である.
そこで,力とモーメントの釣り合いから,支点反力と断面力を求めることができ,
ここで必要となる曲げモーメント$M$と$\tilde M$の分布を示せば,
図\ref{fig:fig3}の(c)と(d)のようになる.これらは,同図に示した座標$x$あるいは
$s$を用いて次のように表される.
\begin{equation}
	M = \left\{
	\begin{array}{cc}
		-\frac{1}{2}q_0x^2+\frac{1}{2}q_0x & (0<x<l) \\
		0 & \left( 0< s< \frac{l}{2}\right)
	\end{array}
	\right.
	\label{eqn:}
\end{equation}
\begin{equation}
	\tilde M = \left\{
	\begin{array}{cc}
		-\frac{1}{2}\tilde P x & \left( 0<x<l \right) \\
		-\tilde P s & \left( 0< s< \frac{l}{2}\right)
	\end{array}
	\right.
	\label{eqn:}
\end{equation}
従って,$\int_C^B M\tilde Mds=0$だから,
\begin{eqnarray}
	\int_A^CM\tilde M dx & = &  \int_A^BM\tilde M dx  \nonumber \\
	&=&
	\int_0^l
	\left(	-\frac{q_0x^2}{2}+\frac{q_0lx}{2}  \right)
	\left(	-\frac{\tilde P}{2}x \right)
	dx \\
	&= & -\frac{q_0l^4}{48}
	\label{eqn:}
\end{eqnarray}
となり,C点のたわみ$v_C$が
\begin{equation}
	v_C=-\frac{1}{48} \frac{q_0l^4}{EI}
	\label{eqn:}
\end{equation}
と求められる.
\begin{figure}[h]
	\begin{center}
	\includegraphics[width=0.8\linewidth]{fig3.eps} 
	\end{center}
	\caption{問題で与えられた系と単位荷重法における補助系の曲げモーメント(例題1).} 
	\label{fig:fig3}
\end{figure}
\subsubsection{例題2}
\paragraph{問題:}
図\ref{fig:fig6_2}-(a)に示す張出梁の点Cにおけるたわみ$v_C$を求めよ.
\begin{figure}[h]
	\begin{center}
	\includegraphics[width=0.8\linewidth]{fig6_2.eps} 
	\end{center}
	\caption{(a)集中荷重を受ける張出し梁と(b)その曲げモーメント図.} 
	\label{fig:fig6_2}
\end{figure}
\paragraph{解答:}
問題で与えられた系の曲げモーメント分布を釣り合い条件から求めると,
図\ref{fig:fig6_2}-(b)に示すように,点Bに関して左右対称になることが分かる.
この図に示した座標$x$を用いて曲げモーメント分布を表すと,区間ABでは
\begin{equation}
	M(x)=-Px,  \ \ \left(0< x < \frac{l}{2} \right) 
	\label{eqn:Mx_ex1}
\end{equation}
となる.点Cのたわみを求めるための単位荷重法における補助系は,図\ref{fig:fig6_2}-(a)
において$P=1$としたものであるから,補助系の曲げモーメント$\tilde M$も式(\ref{eqn:Mx_ex1})
から直ちに得られ
\begin{equation}
	\tilde M(x)=-x,  \ \ \left(0< x < \frac{l}{2} \right) 
	\label{eqn:Mxt_ex1}
\end{equation}
となる.これより,
\begin{equation}
	\int _A^B M\tilde M dx = 
	\int_0^{l/2} (-Px)(-x) dx = \frac{Pl^3}{24}
	\label{eqn:}
\end{equation}
が得られ,$v_C$が
\begin{equation}
	v_C=\int_A^C \frac{M \tilde M}{EI}dx
	=2\int_A^B \frac{M \tilde M}{EI}dx=
	\frac{Pl^3}{12EI}
	\label{eqn:}
\end{equation}
と決まる.
\subsubsection{練習問題1}
\begin{enumerate}
\item
	図\ref{fig:fig6_1}-(a)に示す張出し梁の, 点Dにおけるたわみ$v_D$を求めよ.
\item
	図\ref{fig:fig6_1}-(b)に示す張出し梁の, 点Cにおけるたわみ$v_C$を求めよ.
\end{enumerate}
\begin{figure}[h]
	\begin{center}
	\includegraphics[width=0.8\linewidth]{fig6_1.eps} 
	\end{center}
	\caption{荷重条件の異なる2種類の張出し梁.} 
	\label{fig:fig6_1}
\end{figure}
%%%%%%%%%%%%%%%%%%%%%%%%%%%%%%%%%%%%%%%%%%%%%%%%%%%%%%%%%%%%%%%%%%%%
\subsection{不静定梁の反力と断面力}
\subsubsection{例題3}
\paragraph{問題:}
図\ref{fig:fig7_1}に示す2径間連続梁に働く支点反力を求め,曲げモーメント図を描け.
\begin{figure}[h]
	\begin{center}
	\includegraphics[width=0.4\linewidth]{fig7_1.eps} 
	\end{center}
	\caption{等分布荷重を受ける2径間連続梁.} 
	\label{fig:fig7_1}
\end{figure}
\paragraph{解答:}
与えられた2径間連続梁に作用する支点反力の正方向を図\ref{fig:fig2_7}-(a)のように定め,
同図の(b)と(c)に示すような2つの静定系(系1,系2)の重ね合わせで表現する.
系1と2の,点Bにおけるたわみをそれぞれ$v_B^{(1)},v_B^{(2)}$とすれば,
これらのたわみが満足すべき適合条件は
\begin{equation}
	v_B^{(1)} + v_B^{(2)}=0
	\label{eqn:vb_vanish}
\end{equation}
と表される.これらのたわみは,前回の講義で単位荷重法を用いて
\begin{equation}
	v_B^{(1)}=
	\frac{5}{384} \frac{q_0l^4}{EI}, \ \ 
	 v_B^{(2)}=
	\frac{-1}{48} \frac{R_Bl^3}{EI}
	\label{eqn:}
\end{equation}
となることが分かっているため,以上から$R_B=\frac{5}{8}q_0l$が得られる.
そこで,梁全体の釣り合い条件を考えれば,全ての支点反力が
\begin{equation}
	H_A=0, \ \ R_A=R_C=\frac{3}{16}q_0l, \ \ R_B=\frac{5}{8}q_0l
	\label{eqn:}
\end{equation}
と求められる.支点反力が決まれば,任意の断面における断面力を釣り条件から求めることができる.あるいは,系1と系2の曲げモーメント分布(それぞれ$M_1, M_2$とする)
が,図\ref{fig:fig9}のようになることを知っていれば,
\begin{equation}
	M=M_1+M_2
	\label{eqn:}
\end{equation}
に上で求めた$R_B$を代入して,曲げモーメント分布を求めることや
曲げモーメント図を描くことができる.
その結果は図\ref{fig:fig10}のようになる.
\begin{figure}
	\begin{center}
	\includegraphics[width=0.5\linewidth]{fig7.eps} 
	\end{center}
	\caption{(a)2径間連続梁と梁に作用する支点反力.(b),(c)2径間連続梁の
	支点反力を計算するために用いる2つの静定系.それぞれ,系1,系2と呼ぶ.} 
	\label{fig:fig2_7}
\end{figure}
\begin{figure}[h]
	\begin{center}
	\includegraphics[width=0.7\linewidth]{fig9.eps} 
	\end{center}
	\caption{2径間連続梁を分解した2つの静定系1と2の曲げモーメント図.} 
	\label{fig:fig9}
\end{figure}
\begin{figure}[h]
	\begin{center}
	\includegraphics[width=0.5\linewidth]{fig10.eps} 
	\end{center}
	\caption{等分布荷重を受ける2径間連続梁の曲げモーメント図.} 
	\label{fig:fig10}
\end{figure}
\subsubsection{例題4}
\paragraph{問題:}
図\ref{fig:fig8_1}に示す梁に作用する支点反力を求め,曲げモーメント図を描け.
\begin{figure}
	\begin{center}
	\includegraphics[width=0.4\linewidth]{fig8_1.eps} 
	\end{center}
	\caption{鉛直方向の集中荷重を受ける不静定梁.} 
	\label{fig:fig8_1}
\end{figure}
\paragraph{解答:}
問題で与えられた不静定構造を,図\ref{fig:fig11}の(b)と(c)に示す
2つの片持梁に分解する.ただし,$R_B$は元の不静定系において支点Bで作用する反力である.
静定系1と2,それぞれの点Bにおけるたわみを$v_B^{(1)},v_B^{(2)}$とすれば,
たわみの適合条件はここでも式(\ref{eqn:vb_vanish})で与えられる.
また,これらのたわみを単位荷重法で求めると
\begin{equation}
	v_B^{(1)}=
	\frac{5}{48} \frac{Pl^3}{EI}, \ \ 
	 v_B^{(2)}=
	-\frac{1}{24} \frac{R_Bl^3}{EI}
	\label{eqn:}
\end{equation}
となる.よって,$R_B=\frac{5}{2}P$と決まり,これ以外の反力は,不静定系全体の釣り合い条件から
\begin{equation}
	R_A=-\frac{3}{2}P, \ \ M_A=\frac{1}{4}Pl
	\label{eqn:}
\end{equation}
と求められる.これらの結果を元に曲げモーメント分布を求め,曲げモーメント図を描くと,
図\ref{fig:fig12}のようになる.
なお,単位荷重法の適用にあたって必要となる曲げモーメント分布は,全て,前回講義の例題1
で求めた結果から直ちに得ることができる(講義ノート2,図3-(b)および(c)を参照).
\begin{figure}[h]
	\begin{center}
	\includegraphics[width=1.0\linewidth]{fig11.eps} 
	\end{center}
	\caption{不静定梁の2つの片持梁への分解.(a)不静定系に働く支点反力, (b)系1と(c)系2.} 
	\label{fig:fig11}
\end{figure}
\begin{figure}[h]
	\begin{center}
	\includegraphics[width=0.4\linewidth]{fig12.eps} 
	\end{center}
	\caption{曲げモーメント図(例題4の解答).} 
	\label{fig:fig12}
\end{figure}
\subsubsection{練習問題2}
図\ref{fig:fig8_2}に示す梁(a)から(c)について,梁に作用する支点反力を求め,
曲げモーメント図を描け.
\begin{figure}
	\begin{center}
	\includegraphics[width=1.0\linewidth]{fig8_2.eps} 
	\end{center}
	\caption{支持条件と荷重条件の異なる3種類の不静定梁.} 
	\label{fig:fig8_2}
\end{figure}
\end{document}

