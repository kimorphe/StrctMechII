\documentclass[10pt,a4j]{jarticle}
%\usepackage{graphicx,wrapfig}
\usepackage{graphicx}
\setlength{\topmargin}{-1.5cm}
\setlength{\textwidth}{16.5cm}
\setlength{\textheight}{25.2cm}
\newlength{\minitwocolumn}
\setlength{\minitwocolumn}{0.5\textwidth}
\addtolength{\minitwocolumn}{-\columnsep}
%\addtolength{\baselineskip}{-0.1\baselineskip}
%
\def\Mmaru#1{{\ooalign{\hfil#1\/\hfil\crcr
\raise.167ex\hbox{\mathhexbox 20D}}}}
%
\begin{document}
\newcommand{\fat}[1]{\mbox{\boldmath $#1$}}
\newcommand{\D}{\partial}
\newcommand{\w}{\omega}
\newcommand{\ga}{\alpha}
\newcommand{\gb}{\beta}
\newcommand{\gx}{\xi}
\newcommand{\gz}{\zeta}
\newcommand{\vhat}[1]{\hat{\fat{#1}}}
\newcommand{\spc}{\vspace{0.7\baselineskip}}
\newcommand{\halfspc}{\vspace{0.3\baselineskip}}
\bibliographystyle{unsrt}
%\pagestyle{empty}
\newcommand{\twofig}[2]
 {
   \begin{figure}
     \begin{minipage}[t]{\minitwocolumn}
         \begin{center}   #1
         \end{center}
     \end{minipage}
         \hspace{\columnsep}
     \begin{minipage}[t]{\minitwocolumn}
         \begin{center} #2
         \end{center}
     \end{minipage}
   \end{figure}
 }
%%%%%%%%%%%%%%%%%%%%%%%%%%%%%%%%%
%\vspace*{\baselineskip}
\begin{flushright}
	構造力学II\\
	2020/06/01
\end{flushright}
\begin{center}
	{\LARGE \bf 講義ノート 6} \\
\end{center}
%%%%%%%%%%%%%%%%%%%%%%%%%%%%%%%%%%%%%%%%%%%%%%%%%%%%%%%%%%%%%%%%
\section{マトリクス構造解析(有限要素法)}
図\ref{fig:fig1}に示すような荷重を受ける軸力部材を考える.
いま,部材の左端部を節点1, 右端を節点2と呼び,これらの節点における荷重を,
$x$軸方向を正として$f_1,f_2$と表す.
同様に,節点1と2における変位を$u_1,\,u_2$とし,節点荷重$(f_1,\, f_2)$と節点変位$(u_1,\, u_2)$
が満足すべき関係を求める.ただしここでは,軸力問題の厳密な解$u(x)$を求めるのでなく,
その近似解を求めることを考える.従って,以下に示す方法で求められるものはあくまで近似解だが,
問題によっては厳密な解が与えられることもできる.また,計算量の増加を許容するならば,
必要に応じて任意に高い精度の近似解を構成することができる.
\begin{figure}[b]
	\begin{center}
	\includegraphics[width=0.4\linewidth]{fig1.eps} 
	\end{center}
	\caption{軸力問題.} 
	\label{fig:fig1}
\end{figure}
\subsection{近似解と内挿関数}
軸力問題の厳密解を$u(x)$, 近似解を$\tilde u(x)$と表す.
はじめに,未知の解$u(x)$を$0<x<l$において直線で近似することを考える.
このとき,$\phi_1(x)$と$\phi_2(x)$を
\begin{equation}
	\phi_1(x)= 1-\frac{x}{l}, \ \ 
	\phi_2(x)= \frac{x}{l}
	\label{eqn:basis}
\end{equation}
とすれば,近似解$\tilde u(x)$を,
\begin{equation}
	u(x)\simeq \tilde u (x)=u_1\phi_1(x)+ u_2\phi_2(x)=\sum_{j=1}^2u_j\phi_j(x)
	\label{eqn:u_tilde}
\end{equation}
と表すことができる.$\phi_1,\, \phi_2$は,節点間の解を内挿する役割を果たすことから
"内挿関数"と呼ばれる.ここで,$x_1=0, x_2=l$とすれば、内挿関数$\phi_1,\phi_2$は
\begin{equation}
	\phi_i(x_j)=\delta_{ij}, \ \ (i,j=1,2)
	\label{eqn:phi_01}
\end{equation}
の関係を満足する.
\subsection{弱形式}
図\ref{fig:fig1}に示す軸力問題に対する弱形式:
\begin{equation}
	a(u,\xi)=b(\xi)
	\label{eqn:WF}
\end{equation}
の各項は,節点における荷重$f_1,f_2$が既知と考えるとき,
\begin{eqnarray}
	a(u,\xi) &= & \int_{0}^{l}EAu'\xi'dx 
	\label{eqn:def_a}
	\\
	b(\xi) &= & \sum_{i=1}^2 f_i\xi(x_i) + \int_{0}^{l}p\xi dx
	\label{eqn:def_b}
\end{eqnarray}
で与えられる.
そこで,近似解(\ref{eqn:u_tilde})を$u=\tilde u$とし,$\xi=\phi_i,\,(i=1,2)$を
として式(\ref{eqn:WF})に代入すれば,
\begin{equation}
	a\left(\sum_{j=1}^2u_j\phi_j, \phi_i \right)
	=
	\sum_{j=1}  a\left( \phi_i, \phi_j \right)u_j=b(\phi_i), \ \ (i=1,2)
	\label{eqn:VW2fem}
\end{equation}
の関係が得られる.ここで,式(\ref{eqn:VW2fem})を得るにあたり,
\begin{equation}
	a(u_j\phi_j,\phi_i)=u_ja(\phi_i,\phi_j), \ \ a(\phi_j,\phi_i)=a(\phi_i,\phi_j)
\end{equation}
とできることを利用した.式(\ref{eqn:VW2fem})を,$i=1,2$の場合をまとめ,行列とベクトルを
用いて整理すると,次のようになる.
\begin{equation}
	\left(
	\begin{array}{cc}
		a(\phi_1,\phi_1) & a(\phi_1,\phi_2)  \\
		a(\phi_2,\phi_1) & a(\phi_2,\phi_2)  
	\end{array}
	\right)
	\left\{
	\begin{array}{c}
		u_1 \\
		u_2
	\end{array}
	\right\}
	=
	\left\{
	\begin{array}{c}
		f_1 \\
		f_2 
	\end{array}
	\right\}
	+
	\left\{
	\begin{array}{c}
		p_1 \\
		p_2
	\end{array}
	\right\}
	\label{eqn:fem_n1}
\end{equation}
ただし,
\begin{equation}
	p_i:=\int_{0}^{l} p \phi_i dx, \ \ (i=1,2)
	\label{eqn:def_pi}
\end{equation}
とし,式(\ref{eqn:fem_n1})の右辺第1項を得るためには,式(\ref{eqn:phi_01})を用いた.
ここで,式(\ref{eqn:fem_n1})左辺の係数行列を
\begin{equation}
	\fat{k}:=
	\left(
	\begin{array}{cc}
		a(\phi_1,\phi_1) & a(\phi_1,\phi_2)  \\
		a(\phi_2,\phi_1) & a(\phi_2,\phi_2)  
	\end{array}
	\right)
	=
	\left(
	\begin{array}{cc}
		k_{11} & k_{12}  \\
		k_{21} & k_{22}  
	\end{array}
	\right)
	\label{eqn:kmat_e}
\end{equation}
と置き,$EA$が$0<x<l$において一定の場合について,行列$\fat{k}$の成分を計算すると,
\begin{equation}
	\fat{k}=
	\frac{EA}{l}
	\left(
	\begin{array}{cc}
		1 & -1  \\
		-1 & 1  
	\end{array}
	\right)
\end{equation}
となる.式(\ref{eqn:kmat_e})で与えられる行列$\fat{k}$は,要素剛性マトリクスと呼ばれる.
さらに,
\begin{eqnarray}
	\fat{u} &:= & \left( u_1,\,u_2\right)^T 
	\label{eqn:def_uvec}
	\\
	\fat{f} &:= & \left( f_1,\,f_2\right)^T 
	\label{eqn:def_fvec}
	\\
	\fat{p} &:= & \left( p_1,\,p_2\right)^T 
	\label{eqn:def_pvec}
\end{eqnarray}
とすれば,式(\ref{eqn:fem_n1})を
\begin{equation}
	\fat{k}\fat{u}=\fat{f}+\fat{p}
	\label{eqn:fe_eq0}
\end{equation}
と書くことができる.式(\ref{eqn:fe_eq0})すなわち式(\ref{eqn:fem_n1})は剛性方程式と呼ばれる.
%%%%%%%%%%%%%%%%%%%%%%%%%
%%%%%%%%%%%%%%%%%%%%%%%%%
\subsection{例題1}
\begin{enumerate}
\item
図\ref{fig:fig2}に示す軸力部材に生じる変位の近似解$\tilde u(x)$を,
剛性方程式(\ref{eqn:fem_n1})を用いて求める.
分布力は$p(x)=0$だから,$\fat{p}=(p_1,p_2)^T=\fat{0}$である.
また,節点1は固定壁であることから,節点1の変位は$u_1=0$である.
さらに,節点2では荷重が与えられていることから$f_2=F$と置くことができる.
以上より,剛性方程式は
\begin{equation}
	\frac{EA}{l}
	\left(
	\begin{array}{cc}
		1 & -1 \\
		-1 & 1
	\end{array}
	\right)
	\left\{
	\begin{array}{c}
		u_1=0 \\
		u_2
	\end{array}
	\right\}
	=
	\left\{
	\begin{array}{c}
		F_1\\
		F_2=F
	\end{array}
	\right\}
	+
	\left\{
	\begin{array}{c}
		p_1=0\\
		p_2=0
	\end{array}
	\right\}
\end{equation}
となる.この方程式を$u_2$と$F_1$について解けば,
\begin{equation}
	u_2=\frac{Fl}{EA}, \ \ F_1=-F
	\label{eqn:}
\end{equation}
が得られる.従って,求めるべき近似解は
\begin{equation}
	\tilde u(x)=u_1\phi_1(x)+u_2\phi_2(x)=\frac{Fx}{EA}
	\label{eqn:apprx1}
\end{equation}
と,この場合は厳密解に一致する結果が得られる.
\begin{figure}[h]
	\begin{center}
	\includegraphics[width=0.4\linewidth]{fig2.eps} 
	\end{center}
	\caption{部材右端部に既知の荷重$F$を受ける軸力部材.} 
	\label{fig:fig2}
\end{figure}
\item
図\ref{fig:fig3}に示すような,部材両端で軸方向の荷重が加えられた軸力部材を考える.
この場合剛性方程式は,$\fat{p}=\fat{0}$と$\fat{f}=(F_1,F_2)^T$から,
\begin{equation}
	\frac{EA}{l}
	\left(
	\begin{array}{cc}
		1 & -1 \\
		-1 & 1
	\end{array}
	\right)
	\left\{
	\begin{array}{c}
		u_1 \\
		u_2
	\end{array}
	\right\}
	=
	\left\{
	\begin{array}{c}
		F_1\\
		F_2
	\end{array}
	\right\}
	\label{eqn:fem_eq_ex2}
\end{equation}
となる.式(\ref{eqn:fem_eq_ex2})の剛性マトリクスは,逆行列を持たないことから,
この連立方程式の条件は互いに独立でない.そのため,$u_1$と$u_2$を一意に決定することができず,
\begin{equation}
	F_1=-F_2, 
\end{equation}
すなわち,力の釣り合い条件が成り立つことだけが結論される.
\begin{figure}[h]
	\begin{center}
	\includegraphics[width=0.4\linewidth]{fig3.eps} 
	\end{center}
	\caption{部材両端部で軸方向の力が加えられた軸力部材.} 
	\label{fig:fig3}
\end{figure}
\item
図\ref{fig:fig4}に示すような,軸方向に等分布荷重を受ける軸力問題を考える.
この場合,$p(x)=p_0$より,$\fat{p}$の成分を計算すると,
\begin{eqnarray}
	p_1 &= & \int_0^l p_0 \phi_1(x) dx = \frac{p_0l}{2} 
	\label{eqn:}
	\\
	p_2 &= & \int_0^l p_0 \phi_2(x) dx = \frac{p_0l}{2} 
	\label{eqn:}
\end{eqnarray}
となる.また,$u_1=0, f_2=0$だから,剛性方程式は
\begin{equation}
	\frac{EA}{l}
	\left(
	\begin{array}{cc}
		1 & -1 \\
		-1 & 1
	\end{array}
	\right)
	\left\{
	\begin{array}{c}
		0 \\
		u_2
	\end{array}
	\right\}
	=
	\left\{
	\begin{array}{c}
		f_1 \\
		0	
	\end{array}
	\right\}
	+
	\frac{p_0l}{2}
	\left\{
	\begin{array}{c}
		1 \\ 
		1	
	\end{array}
	\right\}
\end{equation}
と縮約され,
\begin{equation}
	u_2=\frac{p_0l^2}{2EA}, \ \ f_1=-p_0l
\end{equation}
が得られる.よって,変位の近似解は
\begin{equation}
	\tilde u(x)= \frac{p_0l^2}{2EA}\frac{x}{l}
\end{equation}
となる.厳密解は$x$の2次関数だから,この問題では近似解と厳密解は一致し得ない.
しかしながら,部材端部では,近似解と厳密解が一致することを確かめることができる.
\begin{figure}[h]
	\begin{center}
	\includegraphics[width=0.4\linewidth]{fig4.eps} 
	\end{center}
	\caption{等分布荷重を受ける軸力部材.} 
	\label{fig:fig4}
\end{figure}
\item
図\ref{fig:fig5}に示すような,部材中央の点に集中荷重を受ける軸力部材に生じる変位を求める.
外力$p(x)$は
\begin{equation}
	p(x)=F\delta \left( x-\frac{l}{2}\right)
\end{equation}
だから,デルタ関数の性質より,
\begin{equation}
	p_i=\int_0^l F \delta\left(x-\frac{l}{2}\right)\phi_i dx =F\phi_i\left(\frac{l}{2}\right)= \frac{F}{2}
	, \ \ (i=1,2)
	\label{eqn:}
\end{equation}
で,剛性方程式は
\begin{equation}
	\frac{EA}{l}
	\left(
	\begin{array}{cc}
		1 & -1 \\
		-1 & 1
	\end{array}
	\right)
	\left\{
	\begin{array}{c}
		0 \\
		u_2
	\end{array}
	\right\}
	=
	\left\{
	\begin{array}{c}
		f_1 \\
		0	
	\end{array}
	\right\}
	+
	\frac{F}{2}
	\left\{
	\begin{array}{c}
		1 \\
		1	
	\end{array}
	\right\}
\end{equation}
となる.これを$u_2$と$f_1$について解けば,
\begin{equation}
	u_2=\frac{1}{2}\frac{Fl}{EA}, \ \ f_1=-F
	\label{eqn:}
\end{equation}
が得られる.この問題の厳密な変位分布は,区分的1次関数(折れ線)で表される
ため,近似解と厳密解が完全に一致することは期待できないが,部材端部では
正確な結果が得られていることを示すことができる.
\begin{figure}[h]
	\begin{center}
	\includegraphics[width=0.4\linewidth]{fig5.eps} 
	\end{center}
	\caption{部材中央で集中荷重を受ける軸力部材.} 
	\label{fig:fig5}
\end{figure}
\end{enumerate}
\subsection{問題1}
図\ref{fig:fig8}に示すような,右半分の区間に等分布荷重を受ける
軸力部材に生じる変位を,式(\ref{eqn:fe_eq0})を用いて求めよ.
なお,部材の断面剛性$EA$は一定とする.
\begin{figure}[h]
	\begin{center}
	\includegraphics[width=0.4\linewidth]{fig8.eps} 
	\end{center}
	\caption{右半分の区間に,大きさ$p_0$の等分布荷重を受ける片端が固定された軸力部材.} 
	\label{fig:fig8}
\end{figure}
\subsection{2要素近似}
ここでは,近似精度を改善するために,軸力部材を図\ref{fig:fig6}に示すような
2つの要素に分割して近似解$\tilde u(x)$を構成する.
2つの要素への分割位置を指定するために,軸力部材の左端を節点1, 
右端を節点3, 部材内部の分割位置を節点2とし,
節点1と2の区間を要素1,節点2と3の区間を要素2と呼ぶことにする.
これらの節点1から3は,部材全体の中での節点配置を表すため,
その番号を全体節点番号と呼ぶ.
一方,特定の要素における力や変位,断面位置といった量を表現するには,
当該要素の左端と右端を参照するために,要素毎に設定した節点番号(要素節点番号)
を用いることが便利である.
そこで,図\ref{fig:fig6}の(b)と(c)示すように,要素節点1と2を定め,
要素$e$の要素節点$i$における軸変位を$u^e_i$,軸方向の力$f^e_i$と表す.
さらに,要素長を$l_e$,部材軸方向の座標を$x_e$,断面積を$A_e$等と,要素毎に変化しうる
量を表す際には,要素番号を付すことにする.このとき,要素$e(=1,2)$に対する剛性方程式は,
前節の議論から
\begin{equation}
	\frac{EA_e}{l_e}
	\left(
	\begin{array}{cc}
		1 & -1  \\
		-1 & 1 
	\end{array}
	\right)
	\left\{
	\begin{array}{c}
		u^e_1 \\
		u^e_2
	\end{array}
	\right\}
	=
	\left\{
	\begin{array}{c}
		f^e_1  \\
		f^e_2 
	\end{array}
	\right\}
	+
	\left\{
	\begin{array}{c}
		p^e_1 \\
		p^e_2 
	\end{array}
	\right\}, 
	\ \ 
	(e=1,2)
	\label{eqn:fe_eq_e}
\end{equation}
と書くことができる.
なお, 分布力$p(x)$に起因する節点荷重ベクトル$\fat{p}^e=(p^e_1,\, p^e_2)^T$の成分は,要素$e$上の内挿関数:
\begin{equation}
	\phi^e_1=1-\frac{x_e}{l_e}, \ \ 
	\phi^e_2=\frac{x_e}{l_e} 
\end{equation}
を用いて,次の式で与えられる.
\begin{equation}
	p^e_i:=\int_{x_e=0}^{l_e}p(x_e)\phi^e_i(x_e)dx_e, \ \ (i,e=1,2)
\end{equation}
\begin{figure}[h]
	\begin{center}
	\includegraphics[width=0.6\linewidth]{fig6.eps} 
	\end{center}
	\caption{軸力部材の2つの要素への分割.} 
	\label{fig:fig6}
\end{figure}
ここで,全体節点番号$I$の節点における変位を$u_I$,集中外力を$F_I$とすれば,
要素節点変位$u^e_i$と荷重$f^e_i$との関係は
\begin{equation}
	u^1_1=u_1, \ \ u^1_2=u^2_1=u_2, \ \ u^2_2=u_3
	\label{eqn:ui2uI}
\end{equation}
\begin{equation}
	f^1_1=F_1, \ \ f^1_2+f^2_1=F_2, \ \ f^2_2=F_3
	\label{eqn:fi2FI}
\end{equation}
となる.従って,$e=1$に対する剛性方程式(\ref{eqn:fe_eq_e})は
\begin{equation}
	\frac{EA_1}{l_1}
	\left(
	\begin{array}{ccc}
		1 & -1 & 0 \\
		-1 & 1  & 0 \\
		0 & 0  & 0 
	\end{array}
	\right)
	\left\{
	\begin{array}{c}
		u_1 \\
		u_2 \\
		u_3
	\end{array}
	\right\}
	=
	\left\{
	\begin{array}{c}
		f^1_1  \\
		f^1_2  \\
		0
	\end{array}
	\right\}
	+
	\left\{
	\begin{array}{c}
		p^1_1 \\
		p^1_2 \\
		0
	\end{array}
	\right\}, 
	\label{eqn:fe_eq_1}
\end{equation}
と書くことができる.同様に,要素$e=2$の剛性方程式は
\begin{equation}
	\frac{EA_2}{l_2}
	\left(
	\begin{array}{ccc}
		0 & 0  & 0 \\
		0 & 1 & -1 \\
		0 & -1 & 1  \\
	\end{array}
	\right)
	\left\{
	\begin{array}{c}
		u_1 \\
		u_2 \\
		u_3
	\end{array}
	\right\}
	=
	\left\{
	\begin{array}{c}
		0 \\
		f^2_1  \\
		f^2_2  
	\end{array}
	\right\}
	+
	\left\{
	\begin{array}{c}
		0 \\
		p^2_1 \\
		p^2_2
	\end{array}
	\right\}
	\label{eqn:fe_eq_2}
\end{equation}
と書くことができる.そこで,式(\ref{eqn:fe_eq_1})と(\ref{eqn:fe_eq_2})の
辺々を加え,式(\ref{eqn:fi2FI})を用いれば,
\begin{equation}
	%\frac{2EA}{l}
	\left(
	\begin{array}{ccc}
		\frac{EA_1}{l_1} & -\frac{EA_1}{l_1}  & 0 \\
		-\frac{EA_1}{l_1} & \frac{EA_1}{l_1}+\frac{EA_2}{l_2} & -\frac{EA_2}{l_2} \\
		0 & -\frac{EA_2}{l_2} & \frac{EA_2}{l_2}  \\
	\end{array}
	\right)
	\left\{
	\begin{array}{c}
		u_1 \\
		u_2 \\
		u_3
	\end{array}
	\right\}
	=
	\left\{
	\begin{array}{c}
		F_1 \\
		F_2  \\
		F_3  
	\end{array}
	\right\}
	+
	\left\{
	\begin{array}{c}
		p^1_1 \\
		p^1_2+p^2_1 \\
		p^2_2
	\end{array}
	\right\}
	\label{eqn:FE_eq}
\end{equation}
となる.これは,部材全体の接点変位と節点荷重の関係を表す式であることから,"全体剛性方程式"と呼ばれる.
%%%%%%%%%%%%%%%%%%%%%%%%%%%%
\subsection{例題2}
\begin{enumerate}
\item
図\ref{fig:fig4}に示す軸力部材の近似解を,式(\ref{eqn:FE_eq})を用いて求める.
2つの要素の長さ$l_1,l_2$を
\begin{equation}
	l_1=l_2=\frac{l}{2}
	\label{eqn:l1_l2}
\end{equation}
とすれば,
\begin{equation}
	p^1_1=\int_{x_1=0}^{l_1}p_0 \phi^1_1(x_1)dx_1=\frac{p_0l}{4}
	\label{eqn:}
\end{equation}
で,同様にして
\begin{equation}
	p^e_i=\frac{p_0l}{4}, \ \ (e=1,2,\, i=1,2)
	\label{eqn:}
\end{equation}
であることが示される.また,節点1は固定端,節点2と3に集中荷重は外部から加えられていないため,
\begin{equation}
	u_1=0, \ \ F_2=F_3=0
\end{equation}
とすることができる.よって,全体剛性方程式(\ref{eqn:FE_eq})は
\begin{equation}
	\frac{2EA}{l}
	\left(
	\begin{array}{ccc}
		1 & -1  & 0 \\
		-1 & 2 & -1 \\
		0 & -1 & 1  \\
	\end{array}
	\right)
	\left\{
	\begin{array}{c}
		0 \\
		u_2 \\
		u_3
	\end{array}
	\right\}
	=
	\left\{
	\begin{array}{c}
		F_1 \\
		0  \\
		0  
	\end{array}
	\right\}
	+
	\frac{p_0l}{4}
	\left\{
	\begin{array}{c}
		1 \\
		2 \\
		1	
	\end{array}
	\right\}
	\label{eqn:FE_eq_ex1}
\end{equation}
となる.そこで,式(\ref{eqn:FE_eq_ex1})を,未知の節点変位$u_2,u_3$に関する方程式に縮約して
\begin{equation}
	\frac{2EA}{l}
	\left(
	\begin{array}{cc}
		2 & -1  \\
		-1 & 1 
	\end{array}
	\right)
	\left\{
	\begin{array}{c}
		u_2 \\
		u_3
	\end{array}
	\right\}
	=
	\frac{p_0l}{4}
	\left\{
	\begin{array}{c}
		2 \\
		1 
	\end{array}
	\right\} 
	\label{eqn:}
\end{equation}
を解けば,
\begin{equation}
	u_2=\frac{3}{8}\frac{p_0l^2}{EA}, \ \ 
	u_3=\frac{1}{2}\frac{p_0l^2}{EA}
	\label{eqn:u_nd_ex1}
\end{equation}
が得られる.これを,式(\ref{eqn:FE_eq_ex1})に代入すれば,全体節点1における力
(反力が)$F_1=-p_0l$と求められる.
要素$e$における変位の近似解$\tilde u(x_e)$は,式(\ref{eqn:u_nd_ex1})を
\begin{equation}
	\tilde u(x_e) =\sum_{i=1}^2 u^e_i\phi^e_i(x_e), \ \ (0\leq x_e \leq l_e)
\end{equation}
に代入することで得られる.ここでは,変位分布を要素毎に1次式で
近似しているため,2次式で与えられる厳密解には一致し得ない.
しかしながら,部材全体の変位分布を一つの直線で
近似よりも,解の精度は向上している.
\item
図\ref{fig:fig5}に示す軸力部材に生じる変位分布を求める.
ここでも,式(\ref{eqn:l1_l2})のように,長さの等しい2つの要素に
部材を要素分割すれば,
\begin{equation}
	u_1=0, \ \ F_2=F,\ \ F_3=0, \ \ p(x)\equiv 0
\end{equation}
より,剛性方程式(\ref{eqn:FE_eq})は
\begin{equation}
	\frac{2EA}{l}
	\left(
	\begin{array}{ccc}
		1 & -1  & 0 \\
		-1 & 2 & -1 \\
		0 & -1 & 1  \\
	\end{array}
	\right)
	\left\{
	\begin{array}{c}
		0 \\
		u_2 \\
		u_3
	\end{array}
	\right\}
	=
	F
	\left\{
	\begin{array}{c}
		0 \\
		1  \\
		0  
	\end{array}
	\right\}
	+
	\left\{
	\begin{array}{c}
		F_1 \\
		0  \\
		0  
	\end{array}
	\right\}
	\label{eqn:FE_eq_ex2}
\end{equation}
となる.これを縮約して得られる
\begin{equation}
	\frac{2EA}{l}
	\left(
	\begin{array}{cc}
		2 & -1  \\
		-1 & 1 
	\end{array}
	\right)
	\left\{
	\begin{array}{c}
		u_2 \\
		u_3
	\end{array}
	\right\}
	=
	F
	\left\{
	\begin{array}{c}
		1 \\
		0 
	\end{array}
	\right\} 
	\label{eqn:}
\end{equation}
を解いて未知の節点変位と節点荷重を求めると,
\begin{equation}
	u_2=u_3=\frac{Fl}{2EA}, \ \ F_1=-F
	\label{eqn:sol_ex2}
\end{equation}
と,厳密解に一致する結果が得られる.
\end{enumerate}
\subsection{問題2}
図\ref{fig:fig7}に示すような,両端が固定された2つの軸力部材について,式(\ref{eqn:FE_eq})を用いて変位の近似解を
求めよ.なお,部材の断面剛性$EA$は一定とし,要素1と要素2の長さは$l/2$となるように要素分割が行われているものとする.
\begin{figure}[h]
	\begin{center}
	\includegraphics[width=0.8\linewidth]{fig7.eps} 
	\end{center}
	\caption{両端が固定された荷重条件の異なる2つの軸力部材.} 
	\label{fig:fig7}
\end{figure}
\end{document}
